\begin{abstract}

We propose a novel segmentation approach based on deep convolutional encoder
networks and apply it to the segmentation of multiple sclerosis (MS) lesions in
magnetic resonance images. Our model is a neural network that has both
convolutional and deconvolutional layers, and combines feature extraction and
segmentation prediction in a single model. The joint training of the feature
extraction and prediction layers allows the model to automatically learn
features that are optimized for accuracy for any given combination of image
types. In contrast to existing automatic feature learning approaches, which are
typically patch-based, our model learns features from entire images, which
eliminates patch selection and redundant calculations at the overlap of
neighboring patches and thereby speeds up the training. Our network also uses a
novel objective function that works well for segmenting underrepresented
classes, such as MS lesions. We have evaluated our method on the publicly
available labeled cases from the MS lesion segmentation challenge 2008 data set,
showing that our method performs comparably to the state-of-the-art. In
addition, we have evaluated our method on the images of 500 subjects from an MS
clinical trial and varied the number of training samples from 5 to 250 to show
that the segmentation performance can be greatly improved by having a
representative data set.

\begin{keywords}
Segmentation, multiple sclerosis lesions, MRI, machine learning, unbalanced
classification, deep learning, convolutional neural nets
\end{keywords}

\end{abstract}
