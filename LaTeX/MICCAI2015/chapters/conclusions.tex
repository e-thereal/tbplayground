\section{Conclusions}

\begin{itemize}
\item Future work: use more layers to achieve a hierarchical segmentation
method, but this paper, focus on the simplest possible network to evaluate the
potential of such an approach.
\item Used a simple model to reduce the risk of overfitting. In the future, we
are planning to add more layers and use more filters and apply the model to
larger data sets.
\item We have demonstrated the potential of our approach for MS lesion
segmentation, although the method is not inherently limited to this kind of
segmentation. We are planning to apply this framework to other segmentation
problems and we anticipate that other groups will adopt this approach to a
variety of segmentation problems.
\end{itemize}

% We have introduced a new method for modeling the variability in brain morphology
% and lesion distribution of a large set of MRIs of MS patients. Our method is a
% statistical model of brain images composed of three DBNs: one for morphology,
% one for lesion distribution, and one that jointly models both. We have
% demonstrated that such a model, which requires no built-in priors on image
% similarity, can automatically discover patterns of variability that can be
% parameterized in a low-dimensional space and are clinically relevant. In
% addition, our model can generate sample images from model parameters for
% visualization. For future work, we plan to incorporate clinical parameters into
% the learning stage using discriminative fine-tuning, which we expect would
% result in further improvements in the clinical score correlations. In addition,
% we would like to extend our approach to model longitudinal patterns of
% variability with the goal of predicting future clinical status from current
% images.
