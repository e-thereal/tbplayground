\section{Methods}
\label{sec:method}

\paragraph{Problem definition}
\begin{itemize}

\item Segmentation is a function $s$ that maps an image $I$ to its segmentation
$S$. That is $S = s(I)$.

\item Given a set of training images $I_i$ and corresponding segmentations
$S_i$, we treat finding an appropriate function for segmenting T2 lesions as an
optimization problem of the following form:
 \begin{equation} 
\hat{s} = \arg \max_{s \in \mathcal{S}} \sum_i \text{sim}(S_i, s(I_i)).
\label{eq:segprob}
\end{equation}
where $\text{sim}$ denotes a function that calculates the similarity between
ground truth segmentations and predicted segmentations, and $\mathcal{S}$ is the
set of possible functions.
\end{itemize}

\paragraph{Class of Segmentation Functions}
\begin{itemize}
\item We propose to use neural networks to model the class of allowed
segmentation functions.
\item Neural networks are very flexible and able to learn highly non-linear
functions, which makes them suitable for image segmentation.
\item Can be efficiently learned from training data using gradient descent.
\item Architecture of the neural network can be used to regularize the functions
that can be learned
\item We propose a combined convolutional-deconvolutional neural network with
one convolutional and one deconvolutional layer.
\item Convolutional neural network because it scales better to high-resolution
images and is better regularized than dense networks
\item Filter size can regular local support.
\item Future work: use more layers to achieve a hierarchical segmentation
method, but this paper, focus on the simplest possible network to evaluate the
potential of such an approach.
\item Better to understand a simple model first.
\item Inference in the convolutional and deconvolutional layer
\begin{align}
\vect{h}^j &=
\max\Big(0, \sum_{i}(\tilde{\vect{W}}^{ij}*_\text{v}\vect{x}^i) +
b_j\Big)
\\
\vect{y} &= \sigm\Big(\sum_{j}(\vect{W}^{j}*_\text{f}\vect{h}^j) +
b\Big)
\end{align}
\item Figure of pre-training and fine-tuning or just fine-tuning.
\item Relationship to general segmentation problem
\begin{align}
\vect{x} &= I \\
\vect{y} &= S 
\end{align}
\end{itemize}

\paragraph{Similarity Measure}
\begin{itemize}
\item Neural networks are traditionally trained with SSD
\begin{equation} 
\text{SSD} = \sum_j (d_j - y_j)^2 
\end{equation}
\item SSD is problematic for unbalanced classification tasks as the learning will
greatly favor one class.
\item
A way around that would be to calculate the SSD for
each class individually and then combine the two scores similar to the
calculation of the balanced accuracy.
\begin{equation} 
\text{SSD}_\text{balanced} = \frac{\textstyle\sum_j (d_j - y_j)^2 d_j}{%
\textstyle\sum_j d_j} + \frac{\textstyle\sum_j (d_j - y_j)^2 (1 - d_j)}{%
\textstyle\sum_j (1 - d_j)}
\end{equation}
The first term measures the sensitivity and the second term measures the
specificity.
\item Due to the large number of negative samples in the case of lesion
classification, the sensitivity and specificity and not equally important for a
good classification result. 
\item It is therefore advantageous to weight the
specificity higher than the sensitivity. To allow that, we introduce the
sensitivity ratio $r_\text{sen}$, which yields to the following objective
function:
\begin{equation} 
\text{SSD}_\text{balanced} = r_\text{sen}\frac{\textstyle\sum_j (d_j - y_j)^2
d_j}{\textstyle\sum_j d_j} 
+ (1-r_\text{sen})\frac{\textstyle\sum_j (d_j - y_j)^2 (1 - d_j)}{%
\textstyle\sum_j (1 - d_j)}
\end{equation}
\item Derive gradients
\end{itemize}

\paragraph{Prevent Overfitting}
\begin{itemize}
\item Two main sources: 1) bias terms highly tuned to lesion locations observed
in the training data, and 2) filters tuned to the intensity range observed in
the training data
\item Use shared bias terms and add lesion prior calculated from a large data
set plus smoothing
\item Use data augmentation, i.e. during training, randomly change the
brightness contrast and gamma correction of the training images to artificially
increase the intensity variability in the training set
\end{itemize}

\paragraph{Training Pipeline}
\begin{itemize}
\item Downsample training images and training segmentations from
\SI{0.5x0.5x0.5}{\milli\meter} to \SI{1x1x1}{\milli\meter} voxel resolution.
\item Perform brain extraction
\item Crop to smallest ROI
\item Pad all images to standard size
\item[$\Rightarrow$] Calculate combined cropping parameters
\item Perform training of the NN on the downsampled training set
\item Calculate probability maps for the entire downsampled training set
\item Upsample the probability maps to the native resolution
\item Crop lesion masks in native resolution to fit the same ROI as the
upsampled probability masks
\item Choose the threshold that maximizes the DSC in the training set 
\end{itemize}

\paragraph{Testing Pipeline}
\begin{itemize}
\item Downsample test image
\item Perform brain extraction
\item Apply combined cropping parameters
\item Infer probability using the downsampled images
\item Upsample the probability map to the native resolution
\item Apply threshold
\item Crop lesion masks to combined cropping parameters in native resolution
space
\item Compare segmentation
\end{itemize}









