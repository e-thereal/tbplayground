\section{Introduction}

\paragraph{Motivation}
\begin{itemize}
  \item MS is a neuro-degenerative disease of the central nervous system
  characterized by the formation of lesions
  \item Accurate, reproducible and automatic segmentation of
  lesions is important to assess disease progression, treatment affect, lesion
  LL an important end-point of clinical trials
\end{itemize}

\paragraph{Related Work}

\begin{itemize}

\item Lesion segmentation is treated as a voxel classification problem, where
algorithms mostly differ in the choice of features and the type of
classification algorithm.

\item The classification problem itself can then be solved either in a
supervised way using, e.g., artificial neural networks \cite{zijdenbos1994} or
random forests \cite{geremia2010}, or unsupervised using clustering methods with
one outlier class \cite{souplet2008} or by treating lesions as an outlier of a
generative model \cite{weiss2013}.

\item Trend from simple intensity based features to more complex features
extracted or learned from image patches.
  
\item Early approaches use the intensity values of different modalities at a
particular voxel as the input features \cite{zijdenbos1994}.

\item However, simple intensity features can be sensitive to intensity
variations between images. Carefully chosen context-rich features are more
robust \cite{geremia2010} to intensity variations.

\item Youngjin et. al proposed to learn domain specific features
from image patches from an unlabelled data set using unsupervised feature
learning \cite{yoo2014}.

\item In the context of the segmentation of cell membranes, Ciresan et. al
proposed to perform classification of image patches directly using a
convolutional neural network without a dedicated feature extraction step
\cite{Ciresan2012}. Features are learned indirectly within the lower layers of
the neural network during training, while the higher layers can be regarded as
performing the classification.

\item Advantage, allows the fine-tuning of features that are useful for the
classification task.

\item The time required to train complex patch-based feature extraction methods
can make the approach infeasible when the size and the number of patches is
large. \cite{Ciresan2012} reported a training time of more than a week to train
their patch-based segmentation model using 4 GPUs using 2D images with a
resolution of \num{512x512} pixels. To scale patch-based classification to 3D
images with a resolution of \num{256x256x50}, Youngjin et.
al used only a small fraction (\SI{0.1}{\percent}) of the possible patches for
training, which might lead to suboptimal learning due to introducing a bias of
the learned model towards the selected patches.

\end{itemize}

\paragraph{Proposed Method}

\begin{itemize}

% Main contribution first
\item In this paper, we propose a novel method for segmenting MS lesions that
outperforms the state-of-the-art on the MICCAI 2008 lesion segmentation
challenge data set, the most widely used publicly available clinical data set
for comparing lesion segmentation methods. (Both is to the best of our
knowledge). Might rephrase to comparable to the state of the art on this data
set depending on how my latest cross-validation experiment goes.

% What it is
\item Our network is a combination of a convolutional and a deconvolutional
neural network. The first layer is a convolutional layer \cite{LeCun1998} that
extracts features from multi-modal MRIs at each voxel location. The second layer
is a deconvolutional layer \cite{zeiler2011} that uses the features extracted by
the first layer to classify each voxel of the image in a single operation. Given
a training set consisting of MRIs and lesion masks, the parameters of the model
can be learned using stochastic gradient descent.

\item Our network is similar in architecture to convolutional auto-encoder
networks but instead of predicting the input itself, our network will be trained
on images to predict lesions. Due to the similarity to auto-encoder networks, we
will call our network architecture a convolutional encoder network (CEN).

% Why it is great
\item A major advantage compared to patch-based approaches is that our approach
does not require the selection of patches because it uses all voxels of an
image.

\item Our model is fast because we can segment an entire image in a single
feed-forward pass through the network. Allows to evaluate the segmentation
performance at training time. Allows direct maximization of the similarity
between predicted and ground truth segmentation during the training stage.

\item Combined feature learning and classification like the membrane paper.
Learns features that are tuned for the classification task, instead of using a
set of general features. But scales much better to high-resolution 3D volumes.

% Challenge
\item Traditionally, NNs are trained using sum of squared differences. This
metric is not suitable for the segmentation of MS lesions, because MS lesion
segmentation is a highly unbalanced classification problem and a neural network
trained with SSD would greatly favour one class.

% How we overcome the challenge
\item The second contribution is our new proposed objective function that allows
NN to be applied to the voxel-wise classification in the case of very unbalanced
classification problems. We propose to use a weighted sum of sensitivity and
specificity error to better balance. We will show how convolutional neural
networks can be trained using the modified objective function.

\end{itemize}
