\section{Introduction}

\paragraph{Motivation}
\begin{itemize}
  \item MS is a neuro-degenerative disease of the central nervous system
  characterized by the formation of lesions
  \item Accurate, reproducible and automatic segmentation of
  lesions is important to assess disease progression, treatment affect, lesion
  LL an important end-point of clinical trials
\end{itemize}

\paragraph{Related Work}

\begin{itemize}
  
\item As a voxel-classification problem: early approaches. Using neural
networks, unsupervised clustering and MS lesion as an outlier models.
Sensitive to different contrasts.

\item Patch-based with designed features, classify the center of a patch. Use
more contextual information. More robust to changes in illumination than
voxel-based approaches. Two phase approach: 1) extract features, 2) classify
based on extracted features from a patch. E.g., design features and use a random
forest.

\item Patch-based with learned features. Membrane segmentation: perform
classification in a single model. Combines feature learning phase and
classification phase. Advantage, allows the fine-tuning of features to drive the
classification during the training stage of the classifier.

\item Common disadvantages of patch-based approaches: Too slow to use every
possible patch during training. Requires careful selection of patches. Indirect
optimization of final metric depends on selected patches. Result on the entire
image can be much different from the results during training depending on the
selected patches. Use only a small fraction of the available data.

\end{itemize}

\paragraph{Proposed Method}

\begin{itemize}
\item In this paper, we propose to use a neural network architectural that directly
segments the entire image at once, without requiring the selection of patches. 

\item Uses all available image data during training. 

\item Allows to evaluate the segmentation performance at training time. Allows
direct optimization of the similarity metric.

\item Traditionally, NNs are trained using sum of squared differences. This
metric is not suitable for the segmentation of MS lesions, because MS lesion
segmentation is a highly unbalanced classification problem and a neural network
trained with SSD would greatly favour one class.

\item We propose to use a weighted sum of sensitivity and specificity error to
better balance. We will show how convolutional neural networks can be trained
using the modified objective function.

\item We evaluated our method on a publicly available data set from the MS
lesion segmentation challenge. We will show that our method can achieve state of
the art segmentation performance on this data set.
\end{itemize}
