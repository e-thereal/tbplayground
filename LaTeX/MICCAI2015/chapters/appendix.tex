\appendix

\section{Appending}

Myelin imaging is a form of quantitative magnetic resonance
imaging (MRI) that measures myelin content and can potentially allow
demyelinating diseases such as multiple sclerosis (MS) to be detected
much earlier. Although focal lesions are the most visible sign of MS
pathology on conventional MRI, it has been shown that even tissues
that appear normal may exhibit decreased myelin content as revealed
by myelin-specific images (myelin maps). Current methods for analyzing
myelin maps typically use global or regional mean myelin measurements
to detect abnormalities, but ignore any finer spatial patterns that may be
characteristic of MS. In this paper, we present a method to automatically
learn, from multimodal images, spatial features that can potentially im-
prove the detection of MS pathology. More specifically, 3D image patches
of normal-appearing tissues in T1-weighted (T1w) MRIs are extracted
along with the corresponding patches in myelin maps, and used to learn
a joint latent T1w-myelin feature representation, via deep learning. To
account for missing features due to the presence of focal MS lesions in
some image patches, a method for unbiased missing feature imputation
is applied. The resulting feature vectors are then used to train a random
forest classifier. Using the T1w and myelin images of 55 MS patients
and 44 healthy controls in an 11-fold cross-validation experiment, the
proposed joint T1w-myelin feature-based method achieved an average
classification accuracy of 78.69\%, which is higher than those attained by
global or regional mean myelin measurements, or deeply learned features
in either the T1w or myelin images alone, suggesting that the proposed
method has strong potential for improving the accuracy of MS diagnosis.


\begin{comment} 
\begin{table}[tb]
\centering
\sisetup{
round-precision = 1,
}%
\def\Tabspace{7pt}
\def\tabspace{3pt}
\caption{This table is here just for reference. I'm not planning to keep it,
but it shows the full picture of the MICCAI tests.}
\begin{tabular}{l%
% @{\hspace{10pt}}S[table-format=2.1]@{\hspace{\tabspace}}%
% @{\hspace{\tabspace}}S[table-format=2.1]@{\hspace{\Tabspace}}%
% @{\hspace{\Tabspace}}S[table-format=2.1]@{\hspace{\tabspace}}%
% @{\hspace{\tabspace}}S[table-format=2.1]@{\hspace{\Tabspace}}%
% @{\hspace{\Tabspace}}S[table-format=2.1]@{\hspace{\tabspace}}%
% @{\hspace{\tabspace}}S[table-format=2.1]@{\hspace{\tabspace}}%
% @{\hspace{\tabspace}}S[table-format=2.1]@{\hspace{\Tabspace}}%
% @{\hspace{\Tabspace}}S[table-format=2.1]@{\hspace{\tabspace}}%
% @{\hspace{\tabspace}}S[table-format=2.1]@{\hspace{\tabspace}}%
% @{\hspace{\tabspace}}S[table-format=2.1]%
*{10}{S[table-format=2.1]}%
}

\toprule
Patient & \multicolumn{2}{@{\hspace{10pt}}c@{\hspace{10pt}}}{Souplet} &
\multicolumn{2}{c}{Geremia} &
\multicolumn{3}{c}{Weiss} &
\multicolumn{3}{c}{Our method} \\
\addlinespace
 & {TPR} & {PPV} & {TPR} & {PPV} & {TPR} & {PPV} & {DSC} & {TPR} & {PPV} & {DSC}
 \\
\midrule
CHB01 & 22 & 41 & 49 & 64 & 60 & 58 & 59 & 50 & 69 & 58 \\
CHB02 & 18 & 29 & 44 & 63 & 27 & 45 & 34 & 42 & 52 & 46 \\
CHB03 & 17 & 21 & 22 & 57 & 24 & 56 & 34 & 38 & 70 & 49 \\
CHB04 & 12 & 55 & 31 & 78 & 27 & 66 & 38 & 60 & 63 & 61 \\
CHB05 & 22 & 42 & 40 & 52 & 29 & 33 & 31 & 42 & 43 & 42 \\
CHB06 & 13 & 46 & 32 & 52 & 10 & 36 & 16 & 24 & 63 & 35 \\
CHB07 & 13 & 39 & 40 & 54 & 14 & 48 & 22 & 57 & 65 & 61 \\
CHB08 & 13 & 55 & 46 & 65 & 21 & 73 & 32 & 47 & 75 & 58 \\
CHB09 & 3 & 18 & 23 & 28 & 5 & 22 & 8 & 22 & 49 & 30 \\
CHB10 & 5 & 18 & 23 & 39 & 15 & 12 & 13 & 11 & 64 & 19 \\
\addlinespace
UNC01 & 1 & 1 & 2 & 1 & 33 & 29 & 31 & 3 & 6 & 4 \\
UNC02 & 37 & 39 & 48 & 36 & 54 & 51 & 53 & 54 & 38 & 44 \\
UNC03 & 12 & 16 & 24 & 35 & 64 & 27 & 38 & 62 & 28 & 39 \\
UNC04 & 38 & 54 & 54 & 38 & 40 & 51 & 45 & 59 & 35 & 44 \\
UNC05 & 38 & 8 & 56 & 19 & 25 & 10 & 16 & 10 & 5 & 6 \\
UNC06 & 57 & 9 & 15 & 8 & 13 & 55 & 20 & 24 & 49 & 32 \\
UNC07 & 27 & 18 & 76 & 16 & 44 & 23 & 30 & 33 & 19 & 24 \\
UNC08 & 27 & 20 & 52 & 32 & 43 & 13 & 20 & 51 & 13 & 20 \\
UNC09 & 16 & 43 & 67 & 36 & 69 & 6 & 11 & 51 & 5 & 9 \\
UNC10 & 22 & 28 & 53 & 34 & 43 & 23 & 30 & 56 & 19 & 28 \\
\addlinespace
Average & 20.65 & 30.00 & 39.85 & 40.35 & 33.00 & 36.85 & 29.05 & 39.71 & 41.38
& 35.52
\\
\bottomrule
\end{tabular}
\end{table}
\end{comment}

\begin{comment} 

\begin{table}[tb]
\sisetup{
  round-precision = 2,
}% 
\centering
\caption{Comparison of segmentation performance on the training and test set
for varying number of training samples. The difference between training and
test performance is reduces for increasing number of training samples. A
training set size of 150 is sufficient to prevent overfitting.}
\label{tab:bioms}
\begin{tabular} {c*{6}{S[table-format=3.3]}}
\toprule
Number of & \multicolumn{3}{c}{Training set} &
\multicolumn{3}{c}{Test set}
\\
training samples & {TPR} & {PPV} & {DSC} & {TPR} & {PPV} & {DSC} \\
\midrule 
5 & 77.9679 & 66.2261 & 70.9317 & 47.1575 & 57.0428 & 48.5493 \\
10 & 73.4045 & 68.1603 & 69.7124 & 55.0416 & 59.8499 & 53.9332 \\
15 & 71.765 & 68.234 & 68.8614 & 56.2786 & 60.883 & 55.4736 \\
25 & 69.3865 & 70.6347 & 68.9442 & 57.4285 & 61.0626 & 56.2912 \\
250 & 64.5539 & 58.2538 & 58.5023 & 65.4692 & 56.813 & 57.5434 \\
\bottomrule
\end{tabular}
\end{table}
\end{comment}