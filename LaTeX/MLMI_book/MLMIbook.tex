%%This is a very basic article template.
%%There is just one section and two subsections.
\documentclass{report}

\usepackage{ifthen}

% Charater sets, fonts and basics
\usepackage{fixltx2e}
\usepackage[latin1]{inputenc}
\usepackage[T1]{fontenc}

\usepackage[english]{babel}
\usepackage[english=british]{csquotes}

%\usepackage{lmodern}
\usepackage[sc, osf]{mathpazo}
\renewcommand*\sfdefault{uop}
\usepackage[scaled=.9]{luximono}
\linespread{1.10}
\usepackage{microtype}
\makeatletter
\def\MT@register@subst@font{\MT@exp@one@n\MT@in@clist\font@name\MT@font@list
   \ifMT@inlist@\else\xdef\MT@font@list{\MT@font@list\font@name,}\fi}
\makeatother 

% Math stuff
%\usepackage{dsfont,amsmath,amssymb,latexsym,soul}
\usepackage{amsmath,mathtools}
\usepackage{amsbsy,dsfont,soul,calc}
\DeclareMathOperator{\sinc}{sinc}
\DeclareMathOperator{\sigm}{sigm}
\DeclareMathOperator{\fft}{fft}
\DeclareMathOperator{\ifft}{ifft}
\sodef\so{}{.14em}{.4em plus.1em minus .1em}{.4em plus.1em minus .1em}

\usepackage[autokw]{svn-multi}
\svnid{$Id: preamble.tex 41 2012-08-01 17:29:41Z tbrosch $}

% Tikz
\usepackage{atbegshi}
\usepackage{tikz}
\usetikzlibrary{intersections,decorations.pathreplacing,fit,calc,positioning,shapes.geometric,matrix}
\tikzstyle{every node}+=[font=\sffamily\small]
\pgfdeclarelayer{background}
\pgfsetlayers{background,main}

%\usetikzlibrary{external}
%\tikzexternalize[prefix=tikzfigures/temp/] % activate
%\tikzset{external/export=false}
\newcommand{\tikzexternal}[2][false]{%
%\tikzset{external/export=true, external/remake next=#1}%
%\tikzsetnextfilename{#2}%
}

% Special Table hacks
\usepackage{array,booktabs,dcolumn,rotating,multirow,etoolbox}

\robustify\bfseries % requires etoolbox
\newcommand{\minitab}[2][l]{\begin{tabular}{#1}#2\end{tabular}}

\newcolumntype{d}[1]{%
>{\DC@{.}{.}{#1}}l<{\DC@end}%
}

\newcolumntype{.}{D{.}{.}{-1}}
\newcolumntype{L}[1]{>{\raggedright\arraybackslash}p{#1}}

%\usepackage[sf,SF]{subfigure}
\usepackage[font={sf,small},labelfont=md,subrefformat=parens]{subfig}

% Units
\usepackage{xfrac}
\usepackage[alsoload=binary]{siunitx}
\sisetup{
  seperr,
  %repeatunits=false,
  product-units = single,
  trapambigerr=false,
  trapambigrange=false,
  %tophrase={{ bis }},
  valuemode=math,
  unitmode=math,
  quotient-mode=fraction,
  fraction-function=\sfrac
}

\usepackage{tocstyle}
\newtocstyle[KOMAlike][leaders]{thesis}{
  %\settocfeature[-1]{dothook}{\bfseries}
  %\settocfeature[0]{dothook}{\bfseries}
}
\usetocstyle{thesis}

% Enumerates
\usepackage{paralist}

\usepackage{varioref}
\usepackage[
    pdftex,
	pdfpagemode=UseOutlines,
	pdfdisplaydoctitle=true,
	pdflang=en,
	pagebackref,
	bookmarksopen=true,
	bookmarksopenlevel={1}
]{hyperref}

% % Change Layout of Backref
\renewcommand*{\backref}[1]{%
  % default interface
  % #1: backref list
  %
  % We want to use the alternative interface,
  % therefore the definition is empty here.
}%
\renewcommand*{\backrefalt}[4]{%
  % alternative interface
  % #1: number of distinct back references
  % #2: backref list with distinct entries
  % #3: number of back references including duplicates
  % #4: backref list including duplicates
  \ifnum#1=0%
  \footnotesize\sffamily\color{red!50!gray}(not cited)%
  \else%
  \ifnum#1=1%
  \footnotesize\sffamily\color{gray}(cited on page~#2)%
  \else%
  \footnotesize\sffamily\color{gray}(cited on pages~#2)%
  \fi%
  \fi%
}

% Extraabstand fuer Fussnoten
\setlength{\skip\footins}{2em}

% Handle floats right
%\usepackage[above,below]{placeins}
%\usepackage{placeins}

\usepackage[version=3]{mhchem}

%\bibliographystyle{plain}
\usepackage{bibgerm}
\usepackage[sort]{natbib}
\bibliographystyle{gerapali}
%\bibliographystyle{gerplain}

% Listings
\usepackage{listings}

% Color definition
% Define special colors
\definecolor{listcomment}{rgb}{0.325,0.761,0.259}
\definecolor{listkeyword}{rgb}{0.243,0.306,0.408}
\definecolor{listnumbers}{gray}{0.65}
\definecolor{listlightgray}{gray}{0.955}
\definecolor{shadecolor}{named}{listlightgray}
\definecolor{stringcolor}{rgb}{0.341,0.243,0.416}

% Setup the listings environment
\lstset{
  frame = tb,
  framerule = 0.25pt,
  float,
  backgroundcolor={\color{listlightgray}},
  basicstyle = {\ttfamily\footnotesize},
  keywordstyle = {\ttfamily\color{listkeyword}\textbf},
  identifierstyle = {\ttfamily},
  commentstyle = {\ttfamily\color{listcomment}\textit},
  stringstyle = {\ttfamily\color{stringcolor}},
  showstringspaces = false,
  showtabs = false,
  numbers = left,
  numbersep = 6pt,
  numberstyle={\ttfamily\color{listnumbers}},
  tabsize = 2,
  language=C++,
  floatplacement=tb
  captionpos=b,
  escapeinside={<@}{@>},
  abovecaptionskip=7pt,
  morekeywords={__global__, __host__, __device__, uint, texture, float4, fmatrix4}
}

\usepackage[vlined,linesnumbered]{algorithm2e}
\SetNlSty{}{}{}
\SetArgSty{}
\DontPrintSemicolon

\usepackage[acronym,nonumberlist,toc,nomain]{glossaries}
\renewcommand*{\glspostdescription}{}
\setlength{\glsdescwidth}{0.7\textwidth}
\glossarystyle{super}
\renewcommand*{\glsgroupskip}{}

\AtBeginDocument{\labelformat{lstlisting}{Listing #1}}
\AtBeginDocument{\labelformat{chapter}{Chap\mbox{ter #1}}}
\AtBeginDocument{\labelformat{section}{Sec\mbox{tion #1}}}
\AtBeginDocument{\labelformat{subsection}{Sec\mbox{tion #1}}}
\AtBeginDocument{\labelformat{equation}{(#1)}}
\AtBeginDocument{\labelformat{table}{T\mbox{able #1}}}
\AtBeginDocument{\labelformat{figure}{F\mbox{igure #1}}}
\AtBeginDocument{\labelformat{Definition}{D\mbox{efinition #1}}}
%\AtBeginDocument{\labelformat{Satz}{Satz~#1}}
%\AtBeginDocument{\labelformat{Beispiel}{Beispiel~#1}}
%\AtBeginDocument{\labelformat{Beweis}{Beweis~#1}}

% Headings and footer
\usepackage{scrpage2}
\pagestyle{scrheadings}
\clearscrheadfoot
\ohead{\headmark}
\ofoot[\pagemark]{\pagemark}

\setheadsepline{.4pt}
\addtokomafont{pageheadfoot}{\normalfont\sffamily\small}
\addtokomafont{caption}{\sffamily\small}
\addtokomafont{captionlabel}{\sffamily\bfseries\small}
\addtokomafont{pagefoot}{\normalsize}
%\addtokomafont{chapter}{\glsresetall}
%\addtokomafont{page}{\normalsize}
%\automark[subsection]{section}
\automark{chapter}

%\addtokomafont{section}{
%{\tikzset{external/export=false}\tikz \fill[left color=primary-4,right color=white] (0,0) rectangle (\textwidth,1pt);}\\*}
\addtokomafont{dictumtext}{\normalfont\itshape}
\addtokomafont{dictumauthor}{\normalfont}

%\newcommand{\intro}[1]{\paragraph{Introduction}{\itshape #1}}
\newcommand{\intro}[1]{\noindent {\itshape #1}}
\newcommand{\question}[1]{\paragraph{Question}{\itshape #1}}
\newcommand{\answer}{\paragraph{Answer}}

\newcommand{\qa}[1]{\question{#1}\answer}


\usepackage[textwidth=3.3cm, textsize=footnotesize,
colorinlistoftodos, shadow]{todonotes}
\usepackage{textcomp,xspace}
\usepackage{dsfont,amsmath,amssymb,amsbsy}
\usepackage{setspace,comment}

\newcommand{\vect}[1]{\mathbf{#1}}
\newcommand{\R}{\ensuremath{\mathds{R}}}
\newcommand{\Q}{\ensuremath{\mathds{Q}}}
\newcommand{\N}{\ensuremath{\mathds{N}}}
\newcommand{\I}{\ensuremath{\mathds{I}}}
\newcommand{\E}{\ensuremath{\mathds{E}}}
\newcommand{\Z}{\ensuremath{\mathds{Z}}}
\newcommand{\D}{\ensuremath{\mathds{D}}}
\newcommand{\W}{\ensuremath{\mathds{W}}}
\newcommand{\B}{\ensuremath{\mathds{B}}}
\newcommand{\Sn}{\ensuremath{\mathds{S}}}

\newcommand{\thetas}{\boldsymbol{\theta}}
\newcommand{\deltas}{\boldsymbol{\delta}}
\newcommand{\data}{\mathcal{D}}
\newcommand{\Norm}{\ensuremath{\mathcal{N}}}
\newcommand{\given}{\mid}

% Define a counter for the inserted todonotes.
\newcounter{todoListItems}
\newcommand{\todoTrans}[2][ ]{%
	% Increment counter
	\addtocounter{todoListItems}{1}%
	\todo[%
		caption={\protect\hypertarget{todo\thetodoListItems}{}#2},
		#1]
	{
		\sffamily #2 \hfill
		\hyperlink{todo\thetodoListItems}{$\uparrow$}
	}
}

\newcommand{\addref}{\todoTrans[color=blue!40]{Needs reference.}\xspace}
\newcommand{\addcontent}[1]{\todoTrans[color=green!40, inline]{#1}\xspace}
\newcommand{\rewrite}[1]{\todoTrans[color=red!40]{#1}\xspace}
\newcommand{\clarify}[1]{\todoTrans[color=orange!40]{#1}\xspace}
\newcommand{\thoughts}[1]{\todoTrans[color=orange!40,
inline]{\textbf{Thoughts:} #1}\xspace}


\mathchardef\ordinarycolon\mathcode`\:
\mathcode`\:=\string"8000
\begingroup \catcode`\:=\active
  \gdef:{\mathrel{\mathop\ordinarycolon}}
\endgroup

\usepackage{datetime}

\newdateformat{thesisdate}{%
\monthname[\THEMONTH] \THEYEAR}

% Schuster jungen und hurenkinder
\clubpenalty = 10000
\widowpenalty = 10000
\displaywidowpenalty = 10000

\newcommand{\vs}{\emph{vs}\xspace}


\begin{document}

\chapter{Deep Learning of Brain Images and its Application to Multiple Sclerosis}

\section{Introduction}

What is deep learning? Deep learning is used to describe a multitude of
learning-based methods with certain common characteristics: a) the use of
multiple layers of nonlinear processing units, and b) the unsupervised or
supervised learning of feature representations in each layer, with the layers
forming a hierarchy from low-level to high-level features. In this chapter, we
will introduce the most commonly used deep learning methods for medical image
analysis. We start with a description of unsupervised models like the restricted
Boltzmann Machine (RBM), which are the building blocks of deep belief networks,
a model that can be used for learning a hierarchical set of features from input
images without the need of labels. Later, we will introduce supervised models
like neural networks, how these models can be adapted for unsupervised learning,
e.g., stacked auto encoders, and scaled to larger images using convolutional
neural networks.

The distinction between supervised and unsupervised models is lose. Primarily
unsupervised models can also be used for supervised tasks like classification,
while supervised models can be used for unsupervised feature learning, by
reformulating the unsupervised problem to a supervised problem. Examples will be
given below.

\subsection{Deep Graphical Models}

Unsupervised models are mostly used for the automatic learning of features of an
image domain from a representative data set. They are particularly interesting
for medical image analysis as they can be trained without the need of labelled
data (e.g., segmented images, images with patient data), which can be difficult
to obtain. In this chapter, we will first introduce the restricted Boltzmann
machines, which are the building blocks of the later described deep belief
networks.

\subsubsection{Restricted Boltzmann Machines}

A restricted Boltzmann machine is a probabilistic graphical models defined by a
bipartite graph as shown in Figure~\ref{fig:rbm}. The units of the RBM are
binary random variables. They are divided into visible units $\vect{v}$ and
hidden units $\vect{h}$. There are no direct connections between units of the
visible or hidden layer. The RBM defines the joint probability of visible and
hidden units in terms of the energy $E$ which is given by:
\begin{align}
p(\vect{v}, \vect{h} \given \boldsymbol{\theta}) &=
\frac{1}{Z(\boldsymbol{\theta})}e^{-E(\vect{v}, \vect{h} \given
\boldsymbol{\theta})} \\
\intertext{with}
-E(\vect{v}, \vect{h}\given \boldsymbol{\theta}) &= \sum_{i, j}v_i w_{ij} h_j +
\sum_i b_i v_i + \sum_j c_j h_j \\
&= \vect{v}^\textup{T}\vect{W}\vect{h} + \vect{b}^\textup{T}\vect{v} +
\vect{c}^\textup{T}\vect{h}
\end{align}
and $\thetas = \{\vect{W}, \vect{b}, \vect{c}\}$.
\begin{figure}
\centering
\input{figures/rbmoverview}
\caption{Graph representation of an RBM with 3 hidden and 5 visible units.
  An RBM is used to model the joint probability of input, output and hidden units.
  Output units are optional. Edges between vertices denote conditional
  dependence between the according random variables.}
\label{fig:rbm}
\end{figure}

\paragraph{Inference}
Inference is the basis for most calculations of an RBM (e.g. training, sampling,
encoding, decoding, \dots). Since the hidden units are conditionally independent
given the visible units and vice versa, inference can be calculate efficiently
in an RBM. The posteriors are given by:
\begin{align}
%p(h_j = 1 \given \vect{v}, \thetas) &= \sigm(\Delta E_j)\\
%p(v_i = 1 \given \vect{h}, \thetas) &= \sigm(\Delta E_i)\\
%\intertext{with $\sigm(x) = (1+e^{-x})^{-1}$ is the sigmoid function and}
%\Delta E_j & := E(h_j = 0) - E(h_j = 1) \\
%\Delta E_i & := E(v_i = 0) - E(v_i = 1) \\
%\intertext{Plugging in the energy function of a binary RBM yields}
\label{eq:hgivenv}
p(h_j = 1 \given \vect{v}, \thetas) &= \sigm(\vect{w}_{\cdot,j}^\text{T}\vect{v}
+ c_j)\\
\label{eq:vgivenh}
p(v_i = 1 \given \vect{h}, \thetas) &= \sigm(\vect{w}_{i,
\cdot}^\text{T}\vect{h} + b_i)
\end{align}
where $\sigm(x)$ is the sigmoid function defined as $\sigm(x) = (1+
\exp(-x))^{-1}, x \in \R$.

\paragraph{Sampling}
Important expectations used for classification, $\E[\vect{y} \given
\vect{x}, \thetas]$, or training, $\E[\vect{v}\vect{h}^\text{T}\given
\thetas]$, can not be calculated directly. In this case, they need to be
approximated by the average of samples drawn from the according models. Sampling
from the posteriors forms the basis for generating samples from other models.
The posteriors can be sampled using \ref{eq:hgivenv} and \ref{eq:vgivenh},
respectively:
\begin{align}
h_j &= \I(y_j < \E[h_j \given \vect{v}, \thetas]) & \text{with $y_j \sim
\text{U}(0,1)$}\\
v_i &= \I(x_i < \E[v_i \given \vect{h}, \thetas]) & \text{with $x_i \sim
\text{U}(0,1)$}
\end{align}
where $z \sim \text{U}(0,1)$ denotes a sample from the uniform distribution in
the interval $[0,1]$ and $\I$~is the indicator function, which is defined as $1$ if
the argument is true and $0$ otherwise.

A sample from the joint probability, $p(\vect{v}, \vect{h} \given \thetas)$, can
be drawn using block Gibbs sampling. For this purpose, $\vect{v}$ is initialized
with random values at the beginning. Then, the two posteriors $p(\vect{h}
\given \vect{v}, \thetas)$ and $p(\vect{v} \given \vect{h}, \thetas)$ are
sampled alternately. If this chain is run for sufficiently many iterations,
samples drawn from the posteriors are distributed according to $p(\vect{v},
\vect{h} \given \thetas)$.

\paragraph{Learning}

RBMs can be trained using maximum likelihood estimation (MLE). The gradient of
the log likelihood function is given by:
\begin{equation}
\nabla_{\thetas} \log p(\mathcal{D} \given \thetas) =
\frac{1}{N} \sum_{n = 1}^N
\E[\vect{v}\vect{h}^\text{T} \given \vect{v}_n, \thetas]
-\E[\vect{v}\vect{h}^\text{T} \given \thetas]
\end{equation}
The first expectation can be estimated using a mean field approximation:
\begin{align}
\E[\vect{v}\vect{h}^\text{T} \given \vect{v}_n, \thetas] &\approx
\E[\vect{v} \given \vect{v}_n, \thetas]
\E[\vect{h}^\text{T} \given \vect{v}_n, \thetas] \\
&=\vect{v}_n\E[\vect{h}^\text{T} \given \vect{v}_n, \thetas]
\end{align}
The second expectation needs to be estimated using a Monte Carlo approximation:
\begin{equation}
\E[\vect{v}\vect{h}^\text{T} \given \thetas] \approx
\frac{1}{S}\sum_{s=1}^{S}\vect{v}_s\vect{h}_s^\text{T}
\end{equation}
where $S$ is the number of samples, and $\vect{v}_s$ and $\vect{h}_s$ are
samples drawn from $p(\vect{v}\given \thetas)$ and $p(\vect{h}\given \thetas)$,
respectively. If the Gibbs sampler is initialized at a data point from the
training set and only $1$ Monte Carlo sample is used to approximate the second
expectation, the learning rule is also known as contrastive divergence (CD)
\cite{Hinton2002}. To speed up learning, the data set is divided into small
subsets called mini batches and a gradient step is performed for each mini
batch. To avoid confusion with a gradient step, the term ``iteration'' is
generally avoided and the term ``epoch'' is used instead to indicate a sweep
through the entire data set. Additional tricks to monitor and speed up the
training of an RBM were summarized by Hinton et al. \cite{Hinton_2010}.

\paragraph{Gaussian--Bernoulli RBM} To model real-valued inputs like the
intensities of medical images. This changes the joint probability to:
\begin{equation} 
E = bla
\end{equation}
If the visible units are zero mean and unit variance, inference goes like this:
\begin{align} 
\E[v] &= bla \\
\E[h] &= bla
\end{align}

\paragraph{Rectified linear units} Hidden state is only binary in an RBM. Encode
signal strength by repetition of the same unit with different bias. It can be
shown that this type of unit as an activation of $\exp(...)$, which can be
approximated as $\max(0, x)$. This units are refered to as rectified linear
units and have shown to improve classification performance.

\paragraph{Convolutional RBMs} To scale to larger images, can use weight sharing
and limited receptive field. Can be expressed as a convolution. Greatly reduces
the number of trainable weights, employs translational equivariance. Allows RBMs
to be trained on high-resolution images.

Training can be further optimized by training in the frequency domain as has
been done by some people: ICLR paper, our neural computation paper, fbfft
paper.

\subsubsection{Deep Belief Networks}

RBMs are arguably not deep. To learn a hierarchy of features, RBMs can be
stacked on top of each other to form a deep belief network. DBNs are trained
layer by layer and are often fine-tuned using a neural network.

\paragraph{For hierarchical feature learning}
Train on input images. Can be done with convolutional and dense DBNs. Learns a
hierarchy of features.

\paragraph{For multi-model feature learning}
Different pathways that are then later joint.

\paragraph{For classification}
A special case of multi-modal feature learning, where the output labels are
modelled as an additional modality.

% \subsubsection{Deep Boltzmann Machine}
% 
% Similar to DBN but connections are undirected. Inference of one layer requires
% the unit activation of the previous and following layer. Inference need to be
% performed using Gibbs sampling. Can be pre-trained using RBMs using
% weight-doubling and fine-tuned using persistent contrastive divergence.

\subsection{Deep Neural Networks}

First, we will introduce dense and convolutional neural networks separately.
Later we will talk about specific variations, which apply to both models.

\subsubsection{Dense Neural Networks}

A neural network is a deterministic function. Successive propagation of a signal
through multiple layers. Give the equation of a neural network. Neural networks
are used to predict the output given some inputs and are trained supervised.
Given an error function, we can formulate learning as minimizing the error, as
opposed to maximizing the log-likelihood. Optimization can be performed using
stochastic gradient descent (SGD). The gradient can be calculated as follows:
\begin{itemize}
\item Calculate derivative of error function using the chain rule
\item Calculate derivative of the non-linearity using the chain rule
\item Calculate derivative of total activation using chain rule
\item Proceed to lower layers using the chain rule
\end{itemize}
This procedure is called backpropagation.

\subsubsection{Convolutional Neural Networks}

Very similar to dense models but uses a convolution to calculate activation.
Show equations for convolutional models and derivative of the gradient using
back-propagation.

% I might want to shorten the training section

\subsubsection{Training}

Training algorithms can be roughly categorized into algorithms that use second
order information and algorithms that only use first-order information.

Vanilla stochastic gradient descent is prone to oscillation. This can be limited
by using a momentum term. Momentum update favours persistent directions, as they
are accumulated.

Nesterov momentum is a slightly different version. Since we gonna take a step in
the momentum direction anyways, we calculate the gradient of the look-ahead step
instead of the current step. With some massaging to the equations, we can the
following update rule.

The exists also a large buddy of work on per-parameter update methods. For
example AdaGrad, AdaDelta, Adam, vSGD, eSGD, RMSprop.

There are also a lot of methods that leverage second order information, e.g.,
Hessian-free optimization used information from the Hessian without ever
calculating the full Hessian in order to better navigate pathological curvature.
Recenter, it was hypothesised that characteristic points that were believed to
be local minima are in fact saddle points. To escape saddle points and to
improve optimization, Bengio et al. have developed the so-called saddle-free
Newton's method that uses second order information to quickly escape saddle
points.

\subsubsection{Neural Networks for Unsupervised Feature Learning: Stacked
Auto-Encoders}

Neural networks can be used for feature learning even when no labels are
present. This can be done by using auto-encoder networks. An autoencoder is a
neural network with a bottleneck layers, that is trained to predict the input
images themselves. The bottleneck layer is used to force the network to learn
more abstract features. Auto-encoders can be pre-trained layer-by-layer similar
to DBNs. The first auto-encoder has 2 layers and is trained on the training
images. Subsequent autoencoders are trained on the hidden activations of the
previous auto encoder. I need an image for that. For fine-tuning all layers can
be combined into a single network.

\end{document}
