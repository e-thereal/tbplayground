%%%%%%%%%%%%%%%%%%%%%%%%%%%%%%%%%%%%%%%%%
% Long Lined Cover Letter
% LaTeX Template
% Version 1.0 (1/6/13)
%
% This template has been downloaded from:
% http://www.LaTeXTemplates.com
%
% Original author:
% Matthew J. Miller
% http://www.matthewjmiller.net/howtos/customized-cover-letter-scripts/
%
% License:
% CC BY-NC-SA 3.0 (http://creativecommons.org/licenses/by-nc-sa/3.0/)
%
%%%%%%%%%%%%%%%%%%%%%%%%%%%%%%%%%%%%%%%%%

%----------------------------------------------------------------------------------------
% PACKAGES AND OTHER DOCUMENT CONFIGURATIONS
%----------------------------------------------------------------------------------------

%dateno
\documentclass[version=last,
%fromphone,fromemail,
subject=titled,
firsthead=true,
fromalign=center,
fromrule=aftername,
locfield=wide,
firsthead=false,
parskip=half*
]{scrlttr2} % Extra

\usepackage[english]{babel}
\usepackage{tikz}

% \usepackage{charter} % Use the Charter font for the document text

\begin{document}

\setkomavar{fromname}{Tom Brosch}

\setkomavar{fromphone}{+1(604)446-2704}

\setkomavar{fromemail}{brosch.tom@gmail.com}

\setkomavar{fromaddress}{%
210-1950 West 8th Avenue\\
Vancouver, BC, V6J\,1W3, Canada}

% LAYOUT

% \setkomavar{location}{%
% \usekomavar*{fromphone} \usekomavar{fromphone}\\
% \usekomavar*{fromemail} \usekomavar{fromemail}
% }

\firstfoot{%
\centering%
\usekomavar*{fromphone}\usekomavar{fromphone}\quad $\bullet$\quad%
\usekomavar*{fromemail}\usekomavar{fromemail}%
}

\renewcommand*{\raggedsignature}{\raggedright}
\setkomafont{fromname}{\small}
\setkomafont{fromaddress}{\small}

% BEGIN OF NEW LETTER

\setkomavar{subject}{Letter of motivation}

\setkomavar{signature}{\mbox{}\\[1em]
\tikz \node[above=0.75cm,overlay,xshift=1.75cm,inner sep=0pt] {%
\includegraphics[width=3.5cm]{sig2}};%
Tom Brosch\\
PhD Candidate, Biomedical Engineering}

\begin{letter}{Philips GmbH Innovative Technologies\\
Research Laboratories\\
R\"ontgenstra\ss e 24--26\\
22335 Hamburg
}

\opening{Dear Sir or Madam,}

%----------------------------------------------------------------------------------------
% LETTER CONTENT
%----------------------------------------------------------------------------------------

I'm writing this letter to express my motivation to work as a medical imaging
research scientist for Philips HealthCare.

% What is my background

I began my research journey with the study of ``Computervisualistik''. Always
being very good in math and physics but never in arts, I was fascinated by the
possibility to use my knowledge of the physics of light to generate stunningly
looking images. The new program ``Computervisualistik'' at the University of
Magdeburg promised to teach me the programming and computer graphics skills
needed to pursue my ambition in computer graphics, but soon I realized that I
was much more engaged by the opposite way---not generating, but analysing
images. The apparent contradiction that many image analysis problems, like
recognizing a chair in an image, can be done seemingly effortlessly by a human
but is very difficult to do by a computer has immediately sparked my curiosity.
Nature has traditionally served as a means of inspiration for new inventions.
For image processing, it is a subject as fascinating as the human mind itself
that serves as a source of new ideas; that allows the discovery of new
algorithms simply by the joyful task of self-reflection.

The reason why we are so remarkable at understanding images is probably our
ability to integrate past experiences and current observations. A key question
of many image processing tasks is therefore ``how can we model experience?'',
or in other words, ``how can we model prior knowledge?'' Driven by these
questions, I started my doctoral research with the development of model-based
image segmentation methods such as active appearance models, but soon found them
being to limited to be applicable to the wide range of image understanding
problems that our brains are able to solve.

Driven by my na\"ive ambition to find a segmentation method that can be adapted
to solve a wide range of segmentation problems, I stumbled upon the paper ``To
recognize shapes, first learn to generate images'' by Geoffrey E. Hinton, which
sparked my interest in deep learning. The paper proposed to solve an image
recognition problem by turning it upside-down and instead to solve the image
generation problem first. A generative model of images of a particular domain
can be learned from a training set without the need to know what those images
mean. This is similar to a newborn who is able to build a model of the world
simply by observation. Once the model is trained, the image generation processes
can be inverted to detect patterns based on which new images can be classified.
What seemed to be a reasonable approximation of the inherent learning objective
of the human brain turned out to become my Ph.D. project, during which I have
gained a deep understanding of the promises and limitations of artificial neural
networks.

Developing new models and algorithms is like imagining a story---implementing
them is like telling a story and seeing how your story comes to life. Driven by
the desire to write good stories, I have engaged in many software projects from
3D stereoscopic vision, GPU-accelerated CT cone-beam reconstruction methods and
multimodal image registration, to the implementation of learning based models of
brain images and new multimodal segmentation algorithms of brain pathology based
on deep learning. However, a computer program is only as good as it meets the
requirements posed by its users, not what the programmer beliefs is right.
Therefore, I've enjoyed working in teams with other developers and users from
different disciplines like sports sciences, interventional physicians, imaging
physicists, radiologists, and neurologists.

% Other examples, virtual Karate trainer for the sports department,
% interventional registation


% Image processing is often an ambigues problem that can be only solved when
% past experience and current observations are combined.

% PARAGRAPH ONE: State the reason for the letter, name the position or type of
% work you are applying for and identify the source from which you learned of
 % the opening (i.e. career development center, newspaper, employment service,
 % personal contact).
 


% PARAGRAPH TWO: Indicate why you are interested in the position, the company,
% its products, services - above all, stress what you can do for the employer.
 % If you are a recent graduate, explain how your academic background makes you
 % a qualified candidate for the position. If you have practical work
 % experience, point out specific achievements or unique qualifications. Try not
 % to repeat the same information the reader will find in the resume. Refer the
 % reader to the enclosed resume or application which summarizes your
 % qualifications, training, and experiences. The purpose of this section is to
 % strengthen your resume by providing details which bring your experiences to
 % life.
 
% PARAGRAPH THREE: Request a personal interview and indicate your flexibility as
% to the time and place. Repeat your phone number in the letter and offer
 % assistance to help in a speedy response. For example, state that you will be
 % in the city where the company is located on a certain date and would like to
 % set up an interview. Alternatively, state that you will call on a certain
 % date to set up an interview. End the letter by thanking the employer for
 % taking time to consider your credentials.
 
\closing{Yours sincerely,}
 
\end{letter}

\end{document}
