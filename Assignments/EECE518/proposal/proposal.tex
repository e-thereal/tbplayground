%%This is a very basic article template.
%%There is just one section and two subsections.
\documentclass{assign}

\usepackage{lmodern}

\newcommand{\email}[1]{\href{mailto:#1}{\nolinkurl{#1}}}

\newcommand{\objectives}{\subsubsection{Objective}}
\newcommand{\inputs}{\subsubsection{Inputs}}
\newcommand{\outputs}{\subsubsection{Outputs}}
\newcommand{\details}{\subsubsection{Details}}

\title{{\normalsize \textsc{Proposal}\\}DarkVR: The Virtual Blind Reality}
\author{Tom Brosch (17362104)}

\begin{document}

\maketitle

%\begin{abstract}
%Creating a virtual Dialogue in the Dark experience.
%\end{abstract}

\section{Introduction}

The project aims to create a virtual reality environment without visual cues and
without substituting visual cues by auditory cues (e.g. playing the sound of a
creaking door when you stand right in front of a door). The aim of the project
is, therefore, to create and experience a virtual world as close as possible to
how a blind would experience the real world.

This work is most closely related to the \textsc{Dialogue in the Dark}
project\cite{dialog}. \textsc{Dialogue in the Dark} is an exhibition which lets
non-blind people experience the world of the blinds and, thereby, raises
awareness for the problems blind people are faced everyday while navigate
through the world. Since the created environments are replicas build in an
indoor space, the number of scenarios that are rebuild is very limited. A
virtual reality can help bringing the blindness-experience to scenarios that are
too expensive to rebuild (e.g. navigating through an entire city) or too
dangerous (e.g. due to traffic).

The work is also related to video games for blinds. Video games for blinds can
be roughly classified into two categories: (1) games, which were initially
designed for blinds, and (2) games which have been made accessible to blinds.
Games of the first category often either use a combination of a
narrator, which introduces and describes a new scene, and auditory feedback,
which helps the user to navigate through the virtual environment (e.g. ``Der Tag
wird zur Nacht''\cite{dertag}); or use sound as the primary gaming concept (e.g.
``AudiOdyssey''\cite{audiodyssey}).

There has also been a great amount of research in how to make games accessible
for disabled people (e.g. \cite{chile, terraformers, secondlife,
tankcommander}). See \cite{survey} for a survey of accessible games. A common
method is to replace visuals with audio. This can be done on different levels of
abstraction. Audio cues are the least abstract way of replacing visuals with
audio. Audio cues are sounds that an object would make in the real world (e.g.
footsteps of a walking person). Auditory Icons or ear cons are sounds which are
associated with an object, even though the object would not make this sound in
the current situation (e.g. playing the sound of a creaking door when you stand
in front of it, even if the door is not moving). Sonification is used to
translate physical properties of an object into sound. These sounds are usually
artificial. Properties like pitch or frequency are used to represent other
properties like distance.

In contrast to the many attempts to making a virtual world accessible for the
blinds, this project aims to create an experience which is as closed as
possible to the real experience of blindness. Therefore, accessibility methods
like auditory icons/ear cons and sonifications are not used on purpose. Instead,
the user navigates through the world relying on 3D binaural audio and devices
like a virtual long cane or a virtual GPS localizer. \textsc{Dark VR}, therefore,
does not only give the possibility to experience blindness, it also gives the
possibility to try out imaginary devices that might help blinds in the
future to navigate and orientate in the real world.

\section{Hardware Requirements}

\begin{itemize}
  \item Computer
  \item Webcam (for head-tracking)
  \item Headphones
  \item Wiimote (virtual cane)
\end{itemize}

\section{Specification and Initial Design}

\subsection{Virtual Environment Creation}

\objectives To describe the geometry of the environment and the sound
sources in 3D.

\outputs 3D model of the environment including 3D sound sources with according
sound files

\subsection{Head-tracking}

\objectives To track the 3D position and rotation of the head

\inputs Camera image of the head with coloured balls on each side. Parameters of
the setting.

\outputs 5D coordinates of the head (3D position + 2D rotations)

\subsection{Orientation Devices}

\objectives To supply the user with a variety of devices that will help the user
to navigate the virtual world.

\inputs Sensor inputs (e.g. of the the Wiimote)

\outputs 3D models and current positions of virtual devices.

\details The following devices were reported to be helpful for
navigating in the real world\cite{secondlife} and are, therefore, potential
candidates:
\begin{description}
  \item[Long Cane] Simulate haptic feedback with a Wiimote. Use Wiimote to scan
  objects in front of the user. Will vibrate when hitting a target and generate
  an according sound. This device provides a basic way of orientation in a
  virtual 3D environment.
  \item[UltraCane] The UltraCane\cite{ultracane} can measure the distance to
  objects via ultra sound. The distance is then mapped to vibrations. Thereby,
  the user can feel the distance to objects. It will also be simulated using the
  Wiimote.
  \item[GPS Localizer] On request, a voice interface can inform the user about
  his current position.
\end{description}

\subsection{Interactive Virtual Environment}

\objectives To display the virtual environment to the user.

\inputs 5D head position, 3D environment model, input from virtual devices

\outputs Audio.

\bibliographystyle{plain}
\bibliography{proposal}

\end{document}
