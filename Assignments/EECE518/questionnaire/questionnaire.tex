\documentclass[halfparskip,letterpaper]{scrartcl}

\usepackage{wasysym}

\usepackage{scrpage2,lastpage}
\pagestyle{scrheadings}

\usepackage[utf8]{inputenc}
\usepackage{wasysym}% provides \ocircle and \Box
\usepackage{enumitem}% easy control of topsep and leftmargin for lists
\usepackage{color}% used for background color
\usepackage{forloop}% used for \Qrating and \Qlines
\usepackage{ifthen}% used for \Qitem and \QItem
\usepackage{typearea}
\areaset{17cm}{22cm}
\setlength{\topmargin}{-1cm}

\newdimen\scalewidth
\scalewidth=0.3\hsize

\def\boxit#1{\hbox{\lower0.7ex\vbox{\hrule\hbox{\vrule\kern1pt
    \vbox{\kern1pt\hbox to 1.4em
    {\small\strut\hfil #1\hfil}\kern1pt}\kern1pt\vrule}\hrule}}}

\def\fiveboxes#1#2#3#4#5{\hbox to\scalewidth
    {\boxit{#1}\hfil\boxit{#2}\hfil\boxit{#3}\hfil%
     \boxit{#4}\hfil\boxit{#5}}}
     
\def\xagree{\xscale{strongly disagree}{agree completely}}
\def\boxes{\fiveboxes{}{}{}{}{}\ignorespaces}
\def\xscale#1#2{%
    \setbox0=\hbox{\boxes}%
    \setbox1=\hbox to \wd0{\small\strut\hfill #2 $\to$}%
    \setbox2=\hbox to \wd0{\small\strut $\gets$ #1 \hfill}%
    \vbox{\vbox to 0pt{\vss\box1\box2\kern2pt}\vbox{\box0}}}

\def\question#1\par#2\par{\hbox to \hsize
{\vbox{\hsize=0.71\hsize #1\dotfill}\quad#2\hfil}\medskip\goodbreak}

\def\freequestion#1\par{\subsubsection*{#1}\par\nobreak
    \begingroup\nobreak
    \advance\leftskip by 1.3pc
    \advance\rightskip by 0.3pc
    \hrule width 0pt height 1.7\baselineskip\hrulefill
    \hrule width 0pt height 1.7\baselineskip\hrulefill
    \par
    \medskip
    \endgroup
    }
    
\def\longfreequestion#1\par{\subsubsection*{#1}\par\nobreak
    \begingroup\nobreak
    \advance\leftskip by 1.3pc
    \advance\rightskip by 0.3pc
    \hrule width 0pt height 1.7\baselineskip\hrulefill
    \hrule width 0pt height 1.7\baselineskip\hrulefill
    \hrule width 0pt height 1.7\baselineskip\hrulefill
    \hrule width 0pt height 1.7\baselineskip\hrulefill
    \par
    \medskip
    \endgroup
    }

%% \Qq = Questionaire question. Oh, this is just too simple. It helps
%% making it easy to globally change the appearance of questions.
\newcommand{\Qq}[1]{\textbf{#1}}

%% \QO = Circle or box to be ticked. Used both by direct call and by
%% \Qrating and \Qlist.
\newcommand{\QO}{\boxit{}}% or: $\ocircle$

%% \Qrating = Automatically create a rating scale with NUM steps, like
%% this: 0--0--0--0--0.
\newcounter{qr}
\newcommand{\Qrating}[1]{\QO\forloop{qr}{1}{\value{qr} < #1}{---\QO}}

%% \Qline = Again, this is very simple. It helps setting the line
%% thickness globally. Used both by direct call and by \Qlines.
\newcommand{\Qline}[1]{\rule{#1}{0.6pt}}

%% \Qlines = Insert NUM lines with width=\linewith. You can change the
%% \vskip value to adjust the spacing.
\newcounter{ql}
\newcommand{\Qlines}[1]{\forloop{ql}{0}{\value{ql}<#1}{\vskip0em\Qline{\linewidth}}}

%% \Qlist = This is an environment very similar to itemize but with
%% \QO in front of each list item. Useful for classical multiple
%% choice. Change leftmargin and topsep accourding to your taste.
\newenvironment{Qlist}{%
  \renewcommand{\labelitemi}{\QO}
  \begin{itemize}[leftmargin=2.7em,topsep=.5em]
}{%
  \end{itemize}
}

%% \Qtab = A "tabulator simulation". The first argument is the
%% distance from the left margin. The second argument is content which
%% is indented within the current row.
\newlength{\qt}
\newcommand{\Qtab}[2]{
  \setlength{\qt}{\linewidth}
  \addtolength{\qt}{-#1}
  \hfill\parbox[t]{\qt}{\raggedright #2}
}

%% \Qitem = Item with automatic numbering. The first optional argument
%% can be used to create sub-items like 2a, 2b, 2c, ... The item
%% number is increased if the first argument is omitted or equals 'a'.
%% You will have to adjust this if you prefer a different numbering
%% scheme. Adjust topsep and leftmargin as needed.
\newcounter{itemnummer}
\newcommand{\Qitem}[2][]{% #1 optional, #2 notwendig
  \ifthenelse{\equal{#1}{}}{\stepcounter{itemnummer}}{}
  \ifthenelse{\equal{#1}{a}}{\stepcounter{itemnummer}}{}
  \begin{enumerate}[topsep=2pt,leftmargin=2.8em]
  \item[\textbf{\arabic{itemnummer}#1.}] #2
  \end{enumerate}
}

%% \QItem = Like \Qitem but with alternating background color. This
%% might be error prone as I hard-coded some lengths (-5.25pt and
%% -3pt)! I do not yet understand why I need them.

\ihead[\upshape\sffamily Subject \#: \Qline{1cm}]{\upshape\sffamily Subject \#:
\Qline{1cm}} \ohead[\upshape\sffamily Page: \thepage{} of
\pageref{LastPage}]{\upshape\sffamily Page: \thepage{} of \pageref{LastPage}}
\chead[]{}
\cfoot[\upshape\sffamily Tom Brosch]{\upshape\sffamily Tom Brosch}
\ifoot[\upshape\sffamily EECE 518]{\upshape\sffamily EECE 518}
\ofoot[\upshape\sffamily ver1.01 \today]{\upshape\sffamily ver1.01 \today}

\title{Questionnaire}
\author{EECE 518}

\begin{document}

\vspace*{1em}
\begin{center}
\Huge\sffamily\textbf{Questionnaire}
\end{center}
\vspace{1em}
%\maketitle

\section{Personal information}

\subsubsection*{In what age group are you?}
  \begin{Qlist}
  \item 19 and under 
  \item 20--29 
  \item 30--39 
  \item 40--49 
  \item 50--59 
  \item 60+ 
  \end{Qlist}
  
\subsubsection*{Gender:}
\begin{Qlist}
\item Male
\item Female
\end{Qlist}

\subsubsection*{Please rate your familiarity with the following games/programs}

\question First-person shooters (e.g. Counter Strike)\par\xscale{very
familiar}{not familiar}\par
\question Third-person roll playing games (e.g. World of Warcraft)\par\boxes\par
\question Pseudo-3D roll playing games (e.g. Diablo II, The Sims)\par\boxes\par
\question Computer games in general\par\boxes\par
\question Life simulations (e.g. SecondLife)\par\boxes\par

\section{To be completed before the experiment}

\subsection{Orientation and navigation for visually impaired}

\subsubsection*{Please rate the following tasks according to the difficulty
for a\\ visually impaired compared to a sighted:}

\question Crossing a street\par\xscale{a lot more difficult}{about the same}\par
\question Walking in a park\par\boxes\par
\question Navigating in an unknown apartment\par\boxes\par
\question Navigating in a known apartment\par\boxes\par

\freequestion Besides the tasks mentioned above, which other tasks do you think
are difficult for visually impaired?

\subsubsection*{Please rate how important the following cues are for
navigation:}
\question Listening to environmental sounds\par\xscale{very important}{not
important}\par
\question Listening to food step sounds\par\boxes\par
\question Touching with the hands\par\boxes\par
\question Force-feedback from the cane\par\boxes\par
\question Smelling environmental smells\par\boxes\par

\freequestion Besides the cues mentioned above, which other cues do you
think are important for navigation?

\section{To be completed after the experiments}

\subsection{Program evaluation}

\subsubsection*{Please rate the following aspects of the program:}

\question Navigation with vision was easy and intuitive\par\xagree\par
\question The localization of sounds in 3D was good\par\boxes\par
\question The cause of the sound was clear\par\boxes\par
\question 3D audio was convincing\par\boxes\par

\subsubsection*{Which rotation type did you like more?}
\begin{Qlist}
\item Discrete Rotations
\item Continuous Rotations
\end{Qlist}

\subsection{Orientation and navigation for visually impaired}

\subsubsection*{Please rate the following tasks according to the difficulty
for a\\ visually impaired compared to a sighted:}

\question Crossing a street\par\xscale{a lot more difficult}{about the same}\par
\question Walking in a park\par\boxes\par
\question Navigating in an unknown apartment\par\boxes\par
\question Navigating in a known apartment\par\boxes\par

\freequestion Besides the tasks mentioned above, which other tasks do you think
are difficult for visually impaired?

\clearpage
\subsubsection*{Please rate how important the following cues are for
navigation:}

\question Listening to environmental sounds\par\xscale{very important}{not
important}\par
\question Listening to food step sounds\par\boxes\par
\question Touching with the hands\par\boxes\par
\question Force-feedback from the cane\par\boxes\par
\question Smelling environmental smells\par\boxes\par

\freequestion Besides the cues mentioned above, which other cues do you
think are important for navigation?

\longfreequestion Additional comments about the study (optional):

\end{document}