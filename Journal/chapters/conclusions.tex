\section{Conclusions}

% We have introduced a new method for the automatic segmentation of MS lesions
% based on convolutional encoder networks. The joint training of the feature
% extraction and prediction layers with a novel objective function allows for the
% automatic learning of features that are tuned for a given combination of image
% types and a segmentation task with very unbalanced classes.
% We have evaluated our method on two data sets showing that approximately 100
% images are required to train the model without overfitting but even when only a
% relatively small training set is available, the method still performs comparably
% to the state-of-the-art algorithms. For future work, we plan to increase the
% depth of the network, which would allow the learning of a set of hierarchical
% features. This could further improve segmentation accuracy, but may require
% larger training sets. We would also like to investigate the use of different
% objective functions for training based on other measures of segmentation
% performance.

We have presented a new method for the automatic segmentation of MS lesions
based on deep convolutional encoder networks with shortcut connections.
The joint training of the feature extraction and prediction pathways allows for
the automatic learning of features at different scales that are tuned for a
given combination of image types and segmentation task. In addition, shortcuts
between the two pathways allow high- and low-level features to be leveraged at
the same time for more consistent performance across scales. We have evaluated
our method on two publicly available data sets and a large data set from an MS
clinical trial, with the results showing that our method performs on-par with
state-of-the-art methods, even for relatively small training set sizes, and is
able to segment MS more accurately than widely-used competing methods such as
EMS, LST-LGA, and Lesion-TOADS, when a suitable training set is available.
The substantial gains in accuracy were mostly due to an increase in lesion
sensitivity, especially for small lesions. Overall, the CEN with shortcuts
architecture performed consistently well over a wide range of lesion loads.

Our segmentation framework is very flexible and can be easily extended. One such
extension could be to incorporate prior knowledge about the tissue type of each
non-lesion voxel into the segmentation procedure. The probabilities of each
tissue class could be precomputed by a standard segmentation method, after which
they can be added as an additional channel to the input units of the CEN, which
would allow the CEN to take advantage of intensity information from different
modalities and prior knowledge about each tissue class to carry out the
segmentation. In addition, our method can be applied to other segmentation
tasks. Although we have only focused on the segmentation of MS lesions in this
paper, our method does not make any assumptions specific to MS lesion
segmentation. The features required to carry out the segmentation are solely
learned from training data, which allows our method to be used to segment
different types of pathology or anatomy when a suitable training set is
available.
