\sisetup{separate-uncertainty=true,detect-weight=true,detect-inline-weight=math}

\section{Experiments and Results}

% \uselengthunit{in}\printlength{\textwidth} = 4.8041 in
% \uselengthunit{mm}\printlength{\textwidth} = 121.99854 mm

We evaluated our method on two publicly available data sets, which allows for
the direct comparison with state-of-the-art methods. In addition, we have used a
very challenging data set containing four different MRI sequences of
relapsing-remitting MS patients from a multi-center MS clinical trial, which
represents the large variability in lesion size, shape, location and intensity
as well as varying contrasts produced by different scanners. The trial data set
was used to carry out a detailed analysis of different CEN architectures using
different combinations of modalities, with a comparison to publicly available
state-of-the-art methods.

\subsection{Data Sets and Pre-processing}

% TODO: How was split determined: all scans of the same patient and scanning
% site were put in the same group

% TODO: Why did I choose Lesion-TOADS?

\subsubsection{Trial data set}

The data set was collected from 67 different scanning sites using different
1.5\,T and 3\,T scanners for a clinical trial in relapsing-remitting MS, and
consists of 377 T1-weighted (T1w), T2-weighted (T2w), proton density-weighted
(PDw), and FLAIR MRIs from 195 subjects. The image dimensions are
\num{256x256x60} voxels with a voxel size of
\SI{0.936x0.936x3.000}{\milli\metre}. All images were skull-stripped using the
brain extraction tool (BET) \cite{jenkinson2005bet2}, followed by an intensity
normalization to the interval $[0,1]$, and a 6 degree-of-freedom intra-subject
registration. To speed-up the training, all images were cropped to a
\num{164x206x52} voxel region-of-interest with the brain roughly centered. The
ground truth segmentations were produced using an existing semiautomatic 2D
region-growing technique, which has been used successfully in a number of large
MS clinical trials (e.g., \cite{kappos2006long},
\cite{traboulsee2008reduction}). To carry out the segmentation, each lesion was
manually identified by an experienced radiologist and then interactively grown
from the seed point by a trained technician.

% TODO: out of the 67 scanning sites, the images of 55 were used for training
% and 12 for testing, which constitutes to 300 images of the training set and 77
% images of the test set.

We divided the data set into a training ($n=250$), validation ($n=50$) and test
set ($n=77$) such that images of each set were acquired from different scanning
sites. The three data sets were used for training our model and parameter tuning
of the competing methods, for monitoring the training progress, and to carry out
a detailed analysis of variants of our method and competing methods,
respectively. Pre-training and fine-tuning of a 7-layer CEN-s took approximately
27 hours and 37 hours, respectively, on a single GeForce GTX 780 graphics card.
However, once the network is trained, new multi-contrast images can be segmented
in less than one second.

\subsubsection{Public data sets}
The data set of the MICCAI 2008 MS lesion segmentation challenge
\cite{styner20083d} consists of 43 T1w, T2w, and FLAIR MRIs, divided into 20
training cases for which ground truth segmentation are made publicly available,
and 23 test cases. After training the model on the 20 training cases, we used
the trained model to segment the 23 test cases, which were send to the challenge
organizers for evaluation.

In addition, we evaluated our method on the T1w, T2w, PDw, and FLAIR MRIs of the
21 publicly available labeled cases from the ISBI 2015 Longitudinal MS lesion
segmentation challenge. The challenge was not open for new submissions at the
time of writing this article. Therefore, we evaluated our method on the
training. To allow for a direct comparison with the results of the second and
third placed methods, we followed their evaluation protocol and performed
leave-one-subject-out cross-validation.

\subsection{Competing methods}

We compared our method with four publicly available methods that are widely used
(e.g., \cite{sudre2015,subbanna2015,guizard2015}) as a reference point for the
comparison of MS lesion segmentation methods:
a) the expectation maximization segmentation (EMS) method \cite{vanleemput2001},
b) the lesion growth algorithm (LST-LGA) \cite{schmidt2012automated} as
implemented in the LST toolbox version 2.0.11, c) the lesion prediction
algorithm (LST-LPA) also implemented in the LST toolbox, and d) Lesion-TOADS
version 1.9 R \cite{shiee2010topology}. The Lesion-TOADS method has no tunable
parameters, so we used the default parameters to carry out the segmentation
using T1w and FLAIR MRIs. The performance of EMS depends on the choice of the
Mahalanobis distance $\kappa$, the threshold $t$ used to binarize the
probabilistic segmentation, and the modalities used. We carried out the
segmentation using two combinations of modalities: a) the modalities used in the
original paper \cite{vanleemput2001} (T1w, T2w, and PDw), and b) all four
available modalities (T1w, T2w, PD2, FLAIR). We compared the segmentations
produced for all combinations of $\kappa = 2, 2.2, \dotsc, 4.6$ and $t = 0.05,
0.1, \dotsc, 1$ with the ground truth segmentations on the training set and
chose the values that maximized the average DSC ($\kappa = 2.6, t = 0.75$;
$\kappa = 2.8, t = 0.9$). We used the same procedure to tune the initial
threshold $\kappa$ of LST-LGA using T1w and FLAIR MRIs for $\kappa = 0.05, 0.1,
\dotsc, 1$ and the threshold $t$ used to binarize the probabilistic
segmentations produced by LST-LPA also using T1w and FLAIR MRIs for $t = 0.05,
0.1, \dotsc, 1$. The optimal parameters were $\kappa = 0.1$ and $t = 0.45$,
respectively.

\subsection{Measures of Segmentation Accuracy}

\todo[inline]{Lisa has kindly agreed to update this section (thanks Lisa!). It
will include DSC, lesion TPR, lesion FPR, and volume difference.}

We used three different measures to evaluate segmentation accuracy, with the
primary measure being the Dice similarity coefficient (DSC)
\cite{dice1945measures}, which computes a normalized overlap value between the
produced and ground truth segmentations, and is defined as
\begin{equation}
\text{DSC} = \frac{2 \times \text{TP}}{2 \times \text{TP} + \text{FP} +
\text{FN}},
\end{equation}
where TP, FP, and FN denote the number of true positives, false positives, and
false negatives, respectively. A value of \SI{100}{\percent} indicates a perfect
overlap of the produced segmentation and the ground truth.
The DSC incorporates measures of over- and underestimation into a single
metric, which makes it a suitable measure to compare overall segmentation
accuracy.
In addition, we have used the true positive rate (TPR) and the positive
predictive value (PPV) to provide further information on specific aspects of
segmentation performance. The TPR is used to measure the fraction of the lesion
regions in the ground truth that are correctly identified by
an automatic method. It is defined as
\begin{equation}
\text{TPR} = \frac{\text{TP}}{\text{TP} + \text{FN}},
\end{equation}
where a value of \SI{100}{\percent} indicates that all true lesion voxels are
correctly identified. The PPV is used to determine the extent of the regions
falsely classified as lesion by an automatic method. It is defined as the
fraction of true lesion voxels out of all identified lesion voxels
\begin{equation}
\text{PPV} = \frac{\text{TP}}{\text{TP} + \text{FP}},
\end{equation}
where a value of \SI{100}{\percent} indicates that all voxels that are
classified as lesion voxels are indeed lesion voxels as defined by the ground
truth (no false positives).

\subsection{Setting the Training Parameters}

The most important parameters of the training method are the number of epochs
and the sensitivity ratio. Fig.~\ref{fig:epochs} shows the mean DSC evaluated on
the training and validation set during training of a 7-layer CEN-s for
increasing number of epochs. The mean DSC scores keep improving even after 400
epochs, albeit as a much slower rate. The optimal number of epochs is a
trade-off between accuracy and time required for training. Due to the relatively
small improvements after 400 epochs, we decided to stop the training procedure
after 500 epochs. Due to the relatively small size of the challenge data
sets, we did not use a dedicated validation set to choose the number of epochs.
Instead, we set the number of epochs to 2500, which results in roughly the same
number of gradient updates compared to the trial data set.
\begin{figure}
\centering
\includegraphics[width=\columnwidth]{figures/ems_progress2}
\caption{Development of mean DSC on the training and test set during training.
Only small improvements can be observed after 500 epochs.}
\label{fig:epochs}
\end{figure}

Fig.~\ref{fig:ratio} shows a set of ROC curves for different choices of the
sensitivity ratio ranging from 0.01 to 0.1. A plus marks the TPR and FPR after
thresholding using a threshold that maximizes the DSC on the training set. The
plots confirm our initial findings that our method is not sensitive to the
choice of the sensitivity ratio, which mostly affects the optimal threshold. We
chose a fixed sensitivity ration of 0.02 for all our experiments.

\begin{figure}
\centering
\includegraphics[width=\columnwidth]{figures/roc}
\caption{ROC curves for different sensitivity ratios $r$. A plus marks the TPR
and FPR of the optimal threshold. The ROC curves for different sensitivity
ratios are almost identical and only causes a change of the optimal threshold
$t$, which shows the robustness of our method with respect to the sensitivity
ratio.}
\label{fig:ratio}
\end{figure}

\subsection{Comparison on Public Data Sets}

To allow for a direct comparison with other state-of-the-art methods, we have
evaluated our method on the MICCAI 2008 MS lesion segmentation challenge
\cite{styner20083d} and the ISBI 2015 longitudinal MS lesion segmentation
challenge. In our previous paper \cite{brosch2015}, we showed that approximately
100 images are required to train the 3-layer CEN without overfitting and we
expect the required number of images to be even higher when adding more layers.
Due to the relatively small size of the training data sets provided by the two
challenges, we trained a CEN with only 3 layers on these data sets, to reduce
the risk of overfitting. The parameters of the models are summarized in
Table~\ref{tab:archchallenge}.

\begin{table}[tb]
\caption{Parameters of the 3-layer CEN for the evaluation on the challenge data
sets.}
\label{tab:archchallenge}
\begin{center}
\begin{tabular}{@{}lccr@{}}
\toprule
Layer type & Kernel Size & \#Filters & \multicolumn{1}{c}{Image Size} \\
\midrule
Input & --- & --- & $164\times 206\times 156\times c$\phantom{0} \\
Convolutional & $9\times 9\times 9\times c$\phantom{0} & 32 &
\num{156x198x148x32} \\
Deconvolutional & \num{9x9x9x32} & 1 & \num{164x206x156x1}\phantom{0} \\
\bottomrule
\end{tabular}
\end{center}
Note: The number of input channels $c$ is 3 for the MICCAI challenge and 4 for
the ISBI challenge.
\end{table}

A comparison of our method with other state-of-the-art methods evaluated on the
MICCAI challenge test data set is summarized in Table~\ref{tab:miccai}. Our
method ranked 6th out of 52 entries submitted to the challenge, outperforming
popular methods such as SLS by Roura et al. \cite{roura2015}, the random forest
approach by Geremia et al. \cite{geremia2010}, and Lesion-TOADS by Shiee et al.
\cite{shiee2010topology}, but performing worse than the patch-based segmentation
approach by Guizard et al. \cite{guizard2015}, or the MOPS approach by
Tomas-Fernandez et al. \cite{tomas2015}, which used additional images to build
the intensity model of a healthy population. This is a very impressive result
given the simplicity of the used model, the relatively small training set size,
and because we have not tuned our method for this particular data set in
contrast to most other methods, for which multiple entries were submitted
to the challenge.

\begin{table}
\sisetup{
  round-mode = places,
  round-precision = 2}%
\caption{Selected methods out of the 52 entries submitted for evaluation to the
MICCAI 2008 MS lesion segmentation challenge.}
\label{tab:miccai}
\begin{center}
\begin{tabular}{@{}clS[table-format=2.2]S[table-format=2.2]S[table-format=2.2]S[table-format=2.2]@{}}
\toprule
Rank & Method & {Score} & {LTPR} & {LFPR} & {VD} \\
\midrule
1  & Jesson et al. & 86.9386 & 48.70 & 28.25 & 80.15 \\
2  & Guizard et al. \cite{guizard2015}   & 86.1071 & 49.85 & 42.75 & 48.80 \\
4  & Tomas-Fernandez et. al \cite{tomas2015} & 84.464 & 46.9 & 44.6 & 45.60 \\
%5  & Jerman et al.          & 84.1555 \\
6  & Our method    & 84.0743 & 51.55 & 51.25 & 57.75 \\
%8  & Strumia et al.         & 83.9262 \\
%10 & Zhan et al.            & 82.6455 \\
11 & Roura et al.   \cite{roura2015} & 82.3442 & 50.15 & 41.85 & 111.60 \\
13 & Geremia et al. \cite{geremia2010}     & 82.0691 & 55.1 & 74.1 & 48.90 \\
24 & Shiee et al. \cite{shiee2010topology} & 79.8975 & 52.4 & 72.7 & 74.45 \\
%33 & Sudre et al. \cite{Sudre2015} & 77.9601 & 22.3 & 18.1 & 285.6 \\
\bottomrule
\end{tabular}
\end{center}
Note: Only the best entry per method is shown for multiple submission. Columns
LTPR, LFPR, and VD show the mean scores of the two raters in percent. Last
updated: Dec 2, 2015.
\end{table}

In addition, we evaluated our method on the 21 publicly available labeled cases
from the ISBI 2015 longitudinal MS lesion segmentation challenge. The challenge
organizers have not yet released the challenge results on their website.
Therefore, we have only compared our method to methods, who have given
sufficient details about the test produce and a summary of the mean DSC, LTPR
and LFPR scores for both raters to allow for a direct comparison.
Following the evaluation protocol of the second and third best method of the
challenge, we trained our model using leave-one-subject-out cross-validation on
the training images and compared our results to the segmentations provided by
both raters. Table~\ref{tab:isbi} summarizes the performance of our method and
competing methods as well as the performance of the two rates when compared
against each other. Although our method performs slightly worse the the second
and third best method on the challenge, it produces DSC scores that are in the
range of the competing methods. Furthermore, our method is more sensitive to
lesions then the other methods, but also produces more false positives.

\begin{table}
\caption{Comparison of our method with the second and third ranked methods from
the ISBI MS lesion segmentation challenge.}
\label{tab:isbi}
\begin{center}
\begin{tabular}{@{}lcccccc@{}}
\toprule
Method &
\multicolumn{3}{c}{Rater 1} &
\multicolumn{3}{c}{Rater 2} \\
& DSC & LTPR & LFPR & DSC & LTPR & LFPR \\
\midrule
Rater 1 & --- & --- & --- & 73.2 & 64.5 & 17.4 \\
Rater 2 & 73.2 & 82.6 & 35.5 & --- & --- & --- \\
Jesson et al. &  70.4 & 61.1 & 13.5 & 68.1 & 50.1 & 12.7 \\
Maier et al. (GT1) & 70 & 53 & 48 & 65 & 37 & 44 \\
Maier et al. (GT2) & 70 & 55 & 48 & 65 & 38 & 43 \\
Our method (GT1) & 68.4 & 74.5 & 54.5 & 64.4 & 63.0 & 52.8 \\
Our method (GT2) & 68.3 & 78.3 & 64.5 & 65.8 & 69.3 & 61.9 \\
\bottomrule
\end{tabular}
\end{center}
Note: The evaluation was performed on the training set using
leave-one-subject-out cross-validation. GT1 and GT2 denote that the model was
trained with the segmentations provided by the first and second rater as the
ground truth, respectively.
% Note: LTPR and LFPR values of the third method are shown in parenthesis to
% indicate that a different method was used to calculate those values, which only
% allows for an approximate comparison with the other methods.
\end{table}

\subsection{Comparison of Network Architectures, Input Modalities, and
Competing Methods}

To determine the effect of network architecture and input modalities, we
compared the segmentation performance of five different networks.
Specifically, we trained a 3-layer CEN and two 7-layer CENs, one with shortcut
connections and one without, on T1w and FLAIR MRIs, and two additional 7-layer
CEN-s on the modalities used by EMS (T1w, T2w, PDw) and all four modalities
(T1w, T2w, PDw, FLAIR). The parameters of the networks are given in
Table~\ref{tab:arch3} and Table~\ref{tab:arch7}. To roughly compensate for the
anisotropic voxel size of the input images, we chose an anisotropic filter size
of \num{9x9x5}. We also included the four competing methods discussed in
Section~III-B. A comparison of the segmentation accuracy of the trained networks
and competing methods is summarized in Table~\ref{tab:results1}. All CEN
architectures performed significantly better than the best performing competing
method LST-LGA in overall segmentation accuracy, where the improvements of the
mean DSC scores ranged from 3\,pts for the 3-layer CEN to 17\,pts for the
7-layer CEN with shortcut trained on all four modalities. The improved
segmentation performance was mostly due to an increase in lesion sensitivity.
LST-LGA achieved a mean lesion TPR of only \SI{37.50}{\percent}, whereas the CEN
with shortcut achieved a mean lesion TPR of \SI{54.55}{\percent} when trained on
the same modalities, and a mean lesion TPR of \SI{62.49} when trained on all
four modalities, while achieving a comparable number of lesion false positives.
The mean lesion FPRs and mean volume differences of LST-LGA and the 7-layer
CEN-s were roughly the same, when trained on the same modalities, and could be
further reduced when trained on different modalities.

\begin{table}[tb]
\caption{Parameters of the 3-layer CEN used to evaluate different training
methods.}
\label{tab:arch3}
\centering
\begin{tabular}{@{}lccr@{}}
\toprule
Layer type & Kernel Size & \#Filters & \multicolumn{1}{c}{Image Size} \\
\midrule
Input & --- & --- & \num{164x206x52x2}\phantom{0} \\
Convolutional & \num{9x9x5x2} & 32 & \num{156x198x48x32} \\
Deconvolutional & \num{9x9x5x32} & 1 & \num{164x206x52x1}\phantom{0} \\
\bottomrule
\end{tabular}
\end{table}

\begin{table}[tb]
\caption{Parameters of the 7-layer CEN-s used to evaluate different training
methods.}
\label{tab:arch7}
\centering
\begin{tabular}{@{}lccr@{}}
\toprule
Layer type & Kernel Size & \#Filters & \multicolumn{1}{c}{Image Size} \\
\midrule
Input & --- & --- & \num{164x206x52x2}\phantom{0} \\
Convolutional & \num{9x9x5x2} & 32 & \num{156x198x48x32} \\
Average Pooling & \num{2x2x2} & --- & \num{78x99x24x32} \\
Convolutional & \num{9x10x5x32} & 32 & \num{70x90x20x32} \\
Deconvolutional & \num{9x10x5x32} & 32 & \num{78x99x24x32} \\
Unpooling & \num{2x2x2} & --- & \num{156x198x48x32} \\
Deconvolutional & \num{9x9x5x32} & 1 & \num{164x206x52x1}\phantom{0} \\
\bottomrule
\end{tabular}
\end{table}

\begin{table}
\begin{center}
\caption{Comparison of the segmentation accuracy of different CEN models with
Lesion-TOADS}
\label{tab:results1}
\begin{tabular}{@{}lcccc@{}}
\toprule
Method & DSC [\%] & LTPR [\%] & LFPR [\%] & VD [\%] \\
\midrule
\multicolumn{5}{c}{\textit{Input modalities: T1w and FLAIR}} \\
\midrule
3-layer CEN \cite{brosch2015} & 49.24 & 57.33 & 61.39 & 43.45 \\
7-layer CEN & 52.07 & 43.88 & 29.06 & 37.01 \\ 
7-layer CEN-s & 55.76 & 54.55 & 38.64 & 36.30 \\[0.2em]
Lesion-TOADS \cite{shiee2010topology} & 40.04 & 56.56 & 82.90 & 49.36 \\ 
%SLS \cite{Roura} \\
LST-LGA \cite{schmidt2012automated} & 46.64 & 37.50 & 38.06 & 36.77 \\
LST-LPA \cite{schmidt2012automated} & 46.07 & 48.02 & 52.94 & 41.62 \\
\midrule
\multicolumn{5}{c}{\textit{Input modalities: T1w, T2w, and PDw}} \\
% \midrule
% Method & TPR [\%] & PPV [\%] & DSC [\%] \\
\midrule
7-layer CEN-s & 61.18 & 52.00 & 36.68 & 29.38 \\
EMS \cite{vanleemput2001} & 42.94 & 44.80 & 76.58 & 49.29 \\
\midrule
\multicolumn{5}{c}{\textit{Input modalities: T1w, T2w, FLAIR, and PDw}} \\
% \midrule
% Method & TPR [\%] & PPV [\%] & DSC [\%] \\
\midrule
7-layer CEN-s & 63.83 & 62.49 & 36.10 & 32.89 \\
EMS \cite{vanleemput2001} & 39.70 & 49.08 & 85.01 & 34.51 \\
\bottomrule
\end{tabular}
\end{center}
Note: The table shows the mean of the Dice similarity coefficient (DSC), lesion
true positive rate (LTPR), and lesion false positive rate (LFPR). Because
the volume difference (VD) is not limited to the interval $[0, 100]$, a
single outlier can heavily affect the calculation of the mean. We therefore
excluded outliers before calculating the mean of the VD for all methods.
\end{table}

This experiment also showed that increasing the depth of the CEN
and adding the shortcut connections improves the segmentation accuracy.
Increasing the depth of the CEN from three layers to seven layers improved the
mean DSC by 3\,pts. The improvement was confirmed to be statistically
significant using a one-sided paired $t$-test ($p\text{-value}=\num{1.3e-5}$).
Adding a shortcut to the network further improved the segmentation
accuracy as measured by the DSC by 3\,pts. A second one-sided paired $t$-test
was performed to confirm the statistical significance of the improvement with a
$p$-value of less than \num{1e-10}.

The impact of increasing the depth of the network on the segmentation
performance of very large lesions is illustrated in Fig.~\ref{fig:large}, where
the true positive, false negative, and false positive voxels are highlighted in
green, yellow, and red, respectively. The receptive field of the 3-layer CEN has
a size of only \num{17x17x9} voxels, which reduces its ability to identify very
large lesions marked by two white circles. In contrast, the 7-layer CEN has a
receptive field size of \num{49x53x26} voxels, which allows it to learn features
that can capture much larger lesions than the 3-layer CEN. Consequently, the
7-layer CEN, with and without shortcut, is able to learn a feature set that
captures large lesions much better than the 3-layer CEN, which results in an
improved segmentation. However, increasing the depth of the network without
adding shortcut connections reduces the networks sensitivity to very small
lesions as illustrated in Fig.~\ref{fig:small}. In this example, the 3-layer CEN
was able to detect three small lesions, indicated by the white circles, which
were missed by the 7-layer CEN. Adding shortcut connections enables our model to
learn a feature set that spans a wider range of lesion sizes, which increases
the sensitivity to small lesions and, hence, allows the 7-layer CEN-s to detect
all three small lesions (highlighted by the white circles), while still being
able to segment large lesions (see Fig.~\ref{fig:large}).

\begin{figure}
\begin{tikzpicture}[node distance=2.1cm and 0.334\columnwidth,
  font=\footnotesize, on grid]
\node[inner sep=0] (image1) {
  \includegraphics[width=\columnwidth]{figures/p50s35_large_lesions}
};
\node[above=of image1] (l7) {7-layer CEN};
\node[left=of l7] {3-layer CEN};
\node[right=of l7] {7-layer CEN-s};

\begin{scope}[xshift=0.333\columnwidth]
\draw[white,thick] (-12pt,-15pt) circle (9pt);
\draw[white,thick] (15pt,-15pt) circle (10pt);
\end{scope}
\begin{scope}[xshift=-0.333\columnwidth]
\draw[white,thick] (-12pt,-15pt) circle (9pt);
\draw[white,thick] (15pt,-15pt) circle (10pt);
\end{scope}
\draw[white,thick] (-12pt,-15pt) circle (9pt);
\draw[white,thick] (15pt,-15pt) circle (10pt);
\end{tikzpicture}
\caption{Large lesion problematic for 3-layer CEN due to limited size of the
receptive field. The adding layers increases the size of the receptive field,
which improves the detection of very large lesions.}
\label{fig:large}
\end{figure}

\begin{figure}
\begin{tikzpicture}[node distance=2.1cm and 0.334\columnwidth,
  font=\footnotesize, on grid]
\node[inner sep=0] (image1) {
  \includegraphics[width=\columnwidth]{figures/p25s35_small_lesions}
};
\node[above=of image1] (l7) {7-layer CEN};
\node[left=of l7] {3-layer CEN};
\node[right=of l7] {7-layer CEN-s};
\begin{scope}[xshift=0.3333\columnwidth]
\draw[white,thick] (-12.5pt,-11pt) circle (4pt);
\draw[white,thick] (0pt,4pt) circle (6pt);
\draw[white,thick] (-10pt,20pt) circle (3pt);
\end{scope}
\begin{scope}[xshift=-0.3333\columnwidth]
\draw[white,thick] (-12.5pt,-11pt) circle (4pt);
\draw[white,thick] (0pt,4pt) circle (6pt);
\draw[white,thick] (-10pt,20pt) circle (3pt);
\end{scope}
\draw[white,thick] (-12.5pt,-11pt) circle (4pt);
\draw[white,thick] (0pt,4pt) circle (6pt);
\draw[white,thick] (-10pt,20pt) circle (3pt);
\end{tikzpicture}

\caption{Comparison of segmentation performance of different CEN architectures
for small lesions. The white circles indicate lesions that were detected by the
3-layer CEN and the 7-layer CEN with shortcut. Increasing the size of the
receptive field decreases the sensitivity to small lesions. The addition of a
shortcut allows the detection of small lesions, while still being able
to detect large lesions (see Fig.~\ref{fig:large}).}
\label{fig:small}
\end{figure}

\subsection{Comparison for Different Lesion Sizes}

% TODO: say it's box plots and outliers are denoted by small black circles.

% \todo[inline]{Update groups table to show lesion sizes. Also redo the plots for
% lesion sizes. Give a better justification for dividing by lesion size. Small
% lesions can be more difficult to miss, while large lesions also come with some
% problems.}

To examine the effect of lesion size on segmentation performance, we stratified
the test set into five groups based on their mean reference lesion size as
summarized in Table~\ref{tab:groups}. A comparison of segmentation accuracy and
lesion detection measures of a 7-layer CEN-s trained on different input
modalities and the best performing competing method LST-LGA for different lesion
sizes is illustrated in Fig.~\ref{fig:sizecomp}. The 7-layer CEN-s outperforms
LST-LGA for all lesions sizes except for very large lesions when trained on T1w
and FLAIR MRIs, and for all lesion sizes when trained on all four modalities.
The differences are larger for smaller lesions, which are generally more
challenging to segment for all methods. The differences between the two
approaches are caused by a higher sensitivity to lesions as measured by the
lesion TPR, especially for smaller lesions, while producing approximately the
same number of false positives for all lesion sizes.

% Observations: 
% - CEN better for all lesions loads except for very high loads when
% trained on the same modalities, and outperforms LGA for all lesion sizes when
% trained on all four modalities. Differences are largest for very small
% lesions, which is the most challenging category for all methods.
% - Difference due to higher sensitivity than LGA especially for very small
% lesions, while producing approximately the same number of false positives as
% LGA for all lesion sizes.

% Most segmentation performance measures
% deteriorate with lower lesion loads, because when there are only a few true
% lesion voxels, even small segmentation errors can translate into large relative
% errors.

\begin{table}[tb]
\caption{Lesion size groups as used for the detailed analysis.}
\label{tab:groups}
\centering
\begin{tabular}{@{}lccc@{}}
\toprule
Group & Mean lesion size [\si{\cubic\milli\metre}] & \#Samples & Lesion
load [\si{\cubic\milli\metre}] \\
\midrule
% 0, 1250, 2500, 3800, 10000
Very small & $[0,70]$ & 6 & \num{1457+-1492} \\
Small      & $(70,140]$ & 24 & \num{4298+-2683} \\
Medium & $(140,280]$ & 24 & \num{12620+-9991} \\
Large & $(280,500]$ & 14 & \num{13872+-5814} \\
Very large & $> 500$ & 9 & \num{35238+-27531} \\
\bottomrule
\end{tabular}
\end{table}

% Fig.~\ref{fig:l2vlt} shows a comparison of the 7-layer CEN with shortcut and
% Lesion-TOADS. The CEN approach outperformed Lesion-TOADS for all lesion load
% groups, except the group with very high lesion loads, where Lesion-TOADS
% achieved a slightly higher mean DSC than the CEN approach, but the difference is
% much smaller than the gains in accuracy achieved by the CEN for the other lesion
% loads. The 7-layer CEN with shortcut also performed more consistently across
% lesion load groups, whereas Lesion-TOADS decreased more strongly in performance
% for the smaller lesion loads. As a result, the greatest differences between the
% two methods are seen in the lowest lesion load groups.
% % However, a two-sided paired $t$-test yielded that
% % the difference is not statistically significant ($p\text{-value}=0.2566$).
% Table~\ref{tab:result2} shows a more detailed
% comparison. While the PPV increased consistently with higher lesion loads for
% both methods, the TPR was highest for low to medium lesion loads and decreased
% again for high to very high lesion loads. This shows the difficulty for both
% methods to correctly identify very large lesions that can extend far into the
% white matter.

\begin{figure*}
\centering
\includegraphics[width=\textwidth]{figures/cen_vs_LGA_size}

\caption{Comparison of segmentation accuracy and lesion detection measures of a
7-layer CEN-s trained on different input modalities and the best performing
competing method LST-LGA for different lesion sizes. The 7-layer CEN-s
outperforms LST-LGA for all lesions sizes except for very large lesions when
trained on T1w and FLAIR MRIs, and for all lesion sizes when trained on all four
modalities, due to a higher sensitivity to lesions, while producing
approximately the same number of false positives. Outliers are denoted by black
dots.}

\label{fig:sizecomp}
\end{figure*}

% \begin{figure}[tb] 
% %\centering
% \includegraphics[width=0.92\columnwidth]{figures/boxplot_LTvsL2}
% \caption{Comparison of the segmentation accuracy (DSC) of
% Lesion-TOADS and a 7-layer CEN with shortcut connections for different lesion
% loads. The CEN approach is much more sensitive in detecting small lesions, while
% still being able to detect large lesions.}
% \label{fig:l2vlt}
% \end{figure}

\begin{comment}
\begin{table}
\caption{Comparison of segmentation accuracy for different lesion load
categories.}
\label{tab:result2}
\begin{center}
\begin{tabular}{@{}lcccccc@{}}
\toprule
\multicolumn{1}{@{}l}{Lesion load} & \multicolumn{3}{c}{7-layer CEN-s} &
\multicolumn{3}{c@{}}{Lesion-TOADS}
\\
& TPR & PPV & DSC & TPR & PPV & DSC \\
\midrule
% \phantom{000}$(0,1250]$\phantom{0} & \num{50.00} & \num{41.15} & \num{39.34} &
% \num{49.96} & \num{13.09} & \num{18.86}\\
% $(1250,2500]$\phantom{0} & \num{61.92} & \num{59.01} & \num{57.45} & \num{52.39}
% & \num{29.95} & \num{37.74}\\
% $(2500,3800]$\phantom{0} & \num{57.64} & \num{71.54} & \num{61.31} & \num{54.17}
% & \num{41.83} & \num{46.53}\\
% $(3800,10000]$ & \num{51.14} & \num{81.11} & \num{60.13} & \num{47.97} &
% \num{56.56} & \num{50.76}\\
% $> 10000$ & \num{32.82} & \num{91.95} & \num{48.19} & \num{38.88} & \num{74.6} &
% \num{50.93}\\
Very low & \num{50.00} & \num{41.15} & \num{39.34} &
\num{49.96} & \num{13.09} & \num{18.86}\\
Low & \num{61.92} & \num{59.01} & \num{57.45} & \num{52.39}
& \num{29.95} & \num{37.74}\\
Medium & \num{57.64} & \num{71.54} & \num{61.31} & \num{54.17}
& \num{41.83} & \num{46.53}\\
High & \num{51.14} & \num{81.11} & \num{60.13} & \num{47.97} &
\num{56.56} & \num{50.76}\\
Very high & \num{32.82} & \num{91.95} & \num{48.19} & \num{38.88} & \num{74.6} &
\num{50.93}\\
\bottomrule
\end{tabular}
\end{center}
\end{table}
\end{comment}

% \begin{itemize}
% \item Have some sample images and discuss what we can see here.
% \end{itemize}

%\subsection{Qualitative Results}

%TODO: show and discuss filters

% \todo[inline]{Update images to show good, mean and bad segmentation. It's more
% about showing the range of difficulty in the data set from easy images to
% difficult images. If time permits, show some filters and say something about
% them.}

% A qualitative comparison of segmentation performance for four characteristic
% cases is shown in Fig.~\ref{fig:images}. Our method uses a combination of
% automatically learned intensity and appearance features, which makes it
% inherently robust to noise (see Fig.~\ref{fig:images}a), while still being able
% to detect small isolated lesions (see Fig.~\ref{fig:images}b). Furthermore, our
% method is able to learn a wide spectrum of lesion shapes and appearances from
% training data, which allows our method to correctly identify multiple different
% types of MS lesions. For example, our method was able to correctly identify a
% large T1 black hole that was partially missed by Lesion-TOADS (see
% Fig.~\ref{fig:images}c), which has a known limitation of sometimes
% misclassifying T1 black holes due to different intensity profiles of partially
% overlapping T1 hypointense and T2 hyperintense regions \cite{shiee2010topology}.
% Figure~\ref{fig:images}d shows one of the most challenging cases for our method.
% Very large lesions can extend beyond the size of the receptive field of the CEN,
% which reduces its ability to extract characteristic lesion features.
% Consequently, in some cases our method can underestimate the size of very large
% lesions.

\begin{comment}
\begin{figure*}
\begin{tikzpicture}[node distance=1.5cm and 0.1\textwidth,
  font=\footnotesize, on grid]
  
\node[inner sep=0] (image1) {
  \includegraphics[width=\textwidth]{figures/i10_W9401S01_small_lesion_vessel_outlier}
};
\node[above=of image1,xshift=-0.05\textwidth,align=center] (l3) {%
3-layer CEN\\(T1,FL)};
\node[right=of l3,align=center] (l7) {7-layer CEN-s\\(T1,FL)};
\node[right=of l7,align=center] (l7s){7-layer CEN-s\\(T1,FL,T2,PD)};
\node[right=of l7s,align=center] (ems) {\phantom{g}EMS\phantom{g}\\(T1,T2,PD)};
\node[right=of ems,align=center] (lt) {L-TOADS\\(T1,FL)};
\node[right=of lt,align=center] (LST) {\phantom{g}LST-LGA\phantom{g}\\(T1,FL)};
\node[left=of l3] (pd) {PD-weighted};
\node[left=of pd] (t2) {T2-weighted};
\node[left=of t2] (flair) {\phantom{g}FLAIR\phantom{g}};
\node[left=of flair] (t1) {T1-weighted};

\node[inner sep=0, below=2.4cm of image1] (image2) {
  \includegraphics[width=\textwidth]{figures/i10_W9702S07_poor_contrast_lesions}
};

\node[inner sep=0, below=2.4cm of image2] (image3) {
  \includegraphics[width=\textwidth]{figures/i10_W9902S07_high_LL_improvements}
};

\end{tikzpicture}
\caption{Comparison of segmentation results of different segmentation methods
for 3 subjects. The first and the second row highlight the difficulty}
\end{figure*}
\end{comment}

\begin{comment}
\begin{figure*}
%\centering
\hspace*{-5pt}
\subfloat[Robustness to noise] {
\begin{tikzpicture}[node distance=1.5cm and 0.2\columnwidth,
  font=\footnotesize, on grid]
  
\node[inner sep=0] (image) {
  \includegraphics[width=\columnwidth]{figures/p15s34_robust_to_noise3}
  };
\node[above=of image] (gt) {\phantom{g}Ground truth\phantom{g}};
\node[left=of gt] (flair) {\phantom{g}FLAIR\phantom{g}};
\node[left=of flair] {T1-weighted};
\node[right=of gt,align=center] (cen) {\phantom{g}Our method\phantom{g}};
\node[right=of cen,align=center] {Lesion-\\ TOADS};

\end{tikzpicture}
}
\subfloat[Sensitivity to small isolated lesions.] {
\begin{tikzpicture}[node distance=1.5cm and 0.2\columnwidth,
  font=\footnotesize, on grid]
  
\node[inner sep=0] (image) {
  \includegraphics[width=\columnwidth]{figures/p17s29_noisy_lesionTOADS}};
  \node[above=of image] (gt) {\phantom{g}Ground truth\phantom{g}};
\node[left=of gt] (flair) {\phantom{g}FLAIR\phantom{g}};
\node[left=of flair] {T1-weighted};
\node[right=of gt,align=center] (cen) {\phantom{g}Our method\phantom{g}};
\node[right=of cen,align=center] {Lesion-\\ TOADS};

\draw[green, thick] (-7pt,-3.5pt) circle (3pt);
\begin{scope}[xshift=0.2\columnwidth]
\draw[green, thick] (-7pt,-3.5pt) circle (3pt);
\end{scope}

\end{tikzpicture}
}\\
\hspace*{-5pt}
\subfloat[Example of a T1 black hole that was correctly identified by our
method] {
\begin{tikzpicture}
\node[inner sep=0pt] {
\includegraphics[width=\columnwidth]{figures/p36s30_blackhole}};
\draw[red,thick] (-10pt,12pt) circle (7pt);
\begin{scope}[xshift=-0.2\columnwidth]
\draw[red,thick] (-10pt,12pt) circle (7pt);
\end{scope}
\begin{scope}[xshift=-0.4\columnwidth]
\draw[red,thick] (-10pt,12pt) circle (7pt);
\end{scope}
\begin{scope}[xshift=0.2\columnwidth]
\draw[red,thick] (-10pt,12pt) circle (7pt);
\end{scope}
\begin{scope}[xshift=0.4\columnwidth]
\draw[red,thick] (-10pt,12pt) circle (7pt);
\end{scope}
\end{tikzpicture}
}
\subfloat[Very large lesions can be underestimated] {
\includegraphics[width=\columnwidth]{figures/p54s32_large_lesions}
}

\caption{Four cases illustrating the strengths and limitations of our method
compared to Lesion-TOADS. Our method is inherently robust to noise (a), while
still being able to detect small isolated lesions (b). Furthermore, our method
is able to detect multiple different types of lesions correctly (e.g.,
T1 black holes). However, in some cases our method can underestimate the size
of very large lesions (d).}

\label{fig:images}
\end{figure*}
\end{comment}

\begin{comment}
\subsection{BioMS Data Set}

\begin{table}
\caption{Training results on the BioMS data set using 150 and 250 images for
training.}
\begin{center}
\begin{tabular}{llcc}
\toprule
Pre-training & Fine-tuning & Training DSC & Test DSC \\
\midrule
With dropout & Hessian-free (150) & 67.1 & 61.7 \\
With dropout & Hessian-free (250) & 68.0 & 62.8 \\
Without dropout & Hessian-free (150) & 66.5 & 61.7 \\
With dropout & AdaDelta (150) & 65.9 & 60.6 \\
Without dropout & AdaDelta (150) & 66.0 & 60.6 \\
\bottomrule
\end{tabular}
\end{center}
Notes: No comparison with
state-of-the-art methods are available here. The effect of dropout in the
pre-training phase on the final result is negligible. Hessian-free is slightly
better than AdaDelta and does not require the tuning of parameters. However,
Hessian-free optimization does not support dropout, which might be important
for regularization when the data set size is small or the number of layers and
filters is high.
\end{table}
\end{comment}

% \subsection{On a small data set}
% 
% Include MICCAI challenge results, because it was a comparison with more methods.

\begin{comment}
\subsection{Evaluation of Training Methods}

We have evaluated the impact of different training and initialization methods on
the performance of the trained network using the example of a 7-layer CEN. The
network architecture is summarized in Table~\ref{tab:arch7}. All methods have
hyperparameters, which can be difficult to choose. To find a good set of
hyperparameters for each algorithm (except for Hessian-free), we first trained
the model for 20 epochs with the hyperparameters shown in
Table~\ref{tab:parameters}. We then used the set of parameters that produced the
lowest error on the training set to further fine-tune the model for 500 epochs.
This tuning procedure favors algorithms that are robust to the choice of the
hyperparameters or can make substantial progress within the first few epochs,
which are both desirable properties of a training method for which
time-consuming parameter tuning using a large number of epochs and
cross-validation is not feasible. As is the case for the training on large data
sets on high-resolution 3D medical images. The hyperparameters of the
Hessian-free optimization are very robust to the choice of input data. We found
that tuning these parameters is not necessary. Another difference of the
Hessian-free optimization compared to the other methods is that HF is able to
make more progress within an epoch, albeit at the cost of more time-consuming
updates. To compensate, we trained HF for only 22 epochs. Parameter estimation
and fine-tuning of the model required 2.9 GB of GPU memory and took
approximately 42 hours on a single NVIDIA GeForce GTX 780 graphics card.
However, once the model has been trained, segmentation of an entire 3D image can
be performed in less than half a seconds.

\begin{table}
\caption{Algorithm parameters}%
\label{tab:parameters}
\begin{center}
\begin{tabular}{@{}lp{0.7\columnwidth}@{}}
\toprule
Algorithm & Parameters \\
\midrule
SGD \cite{LeCun1998} & $\text{learning rate} \in \{\num{e-3}, \num{e-4}, \num{e-5},
\num{e-6}\},\newline \text{momentum} = 0.9$ \\

AdaGrad \cite{duchi2011adaptive} & $\alpha \in \{\num{e-4}, \num{e-5},
\num{e-6}, \num{e-7}\}, \epsilon = \num{e-11}$ \\

Adam \cite{kingma2014adam} & $\alpha \in \{\num{3e-5}, \num{e-5}, \num{e-6},
\num{e-7}\}$, $\beta_1 = 0.1$, \newline $\beta_2 = 0.001$, $\epsilon = \num{e-11}$ \\

AdaDelta \cite{zeiler2012adadelta} & $\epsilon \in \{\num{e-9}, \num{e-10},
\num{e-11}\}, \rho= 0.95$
\\

RMSProp \cite{dauphin2015rmsprop} & $\epsilon \in \{\num{e-4}, \num{e-5},
\num{e-6}, \num{e-7}\}, \alpha = 0.9 $\\

Hessian-free \cite{martens2010deep} & $\lambda_0 = 1, \zeta = 0.9$ \\
\bottomrule
\end{tabular}
\end{center}
Note: Please refer to the respective paper for a detailed description of the
hyperparameters.
\end{table}

Figure~\ref{fig:methods} shows a comparison of Dice similarity coefficients
calculated on the test set after training the 7-layer CEN with different
optimization methods as well as with and without pre-training. The results for
SGD are not included in the figure, because we were not able to find a learning
rate for which SGD can make notable progress.
\begin{itemize}
\item SGD is not included because the training method failed to make notable
progress
\item A possible explanation is that the parameters of different layers vary
greatly in magnitude and therefore require different learning rates
\item However, SGD used the same learning rate for all layers and the tuning
phase was not able to find a learning rate that works for all layers
\item In addition, none of the first-order methods were able to train the
CEN without pre-training, which further highlights the difficulty of training deep
CNNs on sparse segmentations.
\item We found that pre-training is crucial to alleviate the difficulty of
training deep CNNs with very unbalanced classes.
\item The training algorithms AdaDelta, Adam, Hessian-free and RMSProp perform
roughly the same with only AdaGrad notably worse than the other methods.
\item Hessian-free optimization was the only method that did not require
pre-training, where the results with pre-training are still slightly better than
without pre-training
\end{itemize}  

\begin{figure}[tb]
\includegraphics[width=\columnwidth]{figures/methods2}
\caption{Comparison of different training methods with pre-training (left) and
without pre-training (right). Hessian-free is the only method that does not
require pre-training to train the model. With pre-training, AdaDelta, Adam,
Hessian-free, and RMSprop consistently learn models that perform significantly
better than lesionTOADS.}
\label{fig:methods}
\end{figure}

A more detailed comparison of the 7-layer CEN trained with different
optimization algorithms and lesionTOADS is summarized in Table~\ref{tab:results1}.
\begin{itemize}
\item TPR is comparable to lesionTOADS, whereby only the CEN trained with
AdaDelta is able to achieve a higher TPR than lesionTOADS
\item Advantage of CEN model is the much reduces number of false positives,
which results in a significantly higher PPV.
\item All methods, except for AdaGrad, outperform lesionTOADS in terms of the
DCS by more than \SI{10}{\percent}. The best performing method achieves a DSC of
\num{54.02} compared to a DSC of \num{40.04} achieved by lesionTOADS on this
challenging data set.
\end{itemize}

\begin{table}
\sisetup{separate-uncertainty=true,detect-weight=true,detect-inline-weight=math}
\begin{center}
\caption{Comparison of the segmentation accuracy of a 7-layer CEN trained with
different algorithms and lesionTOADS}
\label{tab:results1}
\begin{tabular}{@{}lccc@{}}
\toprule
Algorithm & TPR [\%] & PPV [\%] & DSC [\%] \\
\midrule
AdaDelta & \textbf{\num{52.75+-20.31}} & \num{66.58+-20.71} &
\textbf{\num{54.02+-15.24}} \\
AdaGrad & \num{42.14+-18.88} & \num{54.65+-20.24} & \num{43.21+-15.05} \\
Adam & \num{47.41+-18.96} & \textbf{\num{70.33+-20.98}} & \num{51.93+-15.08} \\
Hessian-free & \num{49.21+-18.76} & \num{68.33+-21.12} & \num{52.65+-15.09} \\
RMSprop & \num{47.9+-20.53} & \num{69.12+-20.78} & \num{51.46+-15.77} \\[0.4em]
lesionTOADS & \num{49.83+-14.79} & \num{39.86+-20.95} & \num{40.04+-14.86} \\
\bottomrule
\end{tabular}
\end{center}
Note: The table shows the mean and standard deviation of the true positive rate
(TPR), positive predictive value (PPV), and Dice similarity coefficient (DSC).
All experiments were performed with pre-training and evaluated on the test set.
\end{table}

\subsection{Evaluation of Network Architectures}

In a second experiment, we evaluated the impact of the network architecture on
the segmentation performance. Therefore, we have trained different networks with
varying number of layers and with and without shortcut connections between the
convolutional and deconvolutional pathway. Table~\ref{tab:arch15} shows the
parameters of a 15-layer CEN. We have used the same parameters for the 11-layer,
7-layer, and 3-layer CENs with the respective layers removed.
\begin{itemize}
\item Shortcut connections are increasingly more important with increasing
depth of the network
\item The DSC decreases with depth without shortcut while it increases with
shortcuts
\item After 11 layers, the network performance does not increase on the test set
as the network starts to overfit more
\item Either more training data or better regularization methods are needed to
train very deep models
\end{itemize}

\begin{table}[tb]
\caption{Parameters of the 15 layers of the CENs used to evaluate different
network architectures.}
\label{tab:arch15}
\begin{center}
\begin{tabular}{@{}lccr@{}}
\toprule
Layer type & Kernel size & Filters & \multicolumn{1}{c}{Image size} \\
\midrule
Input & --- & --- & \num{164x206x52x2}\phantom{0} \\
Convolutional & \num{9x9x5x2} & 32 & \num{156x198x48x32} \\
Average Pooling & \num{2x2x2} & --- & \num{78x99x24x32} \\
Convolutional & \num{9x10x5x32} & 32 & \num{70x90x20x32} \\
Average Pooling & \num{2x2x1} & --- & \num{35x45x20x32} \\
Convolutional & \num{8x10x5x32}  & 32 & \num{28x36x16x32} \\
Average Pooling & \num{2x2x1} & --- & \num{14x18x16x32} \\
Convolutional & \num{7x11x9x32}  & 32 & \num{8x8x8x32} \\
Deconvolutional & \num{7x11x9x32}  & 32 & \num{14x18x16x32} \\
Unpooling & \num{2x2x1}       & --- & \num{28x36x16x32} \\
Deconvolutional & \num{8x10x5x32} & 32 & \num{35x45x20x32} \\
Unpooling & \num{2x2x1}       & --- & \num{70x90x20x32} \\  
Deconvolutional & \num{9x10x5x32} & 32 & \num{78x99x24x32} \\
Unpooling & \num{2x2x2} & --- & \num{156x198x48x32} \\
Deconvolutional & \num{9x9x5x32} & 1 & \num{164x206x52x1}\phantom{0} \\
\bottomrule
\end{tabular}
\end{center}
Note: Networks with fewer layer use the same parameters as the 15-layer
CEN, where the missing layers are simply removed.
\end{table}

Analysis of different training set sizes
\begin{itemize}
\item Varying number of training samples
\item With dropout and without (regularization)
\item Performed on most promising architecture
\item Possible outcome: to harsh regularization decreases performance when
enough training data is available but improves performance for small data sets
due to the reduction of overfitting
\end{itemize}
\end{comment}

% \begin{table}
% \caption{Preliminary segmentation results on the Bravo data set.}
% \begin{center}
% \begin{tabular}{lc}
% \toprule
% Method & Training and test DSCs \\
% \midrule
% Input pre-training, 1-layer CNN &  41.63 / 42.76 \\[0.25em]
% No pre-training, 2-layer CNN & 51.88 / 46.93 \\
% Joint pre-training, 2-layer CNN & 44.52 / 43.82 \\
% Input pre-training, 2-layer CNN & 51.02 / 46.58 \\[0.25em]
% Input pre-training, 4-layer CNN & 59.60 / 46.14 \\[0.25em]
% Lesion-TOADS & \phantom{00.}--- / 34.87 \\
% \bottomrule
% \end{tabular}
% \end{center}
% Note: All tests were performed with Hessian-free optimization. Pre-training was
% always performed with dropout.
% \end{table}

\begin{comment}


To allow for a direct comparison with state-of-the-art lesion segmentation
methods, we evaluated our method on the FLAIR, T1-, and T2-weighted MRIs of the
20 publicly available labeled cases from the MS lesion segmentation challenge
2008 \cite{styner20083d}, which we downsampled from the original isotropic voxel
size of \SI{0.5}{\cubic\milli\metre} to an isotropic voxel size of
\SI{1.0}{\cubic\milli\metre}. In addition, we evaluated our method on an
in-house data set from an MS clinical trial of 500 subjects split equally into
training and test sets. The images were acquired from 45 different scanning
sites. For each subject, the data set contains T2- and PD-weighted MRIs with a
voxel size of \SI{0.937x0.937x3.000}{\milli\metre}. The main preprocessing steps
included rigid intra-subject registration, brain extraction, intensity
normalization, and background cropping.

We used a CEN with 32 filters and filter sizes of \num{9x9x9} and \num{9x9x5}
voxels for the challenge and in-house data sets, respectively. Training on a
single GeForce GTX 780 graphics card took between 6 and 32 hours per model
depending on the training set size. However, once the network is trained,
segmentation of trimodal 3D volumes with a resolution of, e.g.,
\num{159x201x155} voxels can be performed in less than one second. As a
rough\footnote{Ciresan et al. used a more complex network that is composed of 11
layers. However, their network was trained on much smaller images, which roughly
compensates for the increased complexity.} comparison, Ciresan et al.
\cite{Ciresan2012} reported that their patch-based method required 10 to 30
minutes to segment a single 2D image with a resolution of \num{512x512} voxels
using four graphics cards, which demonstrates the large speed-ups gained by
processing entire volumes.

% \begin{figure}[tb]
% \centering
% \small
% \def\MRIwidth{0.15\textwidth}
% 
% \begin{tikzpicture} 
% \tikzstyle{leftlabel}=[rotate=90, align=center,overlay,above]
% 
% \matrix [matrix of nodes, nodes={anchor=center, inner sep=1pt}] {
%         &[4pt] FLAIR & T1w & T2w & Ground truth & Our method \\[4pt]
% \node[leftlabel] {CHB\,07\\(DSC\,=\,\SI{60.58}{\percent})}; &
% \includegraphics[width=\MRIwidth]{figures/CHB07-FLAIR-s88} &
% \includegraphics[width=\MRIwidth]{figures/CHB07-T1w-s88} &
% \includegraphics[width=\MRIwidth]{figures/CHB07-T2w-s88} &
% \includegraphics[width=\MRIwidth]{figures/CHB07-gold-s88} &
% \includegraphics[width=\MRIwidth]{figures/CHB07-pred-s88} \\
% \node[leftlabel] {CHB\,04\\(DSC\,=\,\SI{61.37}{\percent})}; &
% \includegraphics[width=\MRIwidth]{figures/CHB04-FLAIR-s85} &
% \includegraphics[width=\MRIwidth]{figures/CHB04-T1w-s85} &
% \includegraphics[width=\MRIwidth]{figures/CHB04-T2w-s85} &
% \includegraphics[width=\MRIwidth]{figures/CHB04-gold-s85} &
% \includegraphics[width=\MRIwidth]{figures/CHB04-pred-s85} \\
% \node[leftlabel] {UNC\,09\\(DSC\,=\,\SI{9.01}{\percent})}; &
% \includegraphics[width=\MRIwidth]{figures/UNC09-FLAIR-s89} &
% \includegraphics[width=\MRIwidth]{figures/UNC09-T1w-s89} &
% \includegraphics[width=\MRIwidth]{figures/UNC09-T2w-s89} &
% \includegraphics[width=\MRIwidth]{figures/UNC09-gold-s89} &
% \includegraphics[width=\MRIwidth]{figures/UNC09-pred-s89} \\
% };
% \end{tikzpicture}
% 
% \caption{Example segmentations of our method for three different subjects from
% the challenge data set. Our method performed well and consistently despite the
% large contrast differences seen between the first two rows. In the third row,
% our method also segmented lesions that have similar contrast, but these regions
% had not been identified as lesions by the manual rater, which highlights the
% difficulty in distinguishing focal lesions from diffuse damage, even for
% experts.}
% 
% \label{fig:segmentation}
% \end{figure}

We evaluated our method on the challenge data set using 5-fold
cross-valida\-tion and calculated the true positive rate (TPR), positive
predictive value (PPV), and Dice similarity coefficient (DSC) between the
predicted segmentations and the resampled ground truth.
Figure~\ref{fig:segmentation} shows a comparison of three subjects from the
challenge data set. The first two rows show the FLAIR, T1w, T2w, ground truth
segmentations, and predicted segmentations of two subjects with a DSC of
\SI{60.58}{\percent} and \SI{61.37}{\percent}. Despite the large contrast
differences between the two subjects, our method performed well and
consistently, which indicates that our model was able to learn features that are
robust to a large range of intensity variations. The last row shows a subject
with a DSC of \SI{9.01}{\percent}, one of the lowest DSC scores from the data
set. Our method segmented lesions that have similar contrast to the other two
subjects, but these regions were not classified as lesions by the manual rater.
This highlights the difficulty of manual lesion segmentation, as the difference
between diffuse white matter pathology and focal lesions is often indistinct. A
quantitative comparison of our method with other state-of-the-art methods is
summarized in Table~\ref{tab:state}. Our method outperforms the winning method
(Souplet et al. \cite{souplet2008}) of the MS lesion segmentation challenge 2008
and the currently best unsupervised method reported on that data set (Weiss et
al. \cite{weiss2013}) in terms of mean TPR and PPV. Our method performs
comparably to a current method (Geremia et al. \cite{geremia2010}) that uses a
carefully designed set of features specifically designed for lesion
segmentation, despite our method having learned its features solely from a
relatively small training set.

\begin{table}[tb]
\def\tabspace{12pt}

\caption{Comparison of our method with state-of-the-art lesion segmentation
methods in terms of mean TPR, PPV, and DSC. Our method performs comparably to
the best methods reported on the MS lesion segmentation challenge data set.}

\label{tab:state}
\centering
\begin{tabular}{l%
@{\hspace{\tabspace}}S[table-format=2.2]
@{\hspace{\tabspace}}S[table-format=2.2]
@{\hspace{\tabspace}}S[table-format=2.2]
}
\toprule
Method & {TPR} & {PPV} & {DSC} \\ 
\midrule
Souplet et al. \cite{souplet2008} & 20.65 & 30.00 & {---} \\ 
Weiss et al. \cite{weiss2013} & 33.00 & 36.85 & 29.05 \\ 
Geremia et al. \cite{geremia2010} & 39.85 & 40.35 & {---}  \\
Our method & 39.71 & 41.38 & 35.52 \\
\bottomrule
\end{tabular}
\end{table}

To evaluate the impact of the training set size on the segmentation performance,
we trained our model on our in-house data set with a varying number of training
samples and calculated the mean DSC on the training and test sets as illustrated
in Fig.~\ref{fig:bioms}. For small training sets, there is a large difference
between the DSCs on the training and test sets, which indicates that the
training set is too small to learn a representative set of features. At around
100 samples, the model becomes stable in terms of test performance and the small
difference between training and test DSCs, indicating that overfitting of the
training data is no longer occurring. With 100 training subjects, our method
achieves a mean DSC on the test set of \SI{57.38}{\percent}, which shows that
the segmentation accuracy can be greatly improved compared to the results on the
challenge data set, when a representative training set is available.

\begin{figure}[tb]
\centering
\includegraphics[width=0.78\textwidth]{figures/train_count}

\caption{Comparison of DSC scores calculated on the training and test sets for
varying numbers of training samples. At around 100 samples, the model becomes
stable in terms of test performance and the small difference between training
and test DSCs, indicating that overfitting of the training data no longer
occurs.}
\label{fig:bioms}
\end{figure}

\end{comment}
