\begin{abstract}

% TODO: Distinguish from the work of other's more

We propose a novel segmentation approach based on deep 3D convolutional encoder
networks with shortcut connections and apply it to the segmentation of multiple
sclerosis (MS) lesions in magnetic resonance images. Our model is a neural
network that consists of two interconnected pathways, a convolutional pathway,
which learns increasingly more abstract and higher-level image features, and a
deconvolutional pathway, which predicts the final segmentation at the voxel
level. The joint training of the feature extraction and prediction pathways
allows for the automatic learning of features at different scales that are
optimized for accuracy for any given combination of image types and segmentation
task. In addition, shortcut connections between the two pathways allow high- and
low-level features to be integrated, which enable segmentation of lesions across
a wide range of sizes. We have evaluated our method on a large data set from an
MS clinical trial, with a comparison of network architectures of different
depths and with and without shortcut connections. The results show that
increasing depth from three to seven layers improves performance, and adding
shortcut connections further increases accuracy. Overall, our method
demonstrates consistently strong segmentation performance across a wide range of
lesion loads, and in a direct comparison outperforms Lesion-TOADS, a widely used
and freely available automatic MS lesion segmentation method. We found the main
limitation of our model to be the underestimation of very large lesions, but
from our depth comparison we expect that this problem could be solved in future
work by adding more network layers.

% The convolutional encoder network approach learns features on entire images
% and predicts segmentations of the same resolution and size as the input
% images, and therefore does not require the redundant computations of
% patch-based methods, where patches typically need to overlap, nor the special
% border handling of other convolutional network methods.
% In contrast to previously used patch-based feature learning approaches and
% similar to recently proposed neural network architectures, our model learns
% features from entire images instead of from patches, which eliminates patch
% selection and redundant calculations at the overlap of neighboring patches and
% thereby speeds up the training. However, unlike previous such approaches, our
% method predicts segmentations of the same resolution and size as the input
% images, which makes it more accurate and eliminates the need for special
% border handling.
% showing that our method is able to segment MS lesions more accurately than our
% previously proposed 3-layer network and Lesion-TOADS, a widely used and freely
% available method for the automatic segmentation of MS lesions.

%In addition, our network also uses a new objective function
%that works well for segmenting underrepresented classes, such as MS lesions.

% We have evaluated our method on the publicly
% available labeled cases from the MS lesion segmentation challenge 2008 data set,
% showing that our method performs comparably to the state-of-the-art. In
% addition, we have evaluated our method on the images of 500 subjects from an MS
% clinical trial and varied the number of training samples from 5 to 250 to show
% that the segmentation performance can be greatly improved by having a
% representative data set.
\end{abstract}

% Note that keywords are not normally used for peerreview papers.
\begin{IEEEkeywords}
Segmentation, multiple sclerosis lesions, magnetic resonance imaging (MRI), deep
learning, convolutional neural networks, machine learning 
\end{IEEEkeywords}
