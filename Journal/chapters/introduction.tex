\section{Introduction}

\IEEEPARstart{M}{ultiple}
sclerosis (MS) is an inflammatory and demyelinating disease of the central
nervous system with pathology that can be observed in vivo by magnetic resonance
imaging (MRI). MS is characterized by the formation of lesions, primarily
visible in the white matter on conventional MRI. Imaging biomarkers based on the
delineation of lesions, such as lesion load and lesion count, have established
their importance for assessing disease progression and treatment effect.
However, lesions vary greatly in size, shape, intensity and location, which
makes their automatic and accurate segmentation challenging. Many automatic
methods have been proposed for the segmentation of MS \mbox{lesions} over the
last two decades \cite{garcia2013review}, which can be classified into
unsupervised and supervised methods. Unsupervised methods do not require a labeled data set for training.
Instead, lesions are identified as an outlier class using, e.g., clustering
methods \cite{souplet2008} or dictionary learning and sparse coding to model
healthy tissue \cite{weiss2013}. Current supervised approaches typically start
with a large set of features, either predefined by the user \cite{geremia2010}
or gathered in a feature extraction step, which is followed by a separate
training step with labeled data to determine which set of features are the most
important for segmentation in the particular domain. For example, Yoo et al.
\cite{yoo2014} proposed performing unsupervised learning of domain-specific
features from image patches from unlabelled data using deep learning. A major
breakthrough for the fully automatic segmentation of medical images using deep
learning comes from the domain of cell membrane segmentation, in which Ciresan
et al. \cite{Ciresan2012} proposed to classify the centers of image patches
directly using a convolutional neural network (CNN) \cite{LeCun1998} without a
dedicated feature extraction step. Instead, features are learned indirectly
within the lower layers of the neural network during training, while the higher
layers can be regarded as performing the classification. In contrast to
unsupervised feature learning, this approach allows the learning of features
that are specifically tuned to the segmentation task. Although deep
network-based feature learning methods have shown great potential for image
segmentation, the time required to train complex patch-based methods can make
the approach infeasible when the size and number of patches are large.

More recently, different CNN architectures
\cite{ronneberger2015,brosch2015,kang2014fully} have been proposed that are able
to learn to segment entire images instead of patches, which removes the need to
select representative patches, eliminates redundant calculations where patches
overlap, and therefore scales up more efficiently with image resolution. Kang et
al. used a fully convolutional neural network to segment crowds in surveillance
videos \cite{kang2014fully}. However, fully convolutional neural networks
predict segmented images of lower resolution than the input images due to the
successive use of convolutional and pooling layers, which both reduce the
dimensionality. To predict segmentations of the same resolution as the input
images, we proposed to use a 3-layer convolutional encoder network (CEN)
\cite{brosch2015} for multiple sclerosis lesion segmentation. The combination of
convolutional \cite{LeCun1998} and deconvolutional \cite{zeiler2011} layers
allows our network to produce segmentations that are of the same resolution as
the input images and are therefore potentially more accurate compared to
segmentations predicted by a fully convolutional neural network. Ronneberger et
al. \cite{ronneberger2015} found that increasing the depth of the CNN, and
thereby increasing the size of the receptive field of a segmented voxel,
increases the abstraction from the input data, which makes the segmentation
method more robust to anatomical and intensity variations. They proposed an
11-layer u-shaped network architecture called u-net, which leverages high-level
and low-level features in order to predict segmentations that are both robust
and accurate. However, the successive application of convolutional, pooling, and
unpooling layers reduces the size of the predicted segmentation by the size of
the receptive field. This requires special handling of the border regions and
limits the maximum size of the receptive field.

% Easier to apply than u-net because it does not need padding or cropping of the
% segmentation for training.
% 
% In this paper, we extend our previous framework to include more layers. Our
% model consists of two pathways, a convolutional pathway which consists of
% alternating convolutional and pooling layers and learns increasingly more
% abstract and robust features, and a deconvolutional pathway which consists of
% alternating deconvolutional and unpooling layers. In order for the segmentation
% to be driven by both low-level and high-level features, we introduce shortcut
% connections between layers of the two pathways. Similar to the u-net
% architecture, this allows the segmentation to be driven by both low-level and
% high-level features. In addition, the formalization of shortcut connections
% allows errors to be propagated through the shortcut connections, which in tern
% allows the learning of low-level features that are tuned for segmentation.

We propose a new convolutional network architecture with 7-layers that combines
the advantages of a 3-layer CEN and an 11-layer u-net. Our network is divided
into two pathways, a convolutional pathway, which consists of alternating
convolutional and pooling layers and learns increasingly more abstract and
robust features, and a deconvolutional pathway, which consists of alternating
deconvolutional and unpooling layers and predicts the final segmentation. In
order for the segmentation to be driven by both low-level and high-level
features, we introduce shortcut connections between layers of the two pathways,
similar to the u-net architecture. In contrast to the u-net architecture, our
network predicts segmentations that have the same resolution as the input images
and therefore does not require special handling of the border of the image. We
have evaluated the segmentation performance of our method on a large data set
from an MS clinical trial and compared our results with Lesion-TOADS
\cite{shiee2010topology}, a widely used freely available method for fully
automatic lesion segmentation. Our results show that the CEN approach is able to
segment MS lesions more accurately than Lesions-TOADS, and that our new CEN
architecture further improves segmentation accuracy.

% There has been a lot of interest in recent years in the machine learning
% community to develop better training methods for CNNs. However, training CNNs
% for medical image segmentation is particularly challenging due to the relatively
% small size of available training sets and the large size of 3D medical images,
% which only allows a few iterations to be used for training and hyperparameter
% tuning. Our second contribution is a comprehensive comparison of different first
% order and second order training method and parameter initialization strategies
% that can serve as a guideline for other researchers for training convolutional
% models for medical image segmentation.

% \item (Analysis of the relationship between regularization and training set
% size to avoid over- and underfitting)

% In this paper, we
% investigate the relationship between training set size and depth of the model on
% the segmentation performance and overfitting and compared the results to a the
% state-of-the-art lesion segmentation method lesion-TOADS, showing that network
% depth can improve segmentation performance for large data sets, but can also
% decrease the performance due to overfitting for smaller training sets, which
% makes network depth a tunable hyperparameter of the model. We've also
% investigated the influence of pre-training on the resulting model, showing that
% although the impact on the performance is minor, it has a big impact on the
% learned features.

% In this paper, we will build on our previous work. Although these networks have
% shown great potential, training of these networks can be difficult. The main
% contribution of this paper is a comprehensive comparison of different training
% strategies and algorithms that can serve as a guideline of how to design and
% train convolutional encoder networks. We have compared different first order and
% second order training methods and found the the right algorithm can have a major
% impact on the performance of the trained network. In addition, we evaluated the
% impact of pre-training and found that pre-training can significantly increase
% classification accuracy by preventing overfitting, if only small data sets are
% available. In addition, we evaluated the impact of the used objective function,
% comparing popular choices with our own objective function.

% We propose a new method for segmenting MS lesions that processes entire MRI
% volumes through a neural network with a novel objective function to
% automatically learn features tuned for lesion segmentation.
% Similar to fully convolutional networks \cite{kang2014fully}, our network
% processes entire volumes instead of patches, which removes the need to select
% representative patches, eliminates redundant calculations where patches overlap,
% and therefore scales up more efficiently with image resolution. This speeds up
% training and allows our model to take advantage of large data sets.
% Our neural network is 

% composed of three layers: an input layer composed of the
% image voxels of different modalities, a convolutional layer \cite{LeCun1998}
% that extracts features from the input layer at each voxel location, and a
% deconvolutional layer \cite{zeiler2011} that uses the extracted features to
% predict a lesion mask and thereby classify each voxel of the image in a single
% operation. The entire network is trained at the same time, which enables feature
% learning to be driven by segmentation performance. The proposed network is
% similar in architecture to a convolutional auto-encoder \cite{masci2011}, which
% produces a lower dimensional encoding of the input images and uses the decoder
% output to measure the reconstruction error needed for training, while our
% network uses the decoder to predict lesion masks of the input images. Due to the
% structural similarity to convolutional auto-encoders, we call our model a
% convolutional encoder network (CEN). Traditionally, neural networks are trained
% by back-propagating the sum of squared differences of the predicted and expected
% outputs. However, if one class is greatly underrepresented, as is the case for
% lesions, which typically comprise less than \SI{1}{\percent} of the image
% voxels, the algorithm would learn to ignore the minority class completely.
% To overcome this problem, we propose a new objective function based on a
% weighted combination of sensitivity and specificity, designed to deal with
% unbalanced classes and formulated to allow stable gradient computations.
