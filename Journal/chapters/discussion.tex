\section{Discussion}

% We have introduced a new method for the automatic segmentation of MS lesions
% based on convolutional encoder networks. The joint training of the feature
% extraction and prediction layers with a novel objective function allows for the
% automatic learning of features that are tuned for a given combination of image
% types and a segmentation task with very unbalanced classes. 

We have presented a new method for the automatic segmentation of MS lesions
based on multi-layer convolutional encoder networks with shortcut connections.
The joint training of the feature extraction and prediction pathways allows for
the automatic learning of features at different scales that are tuned for a
given combination of image types and segmentation task. We have evaluated our
method on a large data set from an MS clinical trial showing that our method is
able to segment MS lesions more accurately than Lesion-TOADS, a widely used and
freely available method for the automatic segmentation of MS lesions. Our method
produces fewer false positives than Lesion-TOADS especially for low lesion loads
where the lesion size is also small.

% Talk about advantages of the new architecture over the old one.

If the size of the lesion extends beyond the size of the receptive field of our
network, we can not accurately detect lesion features. This problem can be
partially alleviated by increasing the depth of the network. Increasing the
depth allows the network to capture appearance features on different scales,
which results in a moderate increase in segmentation accuracy for low lesion
loads and a significant increase in accuracy for very high lesion loads. While
some improvements could have been made by increasing the depth of the network
from three layers to seven layers, very large lesions still pose a major
challenge. For future work, we are planning to experiment with even deeper
layers. Increasing the depth of the network will allow the network to learn
features on an even greater range of different scales, which we expect will
significantly improve the network's ability to segment even very large lesions.
In contrast to other network architectures, the size of the output segmentation
is equal to the size of the input images and therefore independent of the size
of the receptive field. This allows us to design networks that are able to learn
features that cover large parts of the image, or even global features that cover
the entire image. Such features would be able to learn lesion distributions that
would act as an automatically learned prior for lesion segmentation.

% Extensions: global features, prior knowledge, different segmentation tasks

We have presented a very flexible segmentation framework that can be easily
extended. Depth for global features, include
prior knowledge as channels, apply to different segmentation tasks.

A different direction for future work is to incorporate prior knowledge into the
segmentation. We would like to investigate the use of prior tissue
classifications into the learning process. A tissue classification maps can be
incorporate as additional channels to the input images and provide meaningful
guidance for the lesion segmentation process. However, the usefulness of this
approach depends greatly on the quality of the provided tissue classification.
In the presents of lesions, the tissue classification might fail, which in turn
will introduce confounding information into the segmentation process.
