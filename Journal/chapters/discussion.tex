\section{Discussion}

% We have introduced a new method for the automatic segmentation of MS lesions
% based on convolutional encoder networks. The joint training of the feature
% extraction and prediction layers with a novel objective function allows for the
% automatic learning of features that are tuned for a given combination of image
% types and a segmentation task with very unbalanced classes. 

We have presented a new method for the automatic segmentation of MS lesions
based on multi-layer convolutional encoder networks with shortcut connections.
The joint training of the feature extraction and prediction pathways allows for
the automatic learning of features at different scales that are tuned for a
given combination of image types and segmentation task. We have evaluated our
method on a large data set from an MS clinical trial showing that our method is
able to segment MS lesions more accurately than Lesion-TOADS, a widely used and
freely available method for the automatic segmentation of MS lesions. The gains
in accuracy are mostly due to the reduction of false positives especially for
low lesion loads where the lesion size is also small.

The most challenging type of lesions to segment for our method are very large
lesions, which can extend beyond the receptive field of a particular voxel. This
reduces the network's ability to extract appearance features that would help
the identification of lesion voxels. For future work, we are planning to
investigate the use of deeper networks. Increasing the depth would allow the
network to learn features on a wider range of scales, which we expect will
significantly improve the network's ability to segment even very large lesions.
In contrast to fully convolutional networks and the u-net architecture, the size
of the output segmentation of a CEN is independent of the size of the receptive
field, which allows us to design networks that are able to learn features that cover
large parts of the image, or even global features that cover the entire image.
Such features would be able to estimate the global distribution of lesions and
could act as an automatically learned lesion prior, further improving the
robustness of our method.

% Extensions: global features, prior knowledge, different segmentation tasks

% We have presented a very flexible segmentation framework that can be easily
% extended. Depth for global features, include
% prior knowledge as channels, apply to different segmentation tasks.

We have presented a very flexible segmentation framework that can be easily
extended. One such extension could be to incorporate prior knowledge about the
tissue type of each voxel into the segmentation procedure. Therefore, each image
needs to be segmented into cerebrospinal fluid, gray matter, and white matter as
part of the pre-processing pipeline. The probabilities of each tissue class can
then be added as an additional channel to the input units of the CEN, which
allows the CEN to take advantage of intensity information from different
modalities and prior knowledge about the tissue class to carry out the
segmentation.
%
%For example, we have observed the presence of false positives in
%the region of the ventricles in some rare cases. Knowledge about the extend of
%the vectricles can help to reduce this type of misclassification.
%
However, the benefit from using a prior tissue classification depends on the
accuracy of the segmentation algorithm. In the presence of lesions, a
segmentation method that was designed for healthy tissue might misclassify
regions affected by lesions, which in turn could confound the segmentation. In
addition, our method can be applied to other segmentation tasks. Although we
have only focused on the segmentation of MS lesions in this paper, our method
does not make any assumptions specific to MS lesion segmentation. The features
required to carry out the segmentation are solely learned from training data,
which allows our method to be used to segment different types of pathology or to
perform structural segmentation when a suitable training set is available.
