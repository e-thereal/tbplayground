\section{Discussion}

% We have introduced a new method for the automatic segmentation of MS lesions
% based on convolutional encoder networks. The joint training of the feature
% extraction and prediction layers with a novel objective function allows for the
% automatic learning of features that are tuned for a given combination of image
% types and a segmentation task with very unbalanced classes. 

We have extended our previous approach by adding more layers and shortcut
connections. Evaluated it on a large data set and showed that CEN improves
over Lesion-TOADS. Also showed that the extensions further improve segmentation
accuracy.

We have evaluated our method on a large data set from an MS clinical trial
showing that our method presents a significant improvement in accuracy over
Lesion-TOADS, a widely used and freely available method for the automatic
segmentation of MS lesions. Our method produces much fewer false positives than
Lesion-TOADS especially when the lesion load is low. If the size of the lesion
extends beyond the size of the receptive field of our network, we can not
accurately detect lesion features. This problem can be partially alleviated by
increasing the depth of the network. Increasing the depth allows the network to
capture appearance features on different scales, which results in a moderate
increase in segmentation accuracy for low lesion loads and a significant
increase in accuracy for very high lesion loads. While some improvements could
have been made by increasing the depth of the network from three layers to seven
layers, very large lesions still pose a major challenge. For future work, we are
planning to experiment with even deeper layers. Increasing the depth of the
network will allow the network to learn features on an even greater range of
different scales, which we expect will significantly improve the network's
ability to segment even very large lesions. In contrast to other network
architectures, the size of the output segmentation is equal to the size of the
input images and therefore independent of the size of the receptive field. This
allows us to design networks that are able to learn features that cover large
parts of the image, or even global features that cover the entire image. Such
features would be able to learn lesion distributions that would act as an
automatically learned prior for lesion segmentation.

A different direction for future work is to incorporate prior knowledge into the
segmentation. We would like to investigate the use of prior tissue
classifications into the learning process. A tissue classification maps can be
incorporate as additional channels to the input images and provide meaningful
guidance for the lesion segmentation process. However, the usefulness of this
approach depends greatly on the quality of the provided tissue classification.
In the presents of lesions, the tissue classification might fail, which in turn
will introduce confounding information into the segmentation process.

We are also interested in the application of this method in other areas.
Although we have only shown the appropriateness for lesion segmentation, the
method does not make any assumptions inherit to lesion segmentation. We believe
that this method can also be readily applied to a variate of medical
segmentation problems, provided that a representative data set is available. It
could potentially be used for whole brain parcelletation without the need of an
anatomical atlas. Low-level features will provide useful to carry out an initial
tissue classification, while high-level and global features can guide the
segmentation into different sections of the brain. A method bases on
convolutional encoder networks would be able to perform brain parcelation
without the need for registration. It removes the need for a representative
atlas, which can be difficult to obtain for brain under the influence of
pathology.
