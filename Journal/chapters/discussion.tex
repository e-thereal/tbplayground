\section{Discussion}

% We have introduced a new method for the automatic segmentation of MS lesions
% based on convolutional encoder networks. The joint training of the feature
% extraction and prediction layers with a novel objective function allows for the
% automatic learning of features that are tuned for a given combination of image
% types and a segmentation task with very unbalanced classes. 

We have presented a new method for the automatic segmentation of MS lesions
based on multi-layer convolutional encoder networks with shortcut connections.
The joint training of the feature extraction and prediction pathways allows for
the automatic learning of features at different scales that are tuned for a
given combination of image types and segmentation task. We have evaluated our
method on a large data set from an MS clinical trial showing that our method is
able to segment MS lesions more accurately than Lesion-TOADS, a widely used and
freely available method for the automatic segmentation of MS lesions. The gains
in accuarcy are mostly due to the reduction of false positives especially for
low lesion loads where the lesion size is also small.

The most challenging type of lesions to segment for our method are very large
lesions, which can extend beyond the receptive field of a particular voxel. This
reduces the network's ability to identify appearance features that are required
for the accurate segmentation of lesions. For future work, we are planning to
investigate the use of deeper networks. Increasing the depth would allow the
network to learn features on a wider range of scales, which we expect will
significantly improve the network's ability to segment even very large lesions.
In contrast to fully convolutional networks and the u-net architecture, the size
of the output segmentation of a CEN is independent of the size of the receptive
field, which allows us to design networks that are able to learn features that cover
large parts of the image, or even global features that cover the entire image.
Such features would be able to estimate the global distribution of lesions and
could act as an automatically learned lesion prior.

% Extensions: global features, prior knowledge, different segmentation tasks

% We have presented a very flexible segmentation framework that can be easily
% extended. Depth for global features, include
% prior knowledge as channels, apply to different segmentation tasks.

Another direction for future work is to incorporate prior knowledge directly
into the segmentation pipeline. We would like to investigate the use of prior
tissue classifications into the learning process. A tissue classification maps can be
incorporate as additional channels to the input images and provide meaningful
guidance for the lesion segmentation process. However, the usefulness of this
approach depends greatly on the quality of the provided tissue classification.
In the presents of lesions, the tissue classification might fail, which in turn
will introduce confounding information into the segmentation process.
