\section{Discussion}

We have introduced a new method for the automatic segmentation of MS lesions
based on convolutional encoder networks. The joint training of the feature
extraction and prediction layers with a novel objective function allows for the
automatic learning of features that are tuned for a given combination of image
types and a segmentation task with very unbalanced classes.
We have evaluated our method on two data sets showing that approximately 100
images are required to train the model without overfitting but even when only a
relatively small training set is available, the method still performs comparably
to the state-of-the-art algorithms. For future work, we plan to increase the
depth of the network, which would allow the learning of a set of hierarchical
features. This could further improve segmentation accuracy, but may require
larger training sets. We would also like to investigate the use of different
objective functions for training based on other measures of segmentation
performance.

\begin{itemize}
\item What else is new? 
\item Could incorporate lesion prior in form of white matter lesion mask.
However, the presence of lesions might compromise the correct classification of
tissues, which in turn could further disturb the segmentation instead of help
it.
\end{itemize}
