\usepackage{ifthen}

% Charater sets, fonts and basics
\usepackage{fixltx2e}
\usepackage[latin1]{inputenc}
\usepackage[T1]{fontenc}

\usepackage[english]{babel}
\usepackage[english=british]{csquotes}

%\usepackage{lmodern}
\usepackage[sc, osf]{mathpazo}
\renewcommand*\sfdefault{uop}
\usepackage[scaled=.9]{luximono}
\linespread{1.10}
\usepackage{microtype}
\makeatletter
\def\MT@register@subst@font{\MT@exp@one@n\MT@in@clist\font@name\MT@font@list
   \ifMT@inlist@\else\xdef\MT@font@list{\MT@font@list\font@name,}\fi}
\makeatother 

% Math stuff
%\usepackage{dsfont,amsmath,amssymb,latexsym,soul}
\usepackage{amsmath,mathtools}
\usepackage{amsbsy,dsfont,soul,calc}
\DeclareMathOperator{\sinc}{sinc}
\DeclareMathOperator{\sigm}{sigm}
\DeclareMathOperator{\fft}{fft}
\DeclareMathOperator{\ifft}{ifft}
\sodef\so{}{.14em}{.4em plus.1em minus .1em}{.4em plus.1em minus .1em}

\usepackage[autokw]{svn-multi}
\svnid{$Id: preamble.tex 41 2012-08-01 17:29:41Z tbrosch $}

% Tikz
\usepackage{atbegshi}
\usepackage{tikz}
\usetikzlibrary{intersections,decorations.pathreplacing,fit,calc,positioning,shapes.geometric,matrix}
\tikzstyle{every node}+=[font=\sffamily\small]
\pgfdeclarelayer{background}
\pgfsetlayers{background,main}

%\usetikzlibrary{external}
%\tikzexternalize[prefix=tikzfigures/temp/] % activate
%\tikzset{external/export=false}
\newcommand{\tikzexternal}[2][false]{%
%\tikzset{external/export=true, external/remake next=#1}%
%\tikzsetnextfilename{#2}%
}

% Special Table hacks
\usepackage{array,booktabs,dcolumn,rotating,multirow,etoolbox}

\robustify\bfseries % requires etoolbox
\newcommand{\minitab}[2][l]{\begin{tabular}{#1}#2\end{tabular}}

\newcolumntype{d}[1]{%
>{\DC@{.}{.}{#1}}l<{\DC@end}%
}

\newcolumntype{.}{D{.}{.}{-1}}
\newcolumntype{L}[1]{>{\raggedright\arraybackslash}p{#1}}

%\usepackage[sf,SF]{subfigure}
\usepackage[font={sf,small},labelfont=md,subrefformat=parens]{subfig}

% Units
\usepackage{xfrac}
\usepackage[alsoload=binary]{siunitx}
\sisetup{
  seperr,
  %repeatunits=false,
  product-units = single,
  trapambigerr=false,
  trapambigrange=false,
  %tophrase={{ bis }},
  valuemode=math,
  unitmode=math,
  quotient-mode=fraction,
  fraction-function=\sfrac
}

\usepackage{tocstyle}
\newtocstyle[KOMAlike][leaders]{thesis}{
  %\settocfeature[-1]{dothook}{\bfseries}
  %\settocfeature[0]{dothook}{\bfseries}
}
\usetocstyle{thesis}

% Enumerates
\usepackage{paralist}

\usepackage{varioref}
\usepackage[
    pdftex,
	pdfpagemode=UseOutlines,
	pdfdisplaydoctitle=true,
	pdflang=en,
	pagebackref,
	bookmarksopen=true,
	bookmarksopenlevel={1}
]{hyperref}

% % Change Layout of Backref
\renewcommand*{\backref}[1]{%
  % default interface
  % #1: backref list
  %
  % We want to use the alternative interface,
  % therefore the definition is empty here.
}%
\renewcommand*{\backrefalt}[4]{%
  % alternative interface
  % #1: number of distinct back references
  % #2: backref list with distinct entries
  % #3: number of back references including duplicates
  % #4: backref list including duplicates
  \ifnum#1=0%
  \footnotesize\sffamily\color{red!50!gray}(not cited)%
  \else%
  \ifnum#1=1%
  \footnotesize\sffamily\color{gray}(cited on page~#2)%
  \else%
  \footnotesize\sffamily\color{gray}(cited on pages~#2)%
  \fi%
  \fi%
}

% Extraabstand fuer Fussnoten
\setlength{\skip\footins}{2em}

% Handle floats right
%\usepackage[above,below]{placeins}
%\usepackage{placeins}

\usepackage[version=3]{mhchem}

%\bibliographystyle{plain}
\usepackage{bibgerm}
\usepackage[sort]{natbib}
\bibliographystyle{gerapali}
%\bibliographystyle{gerplain}

% Listings
\usepackage{listings}

% Color definition
% Define special colors
\definecolor{listcomment}{rgb}{0.325,0.761,0.259}
\definecolor{listkeyword}{rgb}{0.243,0.306,0.408}
\definecolor{listnumbers}{gray}{0.65}
\definecolor{listlightgray}{gray}{0.955}
\definecolor{shadecolor}{named}{listlightgray}
\definecolor{stringcolor}{rgb}{0.341,0.243,0.416}

% Setup the listings environment
\lstset{
  frame = tb,
  framerule = 0.25pt,
  float,
  backgroundcolor={\color{listlightgray}},
  basicstyle = {\ttfamily\footnotesize},
  keywordstyle = {\ttfamily\color{listkeyword}\textbf},
  identifierstyle = {\ttfamily},
  commentstyle = {\ttfamily\color{listcomment}\textit},
  stringstyle = {\ttfamily\color{stringcolor}},
  showstringspaces = false,
  showtabs = false,
  numbers = left,
  numbersep = 6pt,
  numberstyle={\ttfamily\color{listnumbers}},
  tabsize = 2,
  language=C++,
  floatplacement=tb
  captionpos=b,
  escapeinside={<@}{@>},
  abovecaptionskip=7pt,
  morekeywords={__global__, __host__, __device__, uint, texture, float4, fmatrix4}
}

\usepackage[vlined,linesnumbered]{algorithm2e}
\SetNlSty{}{}{}
\SetArgSty{}
\DontPrintSemicolon

\usepackage[acronym,nonumberlist,toc,nomain]{glossaries}
\renewcommand*{\glspostdescription}{}
\setlength{\glsdescwidth}{0.7\textwidth}
\glossarystyle{super}
\renewcommand*{\glsgroupskip}{}

\AtBeginDocument{\labelformat{lstlisting}{Listing #1}}
\AtBeginDocument{\labelformat{chapter}{Chap\mbox{ter #1}}}
\AtBeginDocument{\labelformat{section}{Sec\mbox{tion #1}}}
\AtBeginDocument{\labelformat{subsection}{Sec\mbox{tion #1}}}
\AtBeginDocument{\labelformat{equation}{(#1)}}
\AtBeginDocument{\labelformat{table}{T\mbox{able #1}}}
\AtBeginDocument{\labelformat{figure}{F\mbox{igure #1}}}
\AtBeginDocument{\labelformat{Definition}{D\mbox{efinition #1}}}
%\AtBeginDocument{\labelformat{Satz}{Satz~#1}}
%\AtBeginDocument{\labelformat{Beispiel}{Beispiel~#1}}
%\AtBeginDocument{\labelformat{Beweis}{Beweis~#1}}

% Headings and footer
\usepackage{scrpage2}
\pagestyle{scrheadings}
\clearscrheadfoot
\ohead{\headmark}
\ofoot[\pagemark]{\pagemark}

\setheadsepline{.4pt}
\addtokomafont{pageheadfoot}{\normalfont\sffamily\small}
\addtokomafont{caption}{\sffamily\small}
\addtokomafont{captionlabel}{\sffamily\bfseries\small}
\addtokomafont{pagefoot}{\normalsize}
%\addtokomafont{chapter}{\glsresetall}
%\addtokomafont{page}{\normalsize}
%\automark[subsection]{section}
\automark{chapter}

%\addtokomafont{section}{
%{\tikzset{external/export=false}\tikz \fill[left color=primary-4,right color=white] (0,0) rectangle (\textwidth,1pt);}\\*}
\addtokomafont{dictumtext}{\normalfont\itshape}
\addtokomafont{dictumauthor}{\normalfont}

%\newcommand{\intro}[1]{\paragraph{Introduction}{\itshape #1}}
\newcommand{\intro}[1]{\noindent {\itshape #1}}
\newcommand{\question}[1]{\paragraph{Question}{\itshape #1}}
\newcommand{\answer}{\paragraph{Answer}}

\newcommand{\qa}[1]{\question{#1}\answer}


\usepackage[textwidth=3.3cm, textsize=footnotesize,
colorinlistoftodos, shadow]{todonotes}
\usepackage{textcomp,xspace}
\usepackage{dsfont,amsmath,amssymb,amsbsy}
\usepackage{setspace,comment}

\newcommand{\vect}[1]{\mathbf{#1}}
\newcommand{\R}{\ensuremath{\mathds{R}}}
\newcommand{\Q}{\ensuremath{\mathds{Q}}}
\newcommand{\N}{\ensuremath{\mathds{N}}}
\newcommand{\I}{\ensuremath{\mathds{I}}}
\newcommand{\E}{\ensuremath{\mathds{E}}}
\newcommand{\Z}{\ensuremath{\mathds{Z}}}
\newcommand{\D}{\ensuremath{\mathds{D}}}
\newcommand{\W}{\ensuremath{\mathds{W}}}
\newcommand{\B}{\ensuremath{\mathds{B}}}
\newcommand{\Sn}{\ensuremath{\mathds{S}}}

\newcommand{\thetas}{\boldsymbol{\theta}}
\newcommand{\deltas}{\boldsymbol{\delta}}
\newcommand{\data}{\mathcal{D}}
\newcommand{\Norm}{\ensuremath{\mathcal{N}}}
\newcommand{\given}{\mid}

% Define a counter for the inserted todonotes.
\newcounter{todoListItems}
\newcommand{\todoTrans}[2][ ]{%
	% Increment counter
	\addtocounter{todoListItems}{1}%
	\todo[%
		caption={\protect\hypertarget{todo\thetodoListItems}{}#2},
		#1]
	{
		\sffamily #2 \hfill
		\hyperlink{todo\thetodoListItems}{$\uparrow$}
	}
}

\newcommand{\addref}{\todoTrans[color=blue!40]{Needs reference.}\xspace}
\newcommand{\addcontent}[1]{\todoTrans[color=green!40, inline]{#1}\xspace}
\newcommand{\rewrite}[1]{\todoTrans[color=red!40]{#1}\xspace}
\newcommand{\clarify}[1]{\todoTrans[color=orange!40]{#1}\xspace}
\newcommand{\thoughts}[1]{\todoTrans[color=orange!40,
inline]{\textbf{Thoughts:} #1}\xspace}


\mathchardef\ordinarycolon\mathcode`\:
\mathcode`\:=\string"8000
\begingroup \catcode`\:=\active
  \gdef:{\mathrel{\mathop\ordinarycolon}}
\endgroup

\usepackage{datetime}

\newdateformat{thesisdate}{%
\monthname[\THEMONTH] \THEYEAR}

% Schuster jungen und hurenkinder
\clubpenalty = 10000
\widowpenalty = 10000
\displaywidowpenalty = 10000

\newcommand{\vs}{\emph{vs}\xspace}
