\chapter{Training on Medical Images: Training in the Frequency Domain}

Deep learning has traditionally been computationally expensive and advances in
training methods have been the prerequisite for expanding its application to a
variety of image classification problems. The development of layer-wise training
methods \citep{Hinton2006b} greatly improved the efficiency of the training of
deep belief networks (DBNs), which has made feasible the use of large sets of
small images (e.g. \num{28x28}), such as those used for hand-written digit
recognition. Subsequently, new directions for speeding up the training of deep
models were opened with the advance of programmable graphics cards (GPUs), which
can perform thousands of operations in parallel. \Citet{Raina2009} demonstrated
that by using graphics cards, training of restricted Boltzmann machines (RBMs)
on small image patches (e.g. \num{24x24}) can be performed up to \num{70} times
faster than on the CPU, facilitating the application to larger training sets.
However, the number of trainable weights of a DBN increases greatly with the
resolution of the training images, which can make training on large images
impracticable. In order to scale DBNs to high-resolution images,
\citet{lee2009,lee2011} introduced the convolutional deep belief network
(convDBN), a deep generative model that uses weight sharing to reduce the number
of trainable weights. They showed that a convDBN can be used to classify images
with a resolution up to \num{200x200} pixels. To speed up the training of
convolutional neural networks (CNNs) on high-resolution images,
\citet{Krizhevsky2012} replaced traditional convolutions of the first layer of
their CNN with strided convolutions, a type of convolution that shifts the
filter kernel as a sliding window with a fixed step size or stride greater than
one. Through the use of strided convolutions, the number of hidden units in each
convolutional layer is greatly reduced, which reduces both training time and
required memory. Using a highly optimized GPU implementation of convolutions,
they were able to train a CNN on images with a resolution of \num{256x256}
pixels, achieving state-of-the-art performance on the ILSVRC-2010 and
ILSVRC-2012 competitions \citep{Krizhevsky2012}. An alternative approach was
proposed by \citet{Mathieu2013} who sped up the training of CNNs by calculating
convolutions between batches of images and filters using fast Fourier transforms
(FFTs), albeit at the cost of additional memory required for storing the
filters.

In this paper, we detail our training algorithm and GPU implementation in full,
with a much more thorough analysis of the running time on high-resolution 2D
images (\num{512x512}) and 3D volumes (\num{128x128x128}), showing speed-ups of
up to 8-fold and 200-fold, respectively. Our proposed method performs training
in the frequency domain, which replaces the calculation of time-consuming
convolutions with simple element-wise multiplications, while adding only a small
number of FFTs. In contrast to similar FFT-based approaches
\citep[e.g.,][]{Mathieu2013}, our method does not use batch processing of the
images as a means to reduce the number of FFT calculations, but rather minimizes
FFTs even when processing a single image, which significantly reduces the
required amount of scarce GPU memory. We show that our method can be efficiently
implemented on multiple graphics cards, further improving the runtime
performance over other GPU-accelerated training methods. In addition, we
formalize the expression of the strided convolutional DBN (sconvDBN), a type of
convDBN that uses strided convolutions to speed up training and reduce memory
requirements, in terms of stride-1 convolutions, which enables the efficient
training of sconvDBNs in the frequency domain.

% TODO: Follow up with how to train them in the medical domain.
% Introduction and motivation. Talks about special challenges. Training in the
% frequency domain is important and a main contribution of this thesis. This will
% be about my algorithm, how it works for convRBMs and CNNs. Especially the tricks
% needed for strided convolutions.

\section{Algorithm}

% Mapping in the frequency domain. Minimize transforms. Also memory
% considerations and the likes. Also trick to do strided convolutions in the
% frequency domain.

\subsection{Training in the Spatial Domain}

% Short revision of how training works. Put all the pieces together to make it
% an algorithm. Make sure that I do this for convRBMs

To train an sconvRBM on a set of images, the
weights and bias terms can be learned by CD. During each iteration of the
algorithm, the gradient of each parameter is estimated and a gradient step with
a fixed learning rate is applied. The gradient of the filter weights can be
approximated by
\begin{equation}
\Delta \vect{W}^{ij} \approx \frac{1}{N}(\vect{V}^i_n*\tilde{\vect{h}}^j_n -
\vect{V}'^i_n*\tilde{\vect{h}}_n'^j)
\label{eq:convgra}
\end{equation}
where $\vect{V}_n, n \in [0,N-1]$ are reindexed images from the training set,
$\vect{h}_n^j$ and $\vect{h}'^j_n$ are samples drawn from $p(\vect{h}^j \given
\vect{V}_n)$ and $p(\vect{h}^j \given \vect{V}'_n)$, and $\vect{V}'^i_n =
\E[\vect{V}^i \given \vect{h}_n]$.
To apply the model to real-valued data like certain types of images, the
visible units can be modeled as Gaussian units. When the visible units are
mean--centered and standardized to unit variance, the expectation of the visible
units is given by
\begin{equation} 
\E[\vect{V}^i \given \vect{h}] = \textstyle\sum_j \vect{W}^{ij}*\vect{h}^j + b_i
\label{eq:convvis}
\end{equation}
A binary hidden unit can only encode two states. In order to increase the
expressive power of the hidden units, we use noisy rectified linear units as the
hidden units, which have been shown to improve the learning performance
of RBMs \citep{Nair2010}. The hidden units can be sampled with
\begin{align} 
\vect{h}^j &\sim \max(0, \mu^j + \Norm(0, \sigm(\mu^j))) \\
\mu^j &= \textstyle\sum_i\tilde{\vect{W}}^{ij}*\vect{V}^i +c_j
\label{eq:convhid}
\end{align} 
where $\Norm(0, \sigma^2)$ denotes Gaussian noise. The learning algorithm in the
spatial domain is summarized in Figure~\ref{alg:spatial}.

\begin{figure}[t!]
\hspace{-1.75em}
\subfloat[Training in the spatial domain]{
\label{alg:spatial}
\begin{minipage}{0.495\linewidth}
\footnotesize
\begin{algorithm}[H]
\setstretch{1.25}
%\renewcommand{\baselinestretch}{1.25}
%\selectfont
\SetKwInOut{Input}{input}
\SetKwInOut{Output}{output}
\Input{Images $\data = \{\vect{V}_n \given n \in [0, N-1]\}$}
\Output{Weights gradient $\Delta \vect{W}$}
$\Delta \vect{W} = 0$\;
\ForEach{image $\vect{V} \in \data$} {
  $\vect{V}' = 0$\;
  \For{$j = 0$ \KwTo $N_\text{k}-1$} {
    $\mu^j = \sum_i \tilde{\vect{W}^{ij}} * \vect{V}^i + c_j$\;
    \mbox{}\\
    \BlankLine
    $h^j \sim \max(0, \mu^j + \Norm(0, \sigm(\mu^j)))$\;
    \mbox{}\\
    \For{$i = 0$ \KwTo $N_\text{C}-1$} {
      $\Delta \vect{W}^{ij} = \Delta \vect{W}^{ij} + \tilde{\vect{h}}^j *
      \vect{V}^i$\;
      $\vect{V}'^i = \vect{V}'^i + \vect{W}^{ij} * \vect{h}^j$\;
    }
  }
  $\forall i \colon \vect{V}'^i = \vect{V}'^i + b_i$\;
  \For{$j = 0$ \KwTo $N_\text{k}-1$} {
    $\mu'^j = \sum_i \tilde{\vect{W}^{ij}} * \vect{V}'^i + c_j$\;
    \mbox{}\\
    $\vect{h}'^j \sim \max(0, \mu'^j +$\\ \hfill $\Norm(0, \sigm(\mu^j)))$\;
    \mbox{}\\
    \For{$i = 0$ \KwTo $N_\text{C}-1$} {
      $\Delta \vect{W}^{ij} = \Delta \vect{W}^{ij} - \tilde{\vect{h}}'^j *
      \vect{V}'^i$\;
    }
  }
}
\end{algorithm}
\end{minipage}
}
\subfloat[Training in the frequency domain]{
\label{alg:frequency}
\begin{minipage}{0.495\linewidth}
\footnotesize
\begin{algorithm}[H]
\setstretch{1.25}
%\renewcommand{\baselinestretch}{1.25}
%\selectfont
\SetKwInOut{Input}{input}
\SetKwInOut{Output}{output}
\Input{Images $\hat{\data} = \{\hat{\vect{V}}_n \given n \in [0, N-1]\}$}
\Output{Weights gradient $\Delta\hat{\vect{W}}$}
$\Delta\hat{\vect{W}} = 0$\;
\ForEach{image $\hat{\vect{V}} \in \hat{\data}$} {
  $\hat{\vect{V}}' = 0$\;
  \For{$j = 0$ \KwTo $N_\text{k} - 1$} {
    $\hat{\mu}^j = \sum_i \overline{\hat{\vect{W}}^{ij}} \cdot
    \hat{\vect{V}}^i$\; $\mu^j = \ifft(\hat{\mu}^j) + c_j$\;
    $\vect{h}^j \sim \max(0, \mu^j + \Norm(0, \sigm(\mu^j)))$\;
    $\hat{\vect{h}}^j = \fft(\vect{h}^j)$\;
    \For{$i = 0$ \KwTo $N_\text{C}-1$} {
      $\Delta\hat{\vect{W}}^{ij} = \Delta\hat{\vect{W}}^{ij} +
      \overline{\hat{\vect{h}}^j} \cdot \hat{\vect{V}}^i$\;
      $\hat{\vect{V}}'^i = \hat{\vect{V}}'^i + \hat{\vect{W}}^{ij} \cdot
      \hat{\vect{h}}^j$\;
    }
  }
  $\forall i \colon \hat{V}_{0,0}'^i = \hat{V}_{0,0}'^i +
  N_\text{V}^2b_i$\;
  \For{$j = 0$ \KwTo $N_\text{k}-1$} {
    $\hat{\mu}'^j = \sum_i \overline{\hat{\vect{W}}^{ij}} \cdot
    \hat{\vect{V}}'^i$\;
    $\mu'^j = \ifft(\hat{\mu}'^j) + c_j$\;
    $\vect{h}'^j \sim \max(0, \mu'^j +$ \\ \hfill $\Norm(0, \sigm(\mu^j)))$\;
    $\hat{\vect{h}}'^j = \fft(\vect{h}'^j)$\;
    \For{$i = 0$ \KwTo $N_\text{C}-1$} {
      $\Delta\hat{\vect{W}}^{ij} = \Delta\hat{\vect{W}}^{ij} -
      \overline{\hat{\vect{h}}'^j} \cdot \hat{\vect{V}}'^i$\;
    }
  }
}
\end{algorithm}
\end{minipage}
}
\caption{Comparison of training algorithms of convRBMs in (a) the
spatial and (b) the frequency domain. Training in the frequency domain replaces the
$5N_\text{k}N_\text{C}$ convolutions required in the spatial domain with simple
element-wise multiplications, while adding only $4N_\text{k}$ Fourier
transforms. The other operations are equivalent in both domains.}
\label{fig:algorithms}
\end{figure}

\subsection{Training in the Frequency Domain}

The computational bottleneck of the training algorithm in the spatial domain is
the calculation of convolutions, which needs to be performed $5 \times
N_\text{C} \times N_\text{k}$ times per iteration. To speed up the calculation
of convolutions, we perform training in the frequency domain, which maps the
convolutions to simple element-wise multiplications. This is especially
important for the training on 3D images due to the relatively large number of
weights of a 3D kernel compared to 2D. To minimize the number of Fourier
transforms, we map all operations needed for training to the frequency domain
whenever possible, which allows the training algorithm to stay almost entirely
in the frequency domain. All of the scalar operations needed for training
(multiplications and additions) can be readily mapped to the frequency domain
because the Fourier transform is a linear operation. Another necessary operation
is the flipping of a convolutional kernel $\tilde{w}(a) = w(-a)$, which can be
expressed by element-wise calculation of the complex conjugate; this follows
directly from the time-reversal property of the Fourier transform and the
reality condition $\hat{f}(-\xi)=\overline{\hat{f}(\xi)}$. Using the
aforementioned mappings, equations \eqref{eq:convgra} to \eqref{eq:convhid} can
be rewritten as
\begin{align}
\Delta \hat{\vect{W}}^{ij} &= \hat{\vect{V}}^i\cdot\overline{\hat{\vect{h}}^j}
- \hat{\vect{V}}'^i \cdot \overline{\hat{\vect{h}}'^j} \\
\E[\hat{V}_{xy}^i \given \vect{\hat{h}}] &=
\begin{dcases}
\textstyle\sum_j\hat{W}_{xy}^{ij}\hat{h}_{xy}^j + N_\text{V}^2 b_i &
\text{for } x,y = 0
\\
\textstyle\sum_j\hat{W}_{xy}^{ij}\hat{h}_{xy}^j & \text{for } x,y \neq 0
\end{dcases} \\
\hat{\vect{h}}^j &\sim \F\left(\max(0, \F^{-1}(\hat{\mu}^j) + c_j + 
\Norm(0, \sigma^2))\right) \\
\hat{\mu}^j
&=\textstyle\sum_i\overline{\hat{\vect{W}}^{ij}}\cdot\hat{\vect{V}}^i
\end{align}
where $\sigma^2 = \sigm(\F^{-1}(\hat{\mu}^j)+c_j)$, $\hat{x} = \F(x)$ denotes
$x$ in the frequency domain, $\F^{-1}$ denotes the inverse Fourier transform,
and $\cdot$ denotes element-wise multiplication. The algorithm for approximating the
gradient in the frequency domain is summarized in Figure~\ref{alg:frequency}.

The only operations that cannot be directly mapped to the frequency domain are
the calculation of the maximum function, the generation of Gaussian noise, and
trimming of the filter kernels. To perform the first two operations, an image
needs to be mapped to the spatial domain and back. However, these operations
need only be calculated $2N_\text{k}$ times per iteration and are therefore not
a significant contributor to the total running time. Because filter kernels are
padded to the input image size, the size of the learned filter kernels must be
explicitly enforced by trimming. This is done by transferring the filter kernels
to the spatial domain, setting the values outside of the specified filter kernel size
to zero, and then transforming the filter kernels back to the frequency domain.
This procedure needs to be performed only once per mini-batch. Since the number
of mini-batches is relatively small compared to the number of training images,
trimming of the filter kernels also does not add significantly to the total
running time of the training algorithm.

\subsection{GPU Implementation and Memory Considerations}

To further reduce training times, the algorithm can be efficiently implemented
on graphics cards, which can perform hundreds of operations in parallel.
To optimize efficiency, it is crucial to maximize the utilization of the large
number of available GPU cores, while minimizing the required amount of GPU
memory. Because our algorithm requires only a relatively small number of FFT
calculations per iteration, the computational bottleneck is the calculation of
the element-wise operations, which can be performed in parallel. Thus we
distribute the processing of a single 2D image over $N_\text{V}(\lfloor
N_\text{V}/2 \rfloor + 1) \times N_\text{C}$ independent threads, with one
thread per element in the frequency domain. The large number of parallel threads
results in a high utilization of the GPU, even when each image of a mini-batch
is processed sequentially, which we do in our method because this greatly
reduces the amount of GPU memory required to store the visible and hidden units
compared to processing batches of images in parallel.

Due to the relatively small amount of GPU memory compared to CPU memory, memory
requirements are an important consideration when designing a GPU algorithm.
However, the total amount of required memory is highly implementation-dependent
(e.g. the use of temporary variables for storing intermediate results) and a
comparison can only be done with common elements at the algorithmic level.
Therefore, in the remainder of this section, we focus on the memory requirements
of key variables such as the visible units, the hidden units, and the filters,
which are required by all implementations. In our training algorithm, all key
variables are stored in the frequency domain, where each element in the
frequency domain is represented by a single-precision complex number. Due to the
symmetry of the Fourier space, the number of elements that need to be stored is
roughly half the number of elements in the spatial domain. Thus, the memory
required for storing the visible and hidden units in the frequency domain is
roughly the same as in the spatial domain. A potential drawback of training in
the frequency domain is that the filters need to be padded to the size of the
visible units before applying the Fourier transformation, which increases the
memory required for storing the filters on the GPU. The total amount of memory
in bytes required for storing the visible units, hidden units, and padded
filters is given by the sum of their respective terms
\begin{equation} 
%\text{memory}_{\text{freq}} &= 8N_\text{v}(\lfloor N_\text{v}/2 \rfloor +1)
%\times (N_\text{c} + N_\text{k} + N_\text{k}N_\text{c}) \\
%&\approx 4N_\text{v}^2 \times (N_\text{c} + N_\text{k} + N_\text{k}N_\text{c})
% \\
% &\approx 
4N_\text{v}^2N_\text{c} + \frac{4N_\text{v}^2 N_\text{k}}{s^2} +
4N_\text{v}^2N_\text{k}N_\text{c}
\end{equation}
As a comparison, the training method of \cite{Krizhevsky2012} processes batches
of images in order to fully utilize the GPU, which requires the storing of
batches of visible and hidden units. The memory needed for storing the key
variables is given by
\begin{equation} 
%\text{memory}_{\text{spat}}= 4N_\text{v}^2 \times (N_\text{b}N_\text{c} +
% N_\text{b}N_\text{k}) + 4N_\text{w}^2N_\text{k}N_\text{c}
4N_\text{v}^2 N_\text{b}N_\text{c} + \frac{4N_\text{v}^2
N_\text{b}N_\text{k}}{s^2} + 4N_\text{w}^2N_\text{k}N_\text{c}
\end{equation}
where $N_\text{b}$ is the number of images per mini-batch. Depending on the
choice of the batch size, this method requires more memory for storing the
visible and hidden units, while requiring less memory for storing the filters.
Alternatively, \citet{Mathieu2013} proposed to speed up the training of CNNs by
calculating convolutions between batches of images and filters using FFTs. The
memory required for storing the visible units, hidden units, and filters using
this method is given by
\begin{equation}
%\text{memory}_{\text{freq}} &= 8 N_\text{v}(\lfloor N_\text{v}/2 \rfloor +1)
%\times (N_\text{b}N_\text{c} + N_\text{b}N_\text{k} + N_\text{k}N_\text{c}) \\
%&\approx 
4 N_\text{v}^2 N_\text{b}N_\text{c} + 4
N_\text{v}^2 N_\text{b}N_\text{k} + 4 N_\text{v}^2 N_\text{k}N_\text{c}
\end{equation}
Table~\ref{tab:memory} shows a comparison of the memory per GPU required for
storing key variables when training a network used in previous work by
\citet{Krizhevsky2012}. For the first layer, a comparison with Mathieu et al.'s
training method could not be performed, because that method does not support
strided convolutions, which would significantly reduce the memory required for
storing the hidden units. In all layers, the proposed approach compensates for
the increased memory requirements for the filters by considering one image at a
time rather than a batch, and still outperforms batched learning in the spatial
domain in terms of speed (see Section 3), despite not using batches.

% In all layers, the additional memory required for
% storing the padded filters is smaller than the memory required for storing
% batches of visible and hidden units. Thus, our method requires less memory for
% storing the key variables during training than two state-of-the-art
% GPU-accelerated methods.

\begin{table}[tb]
\caption{Comparison of the memory required for storing key
variables using different training methods: our method (Freq), Krizhevsky et al.'s spatial
domain method (Spat), and Mathieu et al.'s method using batched FFTs (B-FFT). A
comparison with Mathieu et al.'s method could not be made for the first layer,
because that method does not support strided convolutions. In all layers, our
method consumes less memory for storing the key variables than the other two
methods.}

\label{tab:memory}
\vspace{1em}
\centering

\sisetup{
  round-mode = places,
  round-precision = 1,
  exponent-product = \cdot,
  detect-weight=true,
  detect-inline-weight=math,
  tight-spacing = false,
  table-align-text-post = false
}%
\begin{tabular}{l
S[table-format=3.0]
S[table-format=3.0]
S[table-format=3.0]
S[table-format=2.0]
S[table-format=3.0]
S[table-format=1.0]
S[table-format=2.1]
S[table-format=3.1]
S[table-format=3.1]}
\toprule
Layer & {$N_\text{b}$} & {$N_\text{v}$} & {$N_\text{c}$} & {$N_\text{w}$} &
{$N_\text{k}$} & {$s$} & \multicolumn{3}{c}{Memory in MB} \\
& & & & & & & {Freq} & {Spat} & {B-FFT}
%{Out Method} & {\citet{Krizhevsky2012}} & {\cite{Mathieu2013}}
\\
\midrule
%1 (without striding) & 128 & 224 & 3 & 11 & 48 & 4 & 37.32422 & 1249.566 &
%1277.062 \\
1 & 128 & 224 & 3 & 11 & 48 & 4 & 28.71094 & 147.0665 & {---} \\
2 & 128 & 55 & 48 & 5 & 128 & 1 & 72.92938 & 260.5469 & 330.8594 \\
3 & 128 & 27 & 128 & 3 & 192 & 1 & 69.23364 & 114.75 & 182.25 \\
4 & 128 & 13 & 192 & 3 & 192 & 1 & 24.01318 & 32.95312 & 55.45312 \\
5 & 128 & 13 & 192 & 3 & 128 & 1 & 16.05005 & 27.25 & 42.25 \\
%\addlinespace
%Total & & & & & & & 210.9372 & 582.5665 & 330.8594 \\
\bottomrule
\end{tabular}\\[0.2em]
\end{table}

\subsection{Mapping Strided Convolutional to Stride-1 Convolutions}

Strided convolutions are a type of convolution that shifts the filter kernel as
a sliding window with a step size or stride $s > 1$, stopping at only $N_\text{v}
/ s$ positions. This reduces the number of hidden units per channel to
$N_\text{h} = N_\text{v} / s$, hence significantly reducing training times and
memory required for storing the hidden units during training. The energy of a
strided convolutional RBM (sconvRBM) is defined as
\begin{equation} 
E(\vect{v},\vect{h}) = 
-\sum_{i=0}^{N_\text{c}-1}\sum_{j=0}^{N_\text{k}-1}\sum_{x,y=0}^{N_\text{h}-1}
\sum_{u,v=-\lfloor N_\text{w}/2\rfloor}^{\lfloor(N_\text{w}-1)/2\rfloor}
\hspace{-1.2em}h_{xy}^jw_{uv}^{ij}v_{sx+u, sy + v}^i -
\sum_{i=0}^{N_\text{c}-1}b_i\!\sum_{x,y = 0}^{N_\text{v}-1}\!v_{xy}^i -
\sum_{j=0}^{N_\text{k}-1}c_j\!\sum_{x,y = 0}^{N_\text{h}-1}\!h_{xy}^j
\end{equation}  

\begin{figure}[tb]
\centering
\begin{tikzpicture}[node distance=0.9cm and 4.5cm]

\tikzstyle{every node}+=[font=\scriptsize\sffamily]
\tikzstyle{label}=[font=\footnotesize\sffamily]
\tikzstyle{lines}=[dashed,opacity=0]
\tikzstyle{mymatrix}=[%
  matrix of nodes,%
  inner sep=0pt,
  nodes={inner sep=2pt},
  %draw,%
  left delimiter  = (,%
  right delimiter = )%
]

\tikzstyle{every pin}+=[inner sep=2pt,align=center,pin
distance=0.1cm,font=\footnotesize\sffamily]

% \matrix [mymatrix] (image) {
% |[red]|4  & |[green]|5  & |[red]|1  & |[green]|2  & |[red]|1  & |[green]|2  \\
% |[blue]|4 & |[yellow]|2 & |[blue]|5 & |[yellow]|1 & |[blue]|1 & |[yellow]|2 \\
% |[red]|4  & |[green]|5  & |[red]|1  & |[green]|2  & |[red]|1  & |[green]|2  \\
% |[blue]|4 & |[yellow]|2 & |[blue]|5 & |[yellow]|1 & |[blue]|1 & |[yellow]|2 \\
% |[red]|4  & |[green]|5  & |[red]|1  & |[green]|2  & |[red]|1  & |[green]|2  \\
% |[blue]|4 & |[yellow]|2 & |[blue]|5 & |[yellow]|1 & |[blue]|1 & |[yellow]|2 \\
% };

\matrix [mymatrix] (image) {
4 & 2 & 1 & 8 & 2 & 7 \\
4 & 2 & 3 & 4 & 3 & 1 \\
3 & 4 & 5 & 5 & 1 & 2 \\
7 & 5 & 1 & 2 & 2 & 3 \\
2 & 1 & 2 & 2 & 7 & 5 \\
6 & 9 & 5 & 4 & 6 & 6 \\
};

\matrix [mymatrix, below=of image] (kernel) {
0.1 & -0.1 & 0.3 & 0.4 \\
0.2 & -0.3 & 0.1 & 0.1 \\
-0.4 & 0.2 & 0.2 & -0.3 \\
0.5 & -0.2 & -0.2 & 0.4 \\
};

\matrix [mymatrix, below=of kernel,thick] (map) {
%\matrix [mymatrix, above=of image,thick] (map) {
|[red]|\bfseries 6.8 & |[blue]|\bfseries 2  \\
\bfseries 4.6 &  \bfseries 3.6 \\
};

\node[fit=(image-1-1)(image-4-4), xshift=-0.5pt, yshift=-0.5pt, draw=red,inner
sep=0,thick] (window1) {};

\node[fit=(image-1-3)(image-4-6), xshift=0.5pt, yshift=0.5pt, draw=blue,inner
sep=0,thick,pin=45:{Sliding filter\\ window}] (window2) {};

\draw[decorate,decoration={brace,raise=1pt}] (window1.north
west|-window2.north west)--node[above=3pt,fill=white,fill opacity=0.5,text
opacity=1,font=\small]{$s$}(window2.north west);

\begin{pgfonlayer}{background}
\foreach \x in {1,3,5} {
  \foreach \y in {1,3,5}
    \node[fit=(image-\x-\y),inner sep=0, fill=red!15] {};
  \foreach \y in {2,4,6}
    \node[fit=(image-\x-\y),inner sep=0, fill=green!15] {};
}
\foreach \x in {2,4,6} {
  \foreach \y in {1,3,5}
    \node[fit=(image-\x-\y),inner sep=0, fill=blue!15] {};
  \foreach \y in {2,4,6}
    \node[fit=(image-\x-\y),inner sep=0, fill=yellow!30] {};
}
\foreach \x in {1,3} {
  \foreach \y in {1,3}
    \node[fit=(kernel-\x-\y),inner sep=0, fill=red!15] {};
  \foreach \y in {2,4}
    \node[fit=(kernel-\x-\y),inner sep=0, fill=green!15] {};
}
\foreach \x in {2,4} {
  \foreach \y in {1,3}
    \node[fit=(kernel-\x-\y),inner sep=0, fill=blue!15] {};
  \foreach \y in {2,4}
    \node[fit=(kernel-\x-\y),inner sep=0, fill=yellow!30] {};
}
\end{pgfonlayer}

\node[align=center] at ($(image.south)!0.5!(kernel.north)$) {\footnotesize $*$\\
(with a step size of $s$)};

%\node at ($(image.north)!0.5!(map.south)$) {$=$};
\node[label] at ($(kernel)!0.65!(map)$) {$=$};

\begin{scope}[on grid,node distance=2cm and 1.95cm] 
\matrix [mymatrix, right=3.8cm of image,fill=red!15] (image1) {
4 & 1 & 2 \\
3 & 5 & 1 \\
2 & 2 & 7 \\
};

\matrix [mymatrix, right=of image1,fill=green!15] (image2) {
2 & 8 & 7 \\
4 & 5 & 2 \\
1 & 2 & 5 \\
};

\matrix [mymatrix, right=of image2,fill=blue!15] (image3) {
4 & 3 & 3 \\
7 & 1 & 2 \\
6 & 5 & 6 \\
};

\matrix [mymatrix, right=of image3,fill=yellow!30] (image4) {
2 & 4 & 1 \\
5 & 2 & 3 \\
9 & 4 & 6 \\
};

\matrix [mymatrix, right=3.8cm of kernel,fill=red!15] (kernel1) {
0.1 & 0.3 \\
-0.4 & 0.2 \\
};

\matrix [mymatrix, right=of kernel1,fill=green!15] (kernel2) {
-0.1 & 0.4 \\
0.2 & -0.3 \\
};

\matrix [mymatrix, right=of kernel2,fill=blue!15] (kernel3) {
0.2 & 0.1 \\
0.5 & -0.2 \\
};

\matrix [mymatrix, right=of kernel3,fill=yellow!30] (kernel4) {
-0.3 & 0.1 \\
-0.2 & 0.4 \\
};

\matrix [mymatrix, right=3.8cm of map] (map1) {
\textcolor{red}{0.5} & \textcolor{blue}{-1.1}  \\
1.4 & 1.4 \\
};
\matrix [mymatrix, right=of map1] (map2) {
\textcolor{red}{2.3} & \textcolor{blue}{2.4}  \\
1.2 & -0.8   \\
};
\matrix [mymatrix, right=of map2] (map3) {
\textcolor{red}{4.4} & \textcolor{blue}{1}  \\
3.5 & 1.7  \\
};
\matrix [mymatrix, right=of map3] (map4) {
\textcolor{red}{-0.4} & \textcolor{blue}{-0.3}  \\
-1.5 & 1.3 \\
};
\matrix [mymatrix, right=of map4, thick] (map5) {
|[red]|\textbf{6.8} & |[blue]|\textbf{2}  \\
\textbf{4.6} & \textbf{3.6}   \\
};

\foreach \x in {1,...,4} {
  \node[fit=(image\x-1-1)(image\x-2-2), draw=red,inner sep=0,thick,
  xshift=-0.5pt, yshift=-0.5pt] {};
  
  \node[fit=(image\x-1-2)(image\x-2-3), draw=blue,inner sep=0,thick,
  xshift=0.5pt, yshift=0.5pt] {};
  \node[label] at ($(image\x.south)!0.5!(kernel\x.north)$) {$*$};
  %\node at ($(image\x.north)!0.5!(map\x.south)$) {$=$};
  \node[label] at ($(kernel\x)!0.65!(map\x)$) {$=$};
}

\node at ($(map1.east)!0.5!(map2.west)$) {$+$};
\node at ($(map2.east)!0.5!(map3.west)$) {$+$};
\node at ($(map3.east)!0.5!(map4.west)$) {$+$};
\node at ($(map4.east)!0.5!(map5.west)$) {$=$};

\node[left=1.85cm of image,rotate=90,label] {Image
$\vect{v}^0$};
\node[left=1.85cm of kernel,rotate=90,align=center,label]
{Filter $\vect{w}^{0,j}$};
\node[left=1.85cm of map,rotate=90,align=center,label] (maplabel)
{Feature};
\node[right=1em of maplabel,rotate=90,label] {map $\vect{h}^j$};

\draw[->,shorten <=10pt, shorten >=10pt] (image)--
node[above,label]{Rearrange}node[below,label]{pixels}(image1);

\draw[->,shorten <=10pt, shorten >=10pt] (kernel)--
node[above,label]{Rearrange}node[below,label]{pixels}(kernel1);

\foreach\x in {1,...,4} {
\draw[decorate,decoration={brace,raise=2pt}]
(image\x.south east)--node[below=4pt,label]
{$\vect{V}^{\pgfmathparse{int(\x-1)}\pgfmathresult{}}$} (image\x.south west);

\draw[decorate,decoration={brace,raise=2pt}]
(kernel\x.south east)--node[below=4pt,label]
{$\vect{W}^{\pgfmathparse{int(\x-1)}\pgfmathresult,j}$} (kernel\x.south west);
}

\draw[decorate,decoration={brace,raise=2pt}]
(map5.south east)--node[below=4pt,label]
{$\vect{h}^{j}$} (map5.south west);

\end{scope}

\end{tikzpicture}

\caption{Illustration of convolutions with a sliding window step
size $s = 2$ as used during filtering in a sconvDBN. A convolution of a given stride size (left
side) can be efficiently calculated as the sum of multiple individual
convolutions with $s = 1$ (right side) after rearranging the pixels of the input
image and the filter kernels. The bottom row shows the actual values produced by
the convolutions, which are the features from the image extracted by the filter.
The figure is best viewed in color.}
\label{fig:npcDBN}
\end{figure}

Convolutions with a stride $s > 1$ can be expressed equivalently as convolutions
with stride $s = 1$ by reorganizing the values of $\vect{v}^i$ and
$\vect{w}^{ij}$ to $\vect{V}^{i'}$ and $\vect{W}^{i'j}$ as illustrated in
Figure~\ref{fig:npcDBN}. This reindexing scheme allows the energy function to be
expressed in terms of conventional (stride-1) convolutions, which facilitates
training in the frequency domain. The new indices of $V^{i'}_{x'y'}$ and
$W^{i'j}_{u'v'}$ can be calculated from the old indices of $v^i_{xy}$ and
$w^{ij}_{uv}$ as follows:
\begin{align}
x' &= \lfloor x / s \rfloor & u' &= \lfloor u / s \rfloor \\
y' &= \lfloor y / s \rfloor & v' &= \lfloor v / s \rfloor \\
i' &= s^2i + s(y \bmod{s}) + (x \bmod{s})
\end{align}
After reorganizing $\vect{v}^i$ and $\vect{w}^{ij}$ to $\vect{V}^{i'}$
and $\vect{W}^{i'j}$, the energy of the model can be rewritten as
\begin{align} 
E(\vect{V},\vect{h}) &= 
-\sum_{i=0}^{N_\text{C}-1}\sum_{j=0}^{N_\text{k}-1}\sum_{x,y=0}^{N_\text{h}-1}
\sum_{u,v=-\lfloor N_\text{W}/2\rfloor}^{\lfloor (N_\text{W}-1)/2\rfloor}
\hspace{-1.2em}h_{xy}^jW_{uv}^{ij}V_{x+u, y+v}^i
- \sum_{i=0}^{N_\text{C}-1}b_i\!\sum_{x,y = 0}^{N_\text{V}-1}\!V_{xy}^i
- \sum_{j=0}^{N_\text{k}-1}c_j\!\sum_{x,y = 0}^{N_\text{h}-1}\!h_{xy}^j\\
&= -\sum_{i=0}^{N_\text{C}-1} \sum_{j=0}^{N_\text{k}-1} \vect{h}^j
\bullet (\tilde{\vect{W}}^{ij} * \vect{V}^i) -
\sum_{i=0}^{N_\text{C}-1}b_i\!\sum_{x,y = 0}^{N_\text{V}-1}\!V_{xy}^i -
\sum_{j=0}^{N_\text{k}-1}c_j\!\sum_{x,y = 0}^{N_\text{h}-1}\!h_{xy}^j
\end{align}
where $*$ denotes periodic convolution. The number of channels, number of
visible units per channel and number of weights per channel after reorganization
are given by $N_\text{C} = N_\text{c} * s^2$, $N_\text{V}^2 = N_\text{v}^2 /
s^2$ and $N_\text{W}^2 = N_\text{w}^2 / s^2$, respectively.

\section{Evaluation of Runtime}

% TODO: comparison with cuDNN in 3D

To demonstrate where the performance gains are produced, we trained a two-layer
sconvDBN on 2D and 3D images using our frequency-domain method and the following
methods that all compute convolutions on the GPU, but using different
approaches: 1) our spatial domain implementation that convolves a single 2D or
3D image with a single 2D or 3D filter kernel at a time, 2)~Krizhevsky's spatial
domain convolution implementation \citep{Krizhevsky2012b}, which is a widely
used method \citep[e.g.,][]{scherer2010,hinton2012,zeiler2013} that calculates
the convolution of batches of 2D images and 2D filter kernels in parallel (note
that this method cannot be applied to 3D images, so it was only used for the 2D
experiments), and 3) our implementation that calculates convolutions using FFTs,
but without mapping the other operations that would allow the algorithm to stay
in the frequency domain when not computing convolutions. The parameters that we
used for training convRBMs on 2D and 3D images are summarized in
Table~\ref{tab:parameters}. The key parameters that we varied for our
experiments are the filter size and stride size of the first layer, and the
filter size and the number of channels of the second layer.
Because the number of channels of the second layer is equal to the number of
filters of the first layer, we also varied the number of filters of the first
layer in order to attain the desired number of channels. For all
implementations, the training time is directly proportional to the number of
filters. Therefore, a detailed comparison of all four methods with a varying
number of filters was not included in this paper. The hardware details of our
test environment are summarized in Table~\ref{tab:hardware}.

\begin{table}
\centering
\caption{Training parameters.}\vspace{1em}
\begin{tabular}{lcccc}
\toprule
Parameter & \multicolumn{2}{c}{ImageNet (2D)} & \multicolumn{2}{c}{OASIS (3D)}
\\
\addlinespace
 & \multicolumn{1}{c}{1st layer} & \multicolumn{1}{c}{2nd layer} 
 & \multicolumn{1}{c}{1st layer} & \multicolumn{1}{c}{2nd layer} \\
\cmidrule(r){1-1} \cmidrule(lr){2-2} \cmidrule(lr){3-3} \cmidrule(lr){4-4}
\cmidrule(l){5-5}
Filter size & 5 to 52 & 5 to 13 & 3 to 36 & 3 to 9 \\
Stride size & 1, 2, 4 & 1 & 1, 2, 4 & 1 \\
Number of channels & 3 & 16, 32, 64 & 1 & 8, 16, 32 \\
Number of filters & 32 & 32 & 32 & 16 \\
Image dimension & $512^2$ & $128^2$ & $128^3$ & $64^3$ \\
Number of images & 256 & 256 & 100 & 100 \\
Batch size & 128 & 128 & 25 & 25 \\
\bottomrule
\end{tabular}
\label{tab:parameters}
\end{table}

\begin{table} 
\centering
\caption{Hardware specification of our test system.}
\vspace{1em}
\label{tab:hardware}
\begin{tabular} {ll}
\toprule
Processor & Intel i7-3770 CPU @ 3.40\,GHz \\
CPU Memory & 8\,GB \\
\addlinespace
Graphics Card & NVIDIA GeForce GTX 660 \\
GPU Cores & 960 cores @ 1.03\,GHz \\
GPU Memory & 2\,GB \\
\bottomrule
\end{tabular}
\end{table}

For the comparison on 2D images, we used a dataset of 256 natural color images
from the ImageNet dataset \citep{deng2009}. All images were resampled to a resolution
of \num{512x512} pixels per color channel. For the evaluation on 3D images, we
used 100 magnetic resonance images (MRIs) of the brain from the OASIS dataset
\citep{Marcus2007}. We resampled all volumes to a resolution of
\num{128x128x128} voxels and a voxel size of \SI{2x2x2}{\milli\metre}.

\subsection{Running Time Analysis on 2D Color Images (ImageNet)}

Figure~\ref{fig:run_imgnet_l1} shows a comparison of running times for training
the first sconvRBM layer on 256 images with varying filter and stride sizes.
Due to internal limitations of Krizhevsky's convolution implementation, it
cannot be applied to images with a resolution of \num{512x512} pixels when using
a stride size smaller than four, and those comparisons could not be made. Our
frequency domain implementation is between 2 to 24 times faster than our
convolution implementation, where the speed gains are larger for larger filter
and stride sizes. For a stride of 1, the impact of the convolution
implementation on the total running time is relatively low, because the
computational bottleneck is the inference and sampling of the hidden
units.
% For a stride of one, the proportion of the time spend to calculate
% convolutions is small compared to calculating the expectation of the hidden
% units and to sample the hidden units, which reduces the impact of the choice
% of the convolution implementation on the total running time.
% For a stride of one, the running time mostly depends on the time spent to
% calculate the expectation of the hidden units and to sample the hidden units.
As the number of hidden units decreases with larger strides, the running time
becomes more dependent on the time spent to calculate convolutions.
Hence, the differences between the four methods are more pronounced for larger
strides. For a stride of four, training in the frequency domain is between 8 to
24 times faster than training in the spatial domain using our convolution
implementation and 2 to 7 times faster than using batched convolutions.
Calculating convolutions by FFTs is the slowest method for all stride sizes and
2D filter sizes up to 44, largely due to the cost of calculating Fourier
transforms.

\begin{figure}[t!]
\subfloat[Running times of training a first layer sconvRBM
with stride sizes of 1, 2, and 4.] {
\label{fig:run_imgnet_l1}
\includegraphics{figures/NECO-03-14-2099-Figure-2}
%\input{r_figures/runtime_imgnet_c3_bw}
}

\subfloat[Running times of training a second layer sconvRBM with
16, 32, and 64 channels (stride size 1).] {
\label{fig:run_imgnet_l2}
\includegraphics{figures/NECO-03-14-2099-Figure-3}
%\input{r_figures/runtime_imgnet_c16-64_bw}
}

\label{fig:run_imgnet}
\caption{Comparison of running times for training a (a) first and
(b) second layer sconvRBM on 2D images using our frequency domain method (freq) and three
alternative methods using different convolution implementations: single image
convolutions (spat1), batched convolutions (spat2), and convolution by using
FFTs (fft). Due to internal limitations of the implementation of batched
convolutions, a comparison with spat2 could not be performed for images with a
resolution of \num{512x512} when using a stride size smaller than four.}
\end{figure}

Figure~\ref{fig:run_imgnet_l2} shows a similar comparison for training the
second convRBM layer for a stride size of 1 and varying filter sizes and numbers
of channels. In contrast to training the first layer, training times mostly
depend on the calculation of convolutions, where the impact of calculating
convolutions on the total running time increases with an increasing number of
channels. Training in the frequency domain is between 5 to 26 times faster than
training in the spatial domain using single-image convolutions, and 2
to 8 times faster than using batched convolutions. For all channel sizes,
batched training is about 3 to 4 times faster than non-batched training and
calculating convolutions using FFTs is much slower than batched training and
training in the frequency domain. To summarize, training of 2D images in the
frequency domain is much faster than training in the spatial domain even for
small filter sizes.
Using the largest filter kernels in both layers, the proposed method is shown to
yield a speedup of 7 to 8 times compared to state-of-the-art GPU
implementations.

\subsection{Running Time Analysis on 3D Volumes (OASIS)}

Figure~\ref{fig:run_oasis} shows the comparison of running times for training a
first  and second layer sconvRBM on 3D volumes for varying filter sizes, stride
sizes, and varying numbers of channels. In contrast to training on 2D images,
the computational costs of calculating 3D convolutions break even with
calculating FFTs for small filter sizes, because the number of multiplications
and additions per convolution increases cubically, instead of quadratically,
with the filter kernel size. As a result, simply training by convolutions in the
frequency domain is faster than in the spatial domain. However, our proposed
training algorithm still outperforms both other methods, even at the smallest
filter size. For filter sizes of 5 and larger, our frequency domain
implementation is between 3.5 to 200 times faster than our spatial domain
implementation using single-image convolutions and 2.7 to 17 times faster than
calculating convolutions by FFTs. Similar to the results on 2D images, training
times of the first layer using a stride of 1 depend strongly on the time
required to calculate the expectation of the hidden units and to sample the
hidden units. Hence, performance improvements of our frequency domain method are
more pronounced for larger strides and numbers of channels, where the impact of
calculating convolutions on the total training time is also larger. This makes
the proposed method particularly suitable for training sconvRBMs on
high-resolution 3D volumes.

\begin{figure}[t!]
\subfloat[Running times of training a first layer sconvRBM with stride sizes of 1, 2, and 4.] {
\label{fig:run_oasis_l1}
\includegraphics{figures/NECO-03-14-2099-Figure-4}
%\input{r_figures/runtime_oasis_c1_bw}
}

\subfloat[Running times of training a second layer sconvRBM with 8, 16, and 32
channels (stride size 1).] {
\label{fig:run_oasis_l2}
\includegraphics{figures/NECO-03-14-2099-Figure-5}
%\input{r_figures/runtime_oasis_c8-32_bw}
}
\caption{Comparison of running times for training a (a) first and (b) second
layer sconvRBM on 3D volumes using a single 3D image convolution implementation
(spat), an implementation that calculates convolutions by using FFTs (fft), and
our proposed implementation in the frequency domain (freq).}
\label{fig:run_oasis}
\end{figure}

