\chapter{Introduction}

Introduction of deep learning and motivation. What is deep learning? Why is
everyone so hyped about it? What is the vision/promise? Thinking in feature
hierarchies. Neuroplasticity suggestions that this can be learned. Inspiration
for different models like neural networks. Old field but met with problems.
Renaissance with faster hardware, more data, layerwise pre-training. Latest
development is fast. Fill this with examples of different domains and how deep
learning helped to revolutionized object recognition, object detection, semantic
segmentation, speech recognition, natural language processing.

Goal of the thesis. What am I up to? If it was so successful in so many
different domain, can deep learning help to solve problems in the medical domain
and where can it have a big impact? In this thesis, we focus on two possible
applications of deep learning, unsupervised learning of features that are
descriptive of images as a whole. This is directly related to manifold learning
and learning the variability in brain MRIs and the relationship to diseases. The
other part is image segmentation with the focus on MS lesion segmentation.

% See literature review as a tool, not as a goal. Try to structure the thesis
% without it and put it where it belongs. Maybe restructure later. I needed a part
% of it already for the motivation. I will put a lot in it when I describe the
% methods. An I will put something in there, when I describe the applications.

Give an outlook. First introduce some deep learning methods. Then talk about
special challenges when it comes to 3D medical images. A big part is training.
Training in the frequency domain explained plus a discussion of related work. A
general introduction is followed by the applications of deep learning. This will
be a chapter on its own. Start with learning variability in images or global
image descriptors and then segmentation.

\section{Multiple Sclerosis}

Give some introduction to MS and why it is important and what the challenges are
and how that links to what I'm doing in this thesis.
