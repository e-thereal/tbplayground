\chapter{Background}

Background chapter on deep learning methods. Start with general introduction. I
want to talk about some deep learning methods. Main goal is learning a feature
hierarchy. Models are often biologically inspired. Neuroplasticity strong
motivator for learning. Start with either unsupervised or supervised models.
Explain what it means and why it is important. Unsupervised no need for labels,
which can be difficult to obtain. Supervised tuned to task. Can perform better
when enough labels are available. Requires careful regularization or lots of
data because more prone to overfitting.

% Reduce overlap with introduction)

% RBMs, DBNs, convRBMs, convDBNs for manifold learning and NN and CNN for lesion
% segmentation.

\section{Terminology}

\subsection{Unit Layers vs. Processing Layers}

The term layer can be confusing because different layers can be meant. One way
is to count the number of layers of units or just unit layers for short. The
transition from the input units to the output units involves one calculation.
Hence, an alternative way of counting layers is to count the number of
operations or processing layers. One processing layer takes the inputs of the
previous unit layers and transforms them to calculate the values of the output
unit layer. In the remainder, the term layer will refer to unit layer, and
processing layers will be named specifically.

\section{Learning from Labeled Data}

% If I put this first, I can argue with inspiration from biological models
% first.

\subsection{Neural Networks}

A dense neural network is a deterministic function that maps input data to the
desired outputs through the successive application of multiple nonlinear
mappings of the following form
\begin{align}
\vect{z}_l &= \vect{W}_l\vect{x}_{l-1} + \vect{b}_l, \\
\vect{x}_l &= f_l(\vect{z}_l),
\end{align}
where $\vect{x}_0$ denotes the input of the neural network, $\vect{x}_L$ denotes
the output, $L$ is the number of computational layers, $f_l$ are transfer
functions, $\vect{W}_l$ are weight matrices, and $\vect{b}_l$ are bias terms.
Popular choices for the transfer function are the sigmoid function $f(x) =
\sigm(x)$ and the rectified linear function $f(x) = \max(0, x)$. The same
transfer function is typically used for all layers except for the output layer.
The choice of the output transfer function depends on the learning task. For
classification, a 1-of-$n$ encoding of the output class is usually used in
combination with the softmax transfer function defined as
\begin{equation}
\text{softmax}(\vect{z})_i = \frac{\exp(z_i)}{\sum_{j=1}^n \exp(z_j)},
\end{equation}
where $\vect{z}$ denotes an $n$-dimensional output vector.

Given a training set $\data = \{(\vect{x}_0^{(i)}, \vect{y}^{(i)})\given i
\in [1, N] \}$, a neural network is trained by minimizing the error
between the predicted outputs $\vect{x}_L^{(i)}$ and the given labels
$\vect{y}^{(i)}$
\begin{equation}
\hat{\thetas} = \arg\min_{\thetas} \sum_{i = 1}^N E\Big(\vect{x}_L^{(i)},
\vect{y}^{(i)}\Big),
\end{equation}
where $\thetas$ denotes the trainable parameters of the neural network. Typical
choices for the error function are the sum of squared differences (SSD), and
cross-entropy.
The minimization problem can be solved using stochastic gradient descent (SGD)
\citep{Rumelhart1986,polyak1992}, which requires the calculation of the gradient
of the error function with respect to the model parameters. The gradient can be
calculated by backpropagation \citep{werbos1974} as follows
\begin{align}
\deltas_L &= \nabla_{\vect{x}_L}E \cdot f_L'(\vect{z}_L), \\
\deltas_l &= (\vect{W}_{l+1}^\text{T}\deltas_{l+1}) \cdot
f_l'(\vect{z}_l) & \text{for $l < L$,}\\
\nabla_{\vect{W}_l}E &= \deltas_l\vect{x}_{l-1}^\text{T}, \\
\nabla_{\vect{b}_l}E &= \deltas_l,
\end{align}
where $\nabla_{\vect{x}_L}E$ denotes the gradient of the error function with
respect to the predicted output and $\cdot$ denotes element-wise multiplication.

\subsection{Convolutional Neural Networks}

The structure of CNNs is inspired by the complex arrangement of simple and
complex cells found in the visual cortex \citep{hubel1962,hubel1968}. Simple
cells are only connected to a small sub-region of the previous layer and need to
be tiled to cover the entire visual field. In a CNN (see \ref{fig:cnn}),
simple cells are represented by convolutional layers, which exhibit a similar mechanism of local
connectivity and weight sharing. Complex cells combine the activation of simple
cells to add robustness to small translations. These cells are represented in
the form of pooling layers similar to the pooling layers found in convDBNs.
After several alternating convolutional and pooling layers, the activations of
the last convolutional layer are fed into one or more dense layers to
carry out the final classification.

\begin{figure}
\centering
\begin{tikzpicture}[%
scale=0.85,
x  = {(1cm,0cm)},
y  = {(0.33cm,0.23cm)},
z  = {(0cm,1cm)}
%z  = {(0.9cm,-0.1cm)},
%x  = {(0.33cm,0.23cm)},
%y  = {(0cm,1cm)}
]

\tikzstyle{every node}=[font=\small, inner sep=3pt, align=center]
\tikzstyle{every pin}=[align=center,fill=white]
\tikzstyle{dbnlabel}=[font=\sffamily\normalsize]

\tikzstyle{image}=[fill=white, fill opacity=0.75]
\tikzstyle{pinline}=[thin, black!35]
\tikzstyle{kernelline}=[very thin]


%%%%%%%%%%%%%%%%%         
% INPUT LAYER
%%%%%%%%%%%%%%%%%

\foreach \x in {1, ..., 3} {
\def\y{0.25*\x}
\draw[image] (\y, -2,-2) coordinate (A1\x) -- (\y, 2,-2)
coordinate (B1\x) -- (\y, 2,2) coordinate (C1\x) -- (\y, -2,2) -- cycle;

\draw[kernelline]
      (\y, -1.6, 1.6) \ifnum\x=3 \else -- +(0.25, 0, 0) +(0,0,0) \fi
   -- (\y, -0.8, 1.6) \ifnum\x=3 \else -- +(0.25, 0, 0) +(0,0,0) \fi
   -- (\y, -0.8, 0.8) \ifnum\x=3 \else -- +(0.25, 0, 0) +(0,0,0) \fi
   -- (\y, -1.6, 0.8) \ifnum\x=3 \else -- +(0.25, 0, 0) +(0,0,0) \fi
   -- (\y, -1.6, 1.6);
   ;
\ifnum\x=3
\draw[kernelline] (\y, -1.6, 1.6) coordinate(a1) -- (1.65, -1.2, 1.2);
\draw[kernelline] (\y, -1.6, 0.8) coordinate(b1) -- (1.65, -1.2, 1.2);
\draw[kernelline] (\y, -0.8, 0.8) coordinate(c1) -- (1.65, -1.2, 1.2);
\draw[kernelline] (\y, -0.8, 1.6) coordinate(d1) -- (1.65, -1.2, 1.2);
\fi
}

%%%%%%%%%%%%%%%%%         
% HIDDEN LAYER
%%%%%%%%%%%%%%%%%

\foreach \x in {0, ..., 7} {
\def\y{0.125*\x+1.65}
\draw[image] (\y,-1.6,-1.6) coordinate(A2\x) -- (\y,1.6,-1.6)
-- (\y,1.6,1.6) coordinate (C2\x) -- (\y,-1.6,1.6) coordinate(D2\x) -- cycle;

\draw[kernelline]
      (\y, -0.8, 0.8) \ifnum\x=7 \else -- +(0.125, 0, 0) +(0,0) \fi
   -- (\y, -0.4, 0.8) \ifnum\x=7 \else -- +(0.125, 0, 0) +(0,0) \fi
   -- (\y, -0.4, 0.4) \ifnum\x=7 \else -- +(0.125, 0, 0) +(0,0) \fi
   -- (\y, -0.8, 0.4) \ifnum\x=7 \else -- +(0.125, 0, 0) +(0,0) \fi
   -- (\y, -0.8, 0.8);
\ifnum\x=7
\draw[kernelline] (\y, -0.8, 0.8) coordinate(a2) -- (3.425, -0.6, 0.6);
\draw[kernelline] (\y, -0.8, 0.4) coordinate(b2) -- (3.425, -0.6, 0.6);
\draw[kernelline] (\y, -0.4, 0.4) coordinate(c2) -- (3.425, -0.6, 0.6);
\draw[kernelline] (\y, -0.4, 0.8) coordinate(d2) -- (3.425, -0.6, 0.6);
\fi 
}

\draw[decorate,decoration={brace,raise=20pt}] (A11|-C11) --node[above=25pt]
{Convolutional layer} (C27|-C11);

\foreach \x in {0, ..., 7} {
\def\y{0.125*\x+3.425}
\draw[image] (\y,-0.8,-0.8) coordinate(A3\x) -- (\y,0.8,-0.8)
-- (\y,0.8,0.8) coordinate (C3\x) -- (\y,-0.8,0.8) -- cycle;

\draw[kernelline]
      (\y, -0.6, +0.6) \ifnum\x=7 \else -- +(0.125, 0, 0) +(0,0) \fi
   -- (\y, +0.2, +0.6) \ifnum\x=7 \else -- +(0.125, 0, 0) +(0,0) \fi
   -- (\y, +0.2, -0.2) \ifnum\x=7 \else -- +(0.125, 0, 0) +(0,0) \fi
   -- (\y, -0.6, -0.2) \ifnum\x=7 \else -- +(0.125, 0, 0) +(0,0) \fi
   -- (\y, -0.6, +0.6);
\ifnum\x=7
\draw[kernelline] (\y, -0.6, +0.6) -- (5.2, -0.2, 0.2);
\draw[kernelline] (\y, -0.6, -0.2) -- (5.2, -0.2, 0.2);
\draw[kernelline] (\y, +0.2, -0.2) coordinate (g7) -- (5.2, -0.2, 0.2);
\draw[kernelline] (\y, +0.2, +0.6) -- (5.2, -0.2, 0.2);
\fi 
}

\draw[decorate,decoration={brace,raise=10pt}] (C21|-C21) --node[above=15pt]
{Pooling layer} (C37|-C21);

\foreach \x in {0, ..., 15} {
\def\y{0.125*\x+5.2}
\draw[image] (\y,-0.4,-0.4) coordinate (A4\x) -- (\y,0.4,-0.4)
-- (\y,0.4,0.4) coordinate (C4\x) -- (\y,-0.4,0.4) -- cycle;
}

\begin{scope}[xshift=9cm]
\foreach \x in {1,...,7} {
  \node[circle, draw] (v\x) at (0,0, 0.5*\x - 0.5*4) {};
}
\end{scope}

\draw[shorten >=10pt,shorten <=7pt,dashed] (A415-|C415)--(v1.south west);
\draw[shorten >=10pt,shorten <=7pt,dashed] (C415)--
node[label=120:Vectorize\\ hidden units] {} (v7.north west);

\begin{scope}[xshift=10.5cm]
%\foreach \x/\y in {1/-1.5, 2/-1, 3/-0.5, 4/0, 5/0.5, 6/1, 7/1.5} {
\foreach \x in {1,...,5} {
  \node[circle, draw] (h\x) at (0,0, 0.5*\x - 0.5*3) {};
}
\end{scope}

\foreach \x in {1,...,7} {
	\foreach \y in {1,...,5} {
		\draw (v\x)--(h\y);
	}
}

\draw[decorate,decoration={brace,raise=10pt}] (v7.west|-C21)
 --node[above=15pt] {Dense layer} (h1.east|-C21);
 
\node[above=15pt, fill=white,xshift=-25pt,inner sep=3pt, fill opacity=1, text
opacity=1] 
(convl) at (d1) {convolution};
\draw[pinline, shorten >=4pt, shorten <=2pt] (convl) -- (d1);

\node[above=15pt,fill=white,xshift=35pt,inner sep=3pt]
(pooll) at (d2) {pooling};
\draw[pinline, shorten >=4pt, shorten <=2pt] (pooll) -- (d2);

% Label units

\node[below=10pt,align=center] (input) at (A12) {Multimodal\\ input images};

\draw[pinline, shorten >=4pt] (input) -- (A11);
\draw[pinline, shorten >=4pt] (input) -- (A12);
\draw[pinline, shorten >=4pt] (input) -- (A13);

\node[below=40pt,xshift=20pt,inner sep=5pt] (hiddens) at (A31) {Hidden units};
\draw[pinline, shorten >=4pt] (hiddens) -- (A27);
\draw[pinline, shorten >=4pt] (hiddens) -- (A34);
\draw[pinline, shorten >=4pt] (hiddens) -- (A47);
\draw[pinline, shorten >=4pt] (hiddens) -- (v1);

\node[below=25pt,align=center] (output) at (h1) {Output\\ units};
\draw[pinline, shorten >=4pt] (output) -- (h1);

\begin{comment}

%%%%%%%%%%%%%%%%%         
% OUTPUT LAYER
%%%%%%%%%%%%%%%%%

\begin{scope}[canvas is xy plane at z=3.425]
\draw[image] (-2,-2) coordinate (A) -- (2,-2)
-- (2,2) coordinate (C) -- (-2,2) -- cycle;
\end{scope}

\node[xshift=-0.7cm, left] (lesion) at (A) {$S$};
\draw[pinline, shorten >=4pt] (lesion) -- (A);

\node[xshift=0.7cm, right] (y2) at (C) {$y^{(2)}$};
\draw[pinline, shorten <=4pt] (C) -- (y2);

\draw[decorate,decoration={brace,raise=75pt,mirror}] (B1-|C) --
node[right=80pt] {Convolutional\\ network} (G7-|C1);

\draw[decorate,decoration={brace,raise=95pt,mirror}] (G0-|C1) --
node[right=100pt] {Deconvolutional\\ network} (C);

\end{comment}

\end{tikzpicture}
\caption[Schematic illustration of a CNN]{Convolutional neural network with two
convolutional layers, one pooling layer and one dense layer. The activations of the last layer is the output of
the network.}
\label{fig:cnn}
\end{figure}

For multimodal 3D volumes, the neurons of convolutional and pooling layers are
arranged in a 4D lattice, where the first three dimensions correspond to the
dimensions of the input volume, and the forth dimension indexes
the input modality or channel. The activations of the output of a
convolutional layer are calculated by
\begin{equation}
x^{(l)}_j = f\Bigg(\sum_{i=1}^C\tilde{w}^{(l)}_{ij}*x^{(l-1)}_i
+ b^{(l)}_j\Bigg),
\end{equation}
where $l$ is the index of a convolutional layer, $x^{(l)}_j$ denotes the
$j$th channel of the output volume, $w^{(l)}_{ij}$ is a 3D filter
kernel connecting the $i$th channel of the input volume to the $j$th channel
of the output volume, and $b_j^{(l)}$ denotes the bias term of the $j$th output
channel. CNNs can be trained using stochastic gradient descent, where the
gradient can be derived analogously to dense neural networks and calculated using
backpropagation \citep{lecun1989,LeCun1998}. 

A major challenge for gradient-based optimization methods is the choice of an
appropriate learning rate. Classic stochastic gradient descent \citep{LeCun1998}
uses a fixed or decaying learning rate, which is the same for all parameters of
the model. However, the partial derivatives of parameters of different layers
can vary substantially in magnitude, which can require different learning rates.
In recent years, there has been an increasing interest in developing methods for
automatically choosing independent learning rates. Most methods (e.g., AdaGrad
\citep{duchi2011}, AdaDelta \citep{zeiler2012}, RMSprop \citep{dauphin2015}, and
Adam \citep{kingma2014}) collect different statistics of the partial derivatives
over multiple iterations and use this information to set an adaptive learning
rate for each parameter. This is especially important for the training of deep
networks, where the optimal learning rates often differ greatly for each layer.

% TODO: Also strided convolutional neural networks.

\subsection{Training and pre-training}

Problem with learning rate and stuff. Can be pre-trained.

\section{Learning from Unlabeled Data}

% More statistical learning and finding patterns of similarity.

One of the most important applications of deep learning is to learn a feature
hierarchy from unlabeled images. The key to learning such a hierarchy is the
ability of deep models to be trained layer by layer, where each layer acts as a
nonlinear feature extractor. Various methods have been proposed for feature
extraction from unlabeled images. In this section, we will first introduce the
restricted Boltzmann machines (RBMs) \citep{freund1992,hinton2010}, which are the
building blocks of deep belief networks (DBNs) \citep{Hinton2006b}, followed by
a short introduction to alternative feature extractors such as stacked denoising
autoencoders (SDAEs) \citep{vincent2010}.

% TODO: comparison to supervised methods similar to the introduction of the
% supervised methods section.

\subsection{From Restricted Boltzmann Machines to Deep Belief Networks}

% TODO: Illustrate with toy example.

An RBM is a probabilistic graphical models defined by a bipartite graph as shown
in Fig.~\ref{fig:rbm}. The units of the RBM are divided into two layers, one of
visible units $\vect{v}$ and the other of hidden units $\vect{h}$. There are no
direct connections between units within either layer. An RBM defines the joint
probability of visible and hidden units in terms of the energy $E$
\begin{align}
p(\vect{v}, \vect{h} \given \thetas) &=
\frac{1}{Z(\thetas)}e^{-E(\vect{v}, \vect{h} \given \thetas)}. \\
\intertext{When the visible and hidden units are binary, the energy is defined
as} 
-E(\vect{v}, \vect{h}\given \thetas) &= \sum_{i, j}v_i w_{ij} h_j +
\sum_i b_i v_i + \sum_j c_j h_j \\
&= \vect{v}^\textup{T}\vect{W}\vect{h} + \vect{b}^\textup{T}\vect{v} +
\vect{c}^\textup{T}\vect{h}
\end{align}
where $Z(\thetas)$ is a normalization constant, $\vect{W}$ denotes the weight
matrix that connects the visible units with the hidden units, $\vect{b}$ is a
vector containing the visible bias terms, $\vect{c}$ is a vector containing the
hidden bias terms, and $\thetas = \{\vect{W}, \vect{b}, \vect{c}\}$ are the
trainable parameters of the RBM.

\begin{figure}
\centering
\begin{tikzpicture}

\tikzstyle{gnode}=[shape=circle,draw=black]

\foreach \x in {1,...,3} {
  \node[gnode] (h\x) at (2*\x-4, 0) {$h_\x$};
}
\foreach \y in {1,...,5} {
  \node[gnode] (v\y) at (1.5*\y-4.5, -2) {$v_\y$};
}
\foreach \x in {1,...,3} {
  \foreach \y in {1,...,5} {
      \draw (h\x)--(v\y);
  }
}

\node[pin=120:$w_{11}$] at ($(v1)!.5!(h1)$) {};

\coordinate (rh) at (4,0);
\coordinate (rv) at (4,-2);

\node[fit=(v1)(v5)(rv),inner sep=0pt] (visibles) {};
\node[fit=(h1)(rh), inner sep=0pt] (hiddens) {};
\node[fit=(v1)(v3), inner xsep=0pt] (inputs) {};
\node[fit=(v4)(v5), inner xsep=0pt] (outputs) {};

\begin{scope}[decoration=brace]

\draw[decorate] ($(visibles.north east)-(0,2pt)$)--node[right=4pt]{Visible
units $\vect{v}$}
(visibles.south east);

\draw[decorate] (hiddens.north  east)--
node[right=4pt,align=center]{Hidden units $\vect{h}$}
($(hiddens.south east)+(0,2pt)$);

\draw[decorate] ($(hiddens.south east)-(0,2pt)$)
--node[right=4pt,align=left] {Weights $\vect{W}$ between\\ visibles and
hiddens} ($(visibles.north east)+(0,2pt)$);

% \draw[decorate] (inputs.south east)--
% node[below=4pt] {Input units: $\vect{x}$}
% (inputs.south west);
% \draw[decorate] (outputs.south east)--
% node[below=4pt] {Output units: $\vect{y}$}
% (outputs.south west);
\end{scope}

\end{tikzpicture}

\caption[Graph representation of an RBM with 3 hidden and 5 visible units]{%
Graph representation of an RBM with 3 hidden and 5 visible units.
An RBM models the joint probability of visible and hidden units. Edges between
vertices denote conditional dependence between the corresponding random
variables.}
  
\label{fig:rbm}
\end{figure}

\paragraph{Inference}
The hidden units represent patterns of similarity that can be observed in groups
of images. Once an RBM is trained, the features of a previously unseen image can
be extracted by calculating the expectation of the hidden units. The posterior
distribution of the hidden units given the visible units can be calculated by
\begin{equation}
\label{eq:dhgivenv}
p(h_j = 1 \given \vect{v}, \thetas) = \sigm(\vect{w}_{\cdot,j}^\text{T}\vect{v}
+ c_j),
\end{equation}
where $\vect{w}_{\cdot,j}$ denotes the $j$th column vector of $\vect{W}$, and
$\sigm(x)$ is the sigmoid function defined as $\sigm(x) = (1+ \exp(-x))^{-1}, x
\in \R$. An RBM is a generative model, which allows for the reconstruction of
an input signal given its features. This is achieved by calculating the expectation
of the visible units given the hidden units. The posterior distribution $p(v_i =
1 \given \vect{h}, \thetas)$ can be calculated by
\begin{equation}
\label{eq:dvgivenh}
p(v_i = 1 \given \vect{h}, \thetas) = \sigm(\vect{w}_{i,
\cdot}^\text{T}\vect{h} + b_i),
\end{equation}
where $\vect{w}_{i,\cdot}$ denotes the $i$th row vector of $\vect{W}$.
Reconstructing the visible units can be used to visualize the learned features.
To visualize the features associated with a particular hidden unit, all other
hidden units are set to zero and the expectation of the visible units is
calculated, which represents the pattern that causes a particular hidden
unit to be activated.

\paragraph{Training}

% \begin{itemize}
%   \item There are different training methods for RBMs.
%   \item Will focus on constrastive divergence.
%   \item Alternatives are stochastic bla with a reference
%   \item RBMs are trained using maximum likelihood estimation (MLE)
% \end{itemize}

RBMs can be trained by maximizing the likelihood or, more commonly, the log
likelihood of the training data, $\data = \{\vect{v}_n \given n \in [1, N] \}$,
which is called maximum likelihood estimation (MLE). The gradient of the log
likelihood function with respect to the weights, $\vect{W}$, is given by
\begin{equation}
\label{eq:mle}
\nabla_{\vect{W}} \log p(\mathcal{D} \given \thetas) =
\frac{1}{N} \sum_{n = 1}^N
\E[\vect{v}\vect{h}^\text{T} \given \vect{v}_n, \thetas]
-\E[\vect{v}\vect{h}^\text{T} \given \thetas].
\end{equation}
The first expectation can be estimated using a mean field approximation:
\begin{align}
\E[\vect{v}\vect{h}^\text{T} \given \vect{v}_n, \thetas] &\approx
\E[\vect{v} \given \vect{v}_n, \thetas]
\E[\vect{h}^\text{T} \given \vect{v}_n, \thetas], \\
&=\vect{v}_n\E[\vect{h}^\text{T} \given \vect{v}_n, \thetas].
\end{align}
The second expectation is typically estimated using a Monte Carlo
approximation
\begin{equation}
\E[\vect{v}\vect{h}^\text{T} \given \thetas] \approx
\frac{1}{S}\sum_{s=1}^{S}\vect{v}_s\vect{h}_s^\text{T},
\end{equation}
where $S$ is the number of generated samples, and $\vect{v}_s$ and $\vect{h}_s$
are samples drawn from $p(\vect{v}\given \thetas)$ and $p(\vect{h}\given
\thetas)$, respectively. Samples from an RBM can be generated efficiently using
block Gibbs sampling, in which the visible and hidden units are initialized
with random values and alternately sampled given the previous state using
\begin{align}
h_j &= \I(y_j < p(h_j = 1 \given \vect{v}, \thetas)) & \text{with $y_j \sim
\text{U}(0,1)$}\\
v_i &= \I(x_i < p(v_i \given \vect{h}, \thetas)) & \text{with $x_i \sim
\text{U}(0,1)$}
\end{align}
where $z \sim \text{U}(0,1)$ denotes a sample drawn from the uniform
distribution in the interval $[0,1]$ and $\I$~is the indicator function, which
is defined as $1$ if the argument is true and $0$ otherwise. After several
iterations, a sample generated by the Gibbs chain is distributed according to
$p(\vect{vh}\given \thetas)$.

If the Gibbs sampler is initialized at a data point from the training set and
only $1$ Monte Carlo sample is used to approximate the second expectation in
\eqref{eq:mle}, the learning algorithm is called contrastive divergence (CD)
\citep{hinton2002}. Alternatively, persistent CD (PCD) \citep{tieleman2008} uses
several separate Gibbs chains to generate data independent samples from the
model, which results in a better approximation of the gradient of the log
likelihood than CD. To speed up the training, the dataset is usually divided
into small subsets called mini-batches and a gradient step is performed for each
mini-batch. To avoid confusion with a gradient step, the term ``iteration'' is
generally avoided and the term ``epoch'' is used instead to indicate a sweep
through the entire dataset. Additional tricks to monitor and speed up the
training of an RBM can be found in Hinton's RBM training guide
\citep{hinton2010a}.

\paragraph{Deep belief networks}

A single RBM can be regarded as a nonlinear feature extractor. To learn a
hierarchical set of features, multiple RBMs are stacked and trained layer by
layer, where the first RBM is trained on the input data and subsequent RBMs are
trained on the hidden unit's activations computed from the previous RBM. The
stacking of RBMs can be repeated to initialize DBNs of any depth.

\paragraph{Alternative unit types}

To model real-valued inputs like the intensities of some medical images, the
binary visible units of an RBM are replaced with Gaussian visible units, which
leads to the following energy function
\begin{equation} 
-E(\vect{v}, \vect{h}\given \boldsymbol{\theta}) = \sum_{i,
j}\frac{v_i}{\sigma_i} w_{ij} h_j + \sum_i \frac{(v_i - b_i)^2}{2\sigma_i^2} +
\sum_j c_j h_j,
\end{equation}
where the mean of the $i$th visible unit is encoded in the bias term $b_i$,
and its standard deviation is given by $\sigma_i$. Although approaches have been
proposed to learn the standard deviation \citep{cho2011}, the
training data is often simply standardized to have zero mean and unit variance,
which yields to the following simplification for the inference of the visible
and hidden units:
\begin{align} 
\label{eq:ghgivenv}
\E[h_j \given \vect{v}, \thetas] &=
\sigm(\vect{w}_{\cdot,j}^\text{T}\vect{v} + c_j),\\
\label{eq:gvgivenh}
\E[v_i \given \vect{h}, \thetas] &= \vect{w}_{i,\cdot}^\text{T}\vect{h} +
b_i.
\end{align}

A binary hidden unit can only encode two states. In order to increase the
expressive power of the hidden units, Nair et al. proposed using noisy rectified
linear units (NReLU) \citep{nair2010} as the hidden units, and showed that this
can improve the learning performance of RBMs. The signal of an NReLU is the sum
of an infinite number of binary units all of which having the same weights but
different bias terms. In the special case where the offsets of their bias
terms are set to $-0.5, -1.5, \dotsc$, the sum of their probabilities and
therefore the expectation of an NReLU is extremely close to having a closed form:
\begin{align}
\E[h_j \given \vect{v}, \thetas] &=
\sum_{i=1}^\infty \sigm(\vect{w}_{\cdot,j}^\text{T}\vect{v} + c_j - i + 0.5),\\
&\approx \log(1+\exp(\vect{w}_{\cdot,j}^\text{T}\vect{v} + c_j)).
\end{align}
However, sampling of this type of unit involves the repeated calculation of the
sigmoid function, which can be time-consuming. If a sample is not constrained
to being an integer, a fast approximation can be calculated with
\begin{align} 
h_j &\sim \max(0, \mu_j + \Norm(0, \sigm(\mu_j))), \\
\mu_j &= \vect{w}_{\cdot,j}^\text{T}\vect{v} + c_j,
\end{align}
where $\Norm(0, \sigma^2)$ denotes Gaussian noise.

\subsection{Convolutional Models}

A potential drawback of DBNs is that the learned features are location
dependent. Hence, features that can occur at many different locations in an
image, such as edges and corners, have to be relearned for every possible
location, which dramatically increases the number of features required to
capture the content of large images. To increase the translational invariance of
the learned features, Lee et al. introduced the convolutional deep belief
network (convDBN) \citep{lee2009,lee2011}. In a convDBN, the units of one layer
are only connected to the units of a sub-region of the previous layer, where
each unit shares the same weights with all other units of the same layer. This
greatly reduces the number of trainable weights, which reduces the risk of
overfitting, reduces the memory required to store the model parameters, speeds
up the training, and thereby facilitates the application to high-resolution
images.

We assume the input images to be square 2D images, but the model generalizes
with little modification to non-square images of higher dimensions.

A convDBN consists of alternating convolutional and pooling layers, which are
followed by one or more dense layers. Each convolutional layer of the model can
be trained in a greedy layerwise fashion by treating it as a convolutional
restricted Boltzmann machine (convRBM). The energy of a convRBM is defined as
\begin{align} 
E(\vect{v},\vect{h}) 
% &= -\sum_{i=0}^{N_\text{c}-1} \sum_{j=0}^{N_\text{k}-1}
% \sum_{x,y=0}^{N_\text{h}-1} \sum_{u,v=0}^{N_\text{w}-1} h_{xy}^jw_{uv}^{ij}v_{x+u, y+v}^i
% - \sum_{i=0}^{N_\text{c}-1}b_i\!\sum_{x,y = 0}^{N_\text{v}-1}\!v_{xy}^i
% - \sum_{j=0}^{N_\text{k}-1}c_j\!\sum_{x,y = 0}^{N_\text{h}-1}\!h_{xy}^j \\
&= -\sum_{i=1}^{N_\text{c}} \sum_{j=1}^{N_\text{k}} \vect{h}^{(j)}
\bullet (\tilde{\vect{w}}^{(ij)} * \vect{v}^{(i)}) -
\sum_{i=1}^{N_\text{c}}b_i\sum_{x,y = 1}^{N_\text{v}}\!v_{xy}^{(i)} -
\sum_{j=1}^{N_\text{k}}c_j\sum_{x,y = 1}^{N_\text{h}}\!h_{xy}^{(j)}.
\end{align}
The key terms and notation are defined in Table~\ref{tab:notation}. At the first
layer, the number of channels $N_\text{c}$ is equal to the number of input
modalities. For subsequent layers, $N_\text{c}$ is equal to the number of
filters of the previous layer.

\begin{table} 
\caption{Key variables and notation. For notational simplicity,
we assume the input images to be square 2D images.}
\label{tab:notation}
\begin{center}
\begin{tabular}{@{}cL{10cm}@{}}
\toprule
Symbol & Description \\
\cmidrule(r){1-1}\cmidrule(l){2-2}
$\vect{v}^{(i)}$ & $i$th input channel \\
$\vect{h}^{(j)}$ & $j$th output channel or feature map \\
$\vect{w}^{(ij)}$ & weights of filter kernels connecting visible units
$\vect{v}^{(i)}$ to hidden units $\vect{h}^{(j)}$ \\
$b_i$ & bias terms of visible units \\
$c_j$ & bias terms of hidden units \\
$N_\text{c}$ & number of channels of the visible units \\
$N_\text{v}^2$ & number of visible units per channel \\
$N_\text{k}$ & number of filters and feature maps \\
$N_\text{h}^2$ & number of hidden units per feature map \\
%$N_\text{w}^2$ & number of weights per filter kernel \\
$\bullet$ & element-wise product followed by summation \\
$*$ & valid convolution \\
$\circledast$ & full convolution \\
$\tilde{\vect{w}}^{(ij)}$ & horizontally and vertically flipped version of
$\vect{w}^{(ij)}$ \\
\bottomrule
\end{tabular}
\end{center}
\end{table}
The posterior distributions $p(\vect{h} \given \vect{v})$ and $p(\vect{v} \given
\vect{h})$ can be derived from the energy equation and are given by
\begin{align}
p(h_{xy}^{(j)} = 1 \given \vect{v}) &= \sigm\Big(\sum_{i=0}^{N_\text{c}-1}
(\tilde{\vect{w}}^{(ij)}*\vect{v}^{(i)})_{xy} + c_j\Big), \\ 
p(v_{xy}^{(i)} = 1 \given \vect{h}) &= \sigm\Big(\sum_{j=0}^{N_\text{k}-1}
(\vect{w}^{(ij)} \circledast \vect{h}^{(j)})_{xy} + b_i\Big).
\end{align}
To train a convRBM on a set of images $\data = \{\vect{v}_n \given n \in
[1,N]\}$, the weights and bias terms can be learned by CD. During each iteration
of the algorithm, the gradient of each parameter is estimated and a gradient
step with a fixed learning rate is applied. The gradient of the filter weights
can be approximated by
\begin{equation}
\Delta \vect{w}^{(ij)} \approx
\frac{1}{N}(\vect{v}^{(i)}_n*\tilde{\vect{h}}^{(j)}_n -
\vect{v}'^{(i)}_n*\tilde{\vect{h}}_n'^{(j)}),
\label{eq:convgra}
\end{equation}
where $\vect{h}_n^{(j)}$ and $\vect{h}'^{(j)}_n$ are samples drawn from
$p(\vect{h}^{(j)} \given \vect{v}_n)$ and $p(\vect{h}^{(j)} \given
\vect{v}'_n)$, and $\vect{v}'^{(i)}_n = \E[\vect{v}^{(i)} \given \vect{h}_n]$.

Different types of operations \citep{scherer2010} have been
proposed for the pooling layers, with the common goal of creating a more compact
representation of the input data. The most commonly used type of pooling is
max-pooling. Therefore, the input to the pooling layer is divided into small
blocks and only the maximum value of each block as passed on to the next layer,
which makes the representation of the input invariant to small
translations in addition to reducing its dimensionality.

\subsection{Strided Convolutional Models}

% Why strided! Less memory, faster, no pooling required. Better correspondence
% to dense DBNs. Draw from the journal paper.

Strided convolutions are a type of convolution that shifts the filter kernel as
a sliding window with a step size or stride $s > 1$, stopping at only $N_\text{v}
/ s$ positions. This reduces the number of hidden units per channel to
$N_\text{h} = N_\text{v} / s$, hence significantly reducing training times and
memory required for storing the hidden units during training. The energy of a
strided convolutional RBM (sconvRBM) is defined as
\begin{equation} 
E(\vect{v},\vect{h}) = 
-\sum_{i=0}^{N_\text{c}-1}\sum_{j=0}^{N_\text{k}-1}\sum_{x,y=0}^{N_\text{h}-1}
\sum_{u,v=-\lfloor N_\text{w}/2\rfloor}^{\lfloor(N_\text{w}-1)/2\rfloor}
\hspace{-1.2em}h_{xy}^jw_{uv}^{ij}v_{sx+u, sy + v}^i -
\sum_{i=0}^{N_\text{c}-1}b_i\!\sum_{x,y = 0}^{N_\text{v}-1}\!v_{xy}^i -
\sum_{j=0}^{N_\text{k}-1}c_j\!\sum_{x,y = 0}^{N_\text{h}-1}\!h_{xy}^j
\end{equation}  

\subsection{Alternative Feature Learning Approaches}

Stacked auto encoders and denoising autoencoders without math. Independent
component analysis and independent subspace analysis. Need to read about this.
