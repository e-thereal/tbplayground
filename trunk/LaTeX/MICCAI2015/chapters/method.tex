\section{Methods}
\label{sec:method}

In this paper, the task of segmenting MS lesions is defined as finding a
function $s$ that maps multi-modal images $I$, e.g., $I = (I_\text{flair},
I_\text{t1}, I_\text{t2})$, to corresponding lesion masks $S$.
Given a set of training images $I_n$, $n \in \N$, and corresponding
segmentations $S_n$, we model finding an appropriate function for segmenting
MS lesions as an optimization problem of the following form
\begin{equation}
\hat{s} = \arg \max_{s \in \mathcal{S}} \sum_n \text{sim}(S_n, s(I_n)).
\label{eq:segprob}
\end{equation}
where $\text{sim}$ denotes a function that calculates the similarity between
ground truth segmentations and predicted segmentations, and $\mathcal{S}$ is the
set of possible segmentation functions.

% where $x^{(1)}_1 = I_\text{flair}$, $x^{(1)}_2 = I_\text{t1}$, $x^{(1)}_3 =
%  I_\text{t2}$, and $x^{(1)}_4 = I_\text{prior}$.

% \paragraph{Problem definition}
% \begin{itemize}
% 
% \item Given a set of multi-modal images $I_n = (I_{n, \text{flair}})$ and
% corresponding segmentations $S_n$, we define the task of finding an appropriate
% segmentation algorithm, as the task of finding a function $s$ that maps images
% to their segmentations, i.e. $S_n = s(I_n)$.
% 
% \item Segmentation is a function $s$ that maps an image $I$ to its segmentation
% $S$. That is $S = s(I)$.
% 
% \item Given a set of training images $I_i$ and corresponding segmentations
% $S_i$, we treat finding an appropriate function for segmenting T2 lesions as an
% optimization problem of the following form:
%  \begin{equation} 
% \hat{s} = \arg \max_{s \in \mathcal{S}} \sum_n \text{sim}(S_n, s(I_n)).
% \label{eq:segprob}
% \end{equation}
% where $\text{sim}$ denotes a function that calculates the similarity between
% ground truth segmentations and predicted segmentations, and $\mathcal{S}$ is the
% set of possible functions.
% \end{itemize}

\begin{figure}[tb]
\centering
\begin{tikzpicture}[%
scale=0.65,
x  = {(0.9cm,-0.1cm)},
y  = {(0.33cm,0.23cm)},
z  = {(0cm,1cm)}]

\tikzstyle{every node}=[font=\small, inner sep=3pt, align=center]
\tikzstyle{every pin}=[align=center,fill=white]
\tikzstyle{dbnlabel}=[font=\sffamily\normalsize]

\tikzstyle{image}=[fill=white, fill opacity=0.75]
\tikzstyle{pinline}=[thin, gray]
\tikzstyle{kernelline}=[very thin]


%%%%%%%%%%%%%%%%%         
% INPUT LAYER
%%%%%%%%%%%%%%%%%

\foreach \x in {1, ..., 3} {
\begin{scope}[canvas is xy plane at z=0.25*\x]
\draw[image] (-2,-2) coordinate (A\x) -- (2,-2)
coordinate (B\x) -- (2,2) coordinate (C\x) -- (-2,2) -- cycle;

\draw[kernelline] 
      (-1.6, -1.6) \ifnum\x=3 \else -- +(0, 0, 0.25) +(0,0) \fi
   -- (-0.8, -1.6) \ifnum\x=3 \else -- +(0, 0, 0.25) +(0,0) \fi
   -- (-0.8, -0.8) \ifnum\x=3 \else -- +(0, 0, 0.25) +(0,0) \fi
   -- (-1.6, -0.8) \ifnum\x=3 \else -- +(0, 0, 0.25) +(0,0) \fi
   -- (-1.6, -1.6);
\ifnum\x=3
\draw[kernelline] (-1.6, -1.6) -- (-1.2, -1.2, 0.9);
\draw[kernelline] (-1.6, -0.8) -- (-1.2, -1.2, 0.9);
\draw[kernelline] (-0.8, -0.8) coordinate(c3) -- (-1.2, -1.2, 0.9);
\draw[kernelline] (-0.8, -1.6) -- (-1.2, -1.2, 0.9);
\fi 

\end{scope}
}

\node[xshift=-0.7cm, yshift=10pt, left] (flair) at (A3) {$I_\text{FLAIR}$};
\node[xshift=-0.7cm, yshift=3pt, left] (t1w) at (A2) {$I_\text{T1}$}; 
\node[xshift=-0.7cm, yshift=-3pt, left] (t2w) at (A1) {$I_\text{T2}$};
% \node[xshift=-0.7cm, yshift=-10pt, left] (prior) at (A0) {$I_\text{prior}$};

\draw[pinline, shorten >=4pt] (flair) -- (A3);
\draw[pinline, shorten >=4pt] (t1w) -- (A2);
\draw[pinline, shorten >=4pt] (t2w) -- (A1);
% \draw[pinline, shorten >=4pt] (prior) -- (A0);

% \draw[decorate,decoration={brace,raise=15pt,mirror}] (C1) --
% node[right=20pt] {$x^{(1)}_i$} (C3);

\node[xshift=0.5cm, right] (weights) at (c3) {$w^{(1)}_{ij}$};
\draw[pinline, shorten >=4pt] (weights) -- (c3);

%%%%%%%%%%%%%%%%%         
% HIDDEN LAYER
%%%%%%%%%%%%%%%%%

\foreach \x in {0, ..., 7} {
\begin{scope}[canvas is xy plane at z=0.125*\x+1.65]
\draw[image] (-1.6,-1.6) -- (1.6,-1.6)
-- (1.6,1.6) coordinate (G\x) -- (-1.6,1.6) -- cycle;

\draw[kernelline]
      (-1.2, -1.2) \ifnum\x=7 \else -- +(0, 0, 0.125) +(0,0) \fi
   -- (-0.4, -1.2) \ifnum\x=7 \else -- +(0, 0, 0.125) +(0,0) \fi
   -- (-0.4, -0.4) \ifnum\x=7 \else -- +(0, 0, 0.125) +(0,0) \fi
   -- (-1.2, -0.4) \ifnum\x=7 \else -- +(0, 0, 0.125) +(0,0) \fi
   -- (-1.2, -1.2);
\ifnum\x=7
\draw[kernelline] (-1.2, -1.2) -- (-0.8, -0.8, 0.9);
\draw[kernelline] (-1.2, -0.4) -- (-0.8, -0.8, 0.9);
\draw[kernelline] (-0.4, -0.4) coordinate (g7) -- (-0.8, -0.8, 0.9);
\draw[kernelline] (-0.4, -1.2) -- (-0.8, -0.8, 0.9);
\fi 

\end{scope}
}

% \draw[decorate,decoration={brace,raise=15pt,mirror}] (G0-|C1) --
% node[right=20pt] {$x^{(2)}_j = y^{(1)}_j$} (G7-|C1);

\node[xshift=0.5cm, yshift=-1pt, right] (weights2) at (g7) {$w^{(2)}_{j}$};
\draw[pinline, shorten >=4pt] (weights2) -- (g7);


%%%%%%%%%%%%%%%%%         
% OUTPUT LAYER
%%%%%%%%%%%%%%%%%

\begin{scope}[canvas is xy plane at z=3.425]
\draw[image] (-2,-2) coordinate (A) -- (2,-2)
-- (2,2) coordinate (C) -- (-2,2) -- cycle;
\end{scope}

\node[xshift=-0.7cm, left] (lesion) at (A) {$S$};
\draw[pinline, shorten >=4pt] (lesion) -- (A);

% \node[xshift=0.7cm, right] (y2) at (C) {$y^{(2)}$};
% \draw[pinline, shorten <=4pt] (C) -- (y2);

\draw[decorate,decoration={brace,raise=15pt,mirror}] (B1-|C) --
node[right=20pt] {Convolutional\\ layer} (G7-|C1);

\draw[decorate,decoration={brace,raise=35pt,mirror}] (G0-|C1) --
node[right=40pt] {Deconvolutional\\ layer} (C);

\end{tikzpicture}
\caption{Convolutional encoder network used to segment MS lesion of images from
the MICCAI 2008 lesion segmentation challenge. The first two layers form a
convolutional neural network, and the last two layers form a deconvolutional
neural network.}
\label{fig:network}
\end{figure}

The set of possible segmentation functions is modeled by the convolutional
encoder network illustrated in Figure~\ref{fig:network}. Our network consists of
two layers, a convolutional layer that extracts automatically learned features
from multi-modal images, and a deconvolutional layer that uses the extracted
features to segment MS lesions. The convolutional layer is a deterministic
function of the following form
\begin{equation}
y^{(1)}_j = \max \left(0, \sum_{i=1}^{C}\tilde{w}^{(1)}_{ij}*x^{(1)}_i +
b^{(1)}_j\right)
\end{equation}
where $x^{(1)}_i$ denotes an image representing the $i$th modality, 
$y^{(1)}_j$ denotes the feature map corresponding to the $j$th feature, $w_{ij}$
and $b_j \in \R$ are trainable parameters of the model, $*$ denotes
valid convolution, and $\tilde{w}_{ij}$ denotes a flipped version of $w_{ij}$.
The convolutional and deconvolutional layer are connected by setting the inputs
of the deconvolutional layer to the outputs of the convolutional layer, i.e.
$x^{(2)}_j = y^{(1)}_j$. The output of the deconvolutional layer can be
calculated with
\begin{equation}
y^{(2)} = \sigm\left(\sum_{j=1}^Fw^{(2)}_{j}\circledast x^{(2)}_j +
b^{(2)}\right)
\end{equation}
where $w^{(2)}_j$ and $b^{(2)}$ are trainable parameters, and $\circledast$
denotes full convolution.\todo{$F$ is the number of features. Also add $C$ as
the number of input channels or modalities to the convolutional layer.}

\paragraph{Class of Segmentation Functions}
\begin{itemize}
% \item We propose to use neural networks to model the class of allowed
% segmentation functions.
% \item Neural networks are very flexible and able to learn highly non-linear
% functions, which makes them suitable for image segmentation.
% \item Can be efficiently learned from training data using gradient descent.
% \item Architecture of the neural network can be used to regularize the functions
% that can be learned
% \item We propose a combined convolutional-deconvolutional neural network with
% one convolutional and one deconvolutional layer as illustrated in
% Figure~\ref{fig:network}
% \item Convolutional neural network because it scales better to high-resolution
% images and is better regularized than dense networks
% \item Filter size can regularize local support.
% \item Inference in the convolutional and deconvolutional layer. The first
% layer input units are set to the voxel intensities of the input image
% $x^{(1)}_1 = I_\text{flair}$, $x^{(1)}_2 = I_\text{t1}$,
% $x^{(1)}_3 = I_\text{t2}$, and $x^{(1)}_4 = I_\text{prior}$.
% \begin{equation}
% y^{(1)}_j = \max \left(0, \sum_{i=1}^{4}\tilde{w}^{(1)}_{ij}*x^{(1)}_i +
% b^{(1)}_j\right)
% \end{equation}
% The input units of the deconvolutional layer are set to the outputs
% of the convolutional layer $x^{(2)}_j = y^{(1)}_j$, and $S = y^{(2)} > t$, where
% $t \in \R$ is the lesion threshold.
% \begin{equation}
% y^{(2)} = \sigm\left(\sum_{j=1}^Fw^{(2)}_{j}\circledast x^{(2)}_j +
% b^{(2)}\right)
% \end{equation}
\item The parameters can be trained by back-propagation using the delta rule
\begin{align}
% Error function
E &= \frac{1}{2}\sum_{\vect{p}}\left(S(\vect{p}) - y^{(2)}(\vect{p})\right)^2
\\
% Delta update
\delta^{(2)} &= \big(y^{(2)} -S\big)y^{(2)}\big(1-y^{(2)}\big)
\end{align}
\begin{align}
% Weight update
\frac{\partial E}{\partial w^{(2)}_j} &= \delta^{(2)} * \tilde{x}^{(2)}_j \\
% Bias update
\frac{\partial E}{\partial b^{(2)}} &= \frac{1}{N^3}\sum_{\vect{p}}
\delta^{(2)}(\vect{p})
% Convolutional model
\end{align}
And for the convolutional layer
\begin{equation}
% Delta update
\delta^{(1)}_j = \big(w^{(2)}_{j}\circledast\delta^{(2)}\big)\I\big(y^{(1)}_j >
0\big)
\end{equation}
\begin{align}
% Weight update
\frac{\partial E}{\partial w^{(1)}_{ij}} &= x^{(1)}_i * \tilde{\delta}^{(1)}_j\\
% Bias update
\frac{\partial E}{\partial b^{(1)}_j} &= \frac{1}{M^3}\sum_{\vect{q}}
\delta^{(1)}_j(\vect{q})
\end{align}
% \item Relationship to general segmentation problem
% \begin{align}
% \vect{x} &= I \\
% \vect{y} &= S 
% \end{align}
% \item Gradients:
% \begin{align}
% \delta^{(n_l)}&=-(y-a^{(n_l)})f'(z^{(n_l)}) \\
% \delta^{(l)}&=((W^{(l)})^T\delta^{(l+1)})f'(z^{(l)})\\
% \nabla_{W^{(l)}}SSD(W,b;x,y)&=\delta^{(l+1)}(a^{(l)})^T\\
% \nabla_{b^{(l)}}SSD(W,b;x,y)&=\delta^{(l+1)}
% \end{align}
\end{itemize}

\paragraph{Similarity Measure}
\begin{itemize}
\item SSD is problematic for unbalanced classification tasks as the learning will
greatly favor one class.
\item To overcome this problem, we calculate the mean squared difference for
lesion and non-lesion voxels separately and then calculate the weighted sum of
the two terms form the final error measure
\begin{multline} 
E = r_\text{sen}\frac{\textstyle\sum_{\vect{p}} \left(S(\vect{p}) -
y^{(2)}(\vect{p})\right)^2 S(\vect{p})}{\textstyle\sum_{\vect{p}} S(\vect{p})}
 \\  +
(1-r_\text{sen})\frac{\textstyle\sum_{\vect{p}} \left(S(\vect{p}) -
y^{(2)}(\vect{p})\right)^2 \big(1 - S(\vect{p})\big)}{%
\textstyle\sum_{\vect{p}}\big(1 - S(\vect{p})\big)}
\end{multline}
In the binary case, the first term is equal to 1 minus the sensitivity and
the second term is equivalent to 1 minus the specificity. We will therefore call
these to terms sensitivity and specificity error.

\item The sensitivity ratio $r_\text{sen}$ can be used to assign different
weights to the two terms. Due to the large number of non-lesion voxels,
weighting the specificity higher than the sensitivity gives better segmentation
results in practice. We've found that a sensitivity ratio between 0.1 and 0.01
gives works best in practice. While the actual choice of the sensitivity ratio
has a big impact on the optimal threshold, the segmentation result is not
sensitive to the sensitivity ratio.

% \begin{align} 
% \alpha &= \frac{2r_\text{sen}}{\textstyle\sum_{\vect{p}}S(\vect{p})}
% &\text{and}& & \beta &= \frac{2(1 - r_\text{sen})}{\textstyle\sum_{\vect{p}}(1 - S(\vect{p}))}
% \end{align}

\item Equations (6) to (10) are a consequence of the chain rule of derivatives
and independent of the chosen similarity measure, so we only need to derive the
new update rule for $\delta^{(2)}$. With $\alpha = 2r_\text{sen}
(\sum_{\vect{p}}S(\vect{p}))^{-1}$ and $\beta = 2(1 -
r_\text{sen})(\sum_{\vect{p}}(1 - S(\vect{p})))^{-1}$ we can rewrite $E$ as
\begin{align} 
E &= \frac{1}{2}\sum_{\vect{p}} \left(S(\vect{p}) - y^{(2)}(\vect{p})\right)^2
\alpha S(\vect{p}) +
\frac{1}{2}\sum_{\vect{p}} \left(S(\vect{p}) - y^{(2)}(\vect{p})\right)^2
\beta\big(1 - S(\vect{p})\big) \\
 &= \frac{1}{2}\sum_{\vect{p}} \big(\alpha S(\vect{p}) +
 \beta(1 - S(\vect{p}))\big)
 \left(S(\vect{p}) - y^{(2)}(\vect{p})\right)^2
\end{align}
The first term does not depend on $y$ and is therefore constant with respect to
the model parameters. The second term is identical to the sum of squared
difference objective function. The derivatives and therefore delta 2 is
also identical to the SSD, whereby the constant term is carried over to the
delta update as follows
\begin{equation} 
\delta^{(2)} = \big(\alpha S + \beta (1 - S)\big)\big(y^{(2)} - S\big) y^{(2)}
\big(1 - y^{(2)}\big)
\end{equation}

\end{itemize}

\paragraph{Choosing the right threshold}

\begin{itemize}
\item A single threshold is chosen such that the DSC is maximized on the
training set and the same threshold is used for the evaluation on the test set.
\end{itemize}

% \paragraph{Prevent Overfitting}
% \begin{itemize}
% \item Two main sources: 1) bias terms highly tuned to lesion locations observed
% in the training data, and 2) filters tuned to the intensity range observed in
% the training data
% \item Use shared bias terms and add lesion prior calculated from a large data
% set plus smoothing
% \item Use data augmentation, i.e. during training, randomly change the
% brightness contrast and gamma correction of the training images to artificially
% increase the intensity variability in the training set
% \end{itemize}
