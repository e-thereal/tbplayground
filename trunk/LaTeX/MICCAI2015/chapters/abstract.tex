\begin{abstract}

In this paper, we propose a novel approach for segmenting hyperintense T2
lesions in magnetic resonance images of multiple sclerosis patients. In contrast
to the recently proposed patch-based classification approaches, our method
performs segmentation of the entire image at once. Segmentation is finding a
function that maps images to lesion masks. Advantages: 1) don't need to select
patches, 2) scales better to large images than patch-based approaches, 3)
combines feature-learning and classification phase which allows the supervised
fine-tuning of features. We have evaluated our method on publicly available data
set from the MICCAI 2008 MS lesion segmentation challenge, to allow a direct
comparison of our method with state-of-the-art lesion segmentation methods. Our
method consistently outperforms the current state-of-the-art method on that data
set.

\begin{keywords}
Segmentation, T2 lesion, multiple sclerosis, machine learning, unbalanced
classification, deep learning, convolutional neural networks
\end{keywords}

\end{abstract}
