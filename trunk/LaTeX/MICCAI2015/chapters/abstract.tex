\begin{abstract}

We propose a novel segmentation approach based on deep convolutional encoder
networks and apply it to the segmentation of multiple sclerosis (MS) lesions in
magnetic resonance images (MRIs). Our model is a neural network that is both
convolutional and deconvolutional, and combines feature extraction and
segmentation prediction in a single model. The joint training of the feature
extraction and prediction layers allows the model to automatically learn
features that are optimized for accuracy for any given combination of image
types and application. In contrast to existing automatic feature learning
approaches, which are typically patch-based, our model learns features from
entire images, which eliminates patch selection and reduces redundant
calculations at the overlap of neighboring patches and thereby speeds up the
training. This allows our model to be trained on larger data sets in order to
learn features that cover the broad spectrum of lesion variability. We have
evaluated our method on the publicly available labeled cases from the MS Lesion
Segmentation Challenge 2008 data set, showing that our method performs
comparably to the state-of-the-art even when a relatively small data set is used
for training, which is typically not the strength of neural networks. In
addition, we have evaluated our method on 500 images (split equally into
training and test sets) from a data set from an MS clinical trial, showing that
the segmentation performance can be greatly improved by having a representative
training set.

% A challenge of using the entire images is that unbalanced classes can not be
% rebalanced as part of the patch selection step, which would make learning
% unstable. To overcome this problem, we propose to the a weighted sum of
% sensitivity and specificity error as the objective function for training and
% show how to train neural networks with this new objective function.

\begin{comment} 
- noval method for MS lesion segmentation
- automatically learn features tuned for lesion segmentation
- scales well to large data sets $\Rightarrow$ can take advantage of large data
sets - combination of convolutional and deconvolutional neural network
- first layer performs feature extraction, while the second layer performs
classification
- Both steps in a single model allows the tuning of the feature extraction step
- Works on entire images instead of image patches, which eliminates redundant
calculations on the overlap of patches, which makes it faster than patch-based
approaches
- SSD is problematic for unbalanced classes.
- Propose new similarity measure for neural networks and show how to train
neural networks with the alternative similarity measure
- Evaluated on a publicly available data set to allow a direct comparison with
other state-of-the-art methods
- Among the best performing methods on that data set
- Show that can take advantage on large internal data sets of an MS clinical
trial
- Average DSC of \SI{58}{\percent}.

In this paper, we propose a novel method for segmenting MS lesions that can
automatically learn features tuned for lesion segmentation and scales better to
large data sets of high-resolution 3D images than previous patch-based feature
learning approaches, which allows our model to take advantage of large data
sets. The proposed model is a combination of a convolutional \cite{LeCun1998}
and a deconvolutional neural network \cite{zeiler2011}. The first layer is a
convolutional layer that extracts features from multi-modal MRIs at each voxel
location. The second layer is a deconvolutional layer that uses the extracted
features from the first layer to classify each voxel of the image in a single
operation. Both layers are trained at the same time, which facilitates the
learning of features that are tuned for lesion segmentation. A key difference to
the network of Ciresan et. al \cite{Ciresan2012} is that our model is trained on
the entire images instead of multiple patches from the same image, which speeds
up the training and eliminates the need to select representative patches. The
proposed network is similar in architecture to a convolutional auto-encoder
\cite{masci2011} but instead of learning a lower dimensional representation of
the input images themselves, the output of your network are the predicted lesion
masks. Due to the structural similarity to convolutional auto-encoders, we will
call our model a convolutional encoder network (CEN). Traditionally, neural
networks are trained by back-propagating the sum of squared differences (SSD) of
the predicted and the expected output. However, if one class is much
overrepresented, as is the case for lesion segmentation, the algorithm would
learn to ignore the minority class completely. To overcome this problem, we
propose to use the weighted sum of sensitivity and specificity error as a new
objective function, which is suitable to deal with very unbalanced
classification problems, and we will derive the gradients of our proposed
objective function in order to train the model using stochastic gradient
descent.

In this paper, we propose a novel approach for segmenting brain lesions in
magnetic resonance images (MRIs) of multiple sclerosis (MS) patients. Our model
is a combination of a convolutional and deconvolutional neural network, which
combines feature extraction and classification in a single model. In
contrast to the recently proposed patch-based segmentation approaches, our
method performs segmentation of the entire image in a single feed-forward pass
through the network, without the need to extract patches to segment each voxel
individually. This approach makes it much faster than comparable patch-based
approach that use convolutional neural networks, which allows it to scale much
better to large data sets of three-dimensional images.

This approach makes segmentation and training much faster, which
allows the method to scale  Advantages:
1) don't need to select patches, 2) scales better to large images than patch-based
approaches, 3) automatically learned features, no predefined features, and 4)
combines feature-learning and classification phase which allows the supervised
fine-tuning of features. We have evaluated our method on publicly available data
set from the MICCAI 2008 MS lesion segmentation challenge, to allow a direct
comparison of our method with state-of-the-art lesion segmentation methods. Our
method performs en par with the state-of-the-art method on that data set. In
addition, our model scales to large data set and can leverage the large amounts
of labelled data achieving a DSC of \SI{57}{\percent} in that case.

\end{comment}

\begin{keywords}
Multiple sclerosis lesions, segmentation, MRI, machine learning, unbalanced
classification, deep learning, convolutional neural nets
\end{keywords}

\end{abstract}
