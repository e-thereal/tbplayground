\sisetup{
  round-mode = places,
  round-precision = 3,
  exponent-product = \cdot,
  detect-weight=true,
  detect-inline-weight=math,
  tight-spacing = false,
  table-align-text-post = false
}%

\section{Experiments and Results}

\paragraph{Experiments to come}

\begin{itemize}
  
\item Comparison with state-of-the-art methods. I will compare our method to
Geremia, dictionary learning and Souplet in terms of TPR, PPV, and DSC. Need to
check if I can get Tianming Zhan and  BRICQ St\'ephanie papers. Can't compare to
Xavier Tomas-Fernandez, because he didn't compute TPR, PPV, or DSC.

\item Sensitivity analysis. Will be performed with best parameters as baseline.
Parameters to be varied are: a) sensitivity ratio, b) number of epochs for
training, c) filter size, d) number of filters, and e) number of training cases.
In all cases, I will compare the influence on the training and testing DSC and
the best threshold.

\item Does the train performance correlate with the lesion load? Visualize good,
average and bad example. If it correlates with lesion load, visualize examples
with high, average, and low lesion load.

\item I need to get my challenge ranking. Inclusion of these results is
optional.

\item \emph{Optional}\quad Stride-1 tests: Requires memory efficient bias term
handling for shared bias terms. Might yield improved results and nicer filters
and I don't need to worry about explaining strided convolutions.

\item \emph{Optional}\quad Visualization of filters.

\end{itemize}



The algorithm will be evaluated on the BioMS data set, some other internal data
sets, the MICCAI 2008 MS Lesion Segmentation Challenge data set, and the ISBI
2015 MS Lesion Segmentation Challenge data set. We can't do much with BioMS,
because it doesn't have FLAIR, which is required by other freely available
methods. If we can get ground truth for data sets that have FLAIR, we can
compare it against Lesion Toad and WMLS from UPenn. In any case, we can compare
our method with other methods using the data provided by the segmentation
challenges. We will use the following metrics: DSC, Sensitivity, Specificity,
and PPV.
\begin{table}[ht]
\def\tabspace{12pt}
\sisetup{
  round-precision = 2,
}%
\caption{Comparison of state of the art methods with our method.}
\centering
\begin{tabular}{l%
@{\hspace{\tabspace}}S[table-format=2.2]
@{\hspace{\tabspace}}S[table-format=2.2]
@{\hspace{\tabspace}}S[table-format=2.2]
}
\toprule
Method & {TPR} & {PPV} & {DSC} \\ 
\midrule
Souplet et. al & 20.65 & 30.00 & {---} \\ 
Weiss et. al & 33.00 & 36.85 & 29.05 \\ 
Geremia et. al & 39.85 & 40.35 & {---}  \\ 
Our method & 41.49 & 36.90 & 34.20 \\ 
\bottomrule
\end{tabular}
\end{table}

% \begin{table}
% \caption{Segmentation results measured using the Dice Similarity Coefficient.
% Two layer auto-encoder, stride size of \num{2x2x1}, 32 filters, sensitivity
% ratio 0.05. Threshold finding methods: a) using a global threshold that
% optimizes the average DSC of the entire data set, b) using the optimal
% thresholds for each sample, and c) using predicted thresholds for each sample.}
% \label{tab:segmentation}
% \centering
% \def\tabspace{12pt}
% % \small
% \begin{tabular}{@{}l%
% @{\hspace{\tabspace}}S[table-format=1.3]%
% @{\hspace{\tabspace}}S[table-format=1.3]
% @{\hspace{\tabspace}}S[table-format=1.3]
% @{\hspace{\tabspace}}S[table-format=1.3]
% @{\hspace{\tabspace}}S[table-format=1.3]
% @{\hspace{\tabspace}}S[table-format=1.3]
% @{}}
% \toprule
% Method & \multicolumn{6}{c}{DSC per Lesion Load Category} \\
% \addlinespace
%  & {0.0--4.0} & {4.0--7.8} & {7.8--14.7} & {14.7--28.5} & {$>28.5$} &
% {Average} \\
% \midrule
% Global threshold & 0.297805 & 0.545574 & 0.60051 & 0.650483 & 0.679614 &
%  0.543181 \\ 
% Optimal thresholds & 0.351185 & 0.568412 & 0.618115 & 0.667473 & 0.727397 &
% 0.572716 \\
% Predicted thresholds & 0.337834 & 0.550028 & 0.587458 & 0.653475 &
% 0.716976 & 0.555662 \\
% \bottomrule
% \end{tabular}
% \end{table}
% 
% \begin{table}
% \caption{Correlations with biomarkers is comparable to the expert
% segmentations. Threshold methods are: global threshold (LLG), optimal threshold
% (LLO), and predicted threshold (LLP).}
% \label{tab:cross}
% \centering
% \def\tabspace{14pt}
% \begin{tabular}{c%
% @{\hspace{\tabspace}}S[table-format=1.4]%
% @{\hspace{\tabspace}}S[table-format=2.6]%
% @{\hspace{\tabspace}}S[table-format=2.6]%
% @{\hspace{\tabspace}}S[table-format=2.6]%
% @{\hspace{\tabspace}}S[table-format=1.5]%
% @{\hspace{\tabspace}}S[table-format=1.6]%
% }
% \toprule
%      & {T25W} & {9-HPT} & {PASAT} & {MSFC} & {EDSS} & {T2LL} \\
% \midrule
% LLG & 0.0889204058874592 & -0.299046682079096*** & -0.35652944988915*** & -0.414587036427664*** & 0.111910491023784* & 0.867904019287295*** \\
% LLO & 0.0955077933828887* & -0.29002627967845*** & -0.36828512897955*** & -0.41898534832637*** & 0.116889743878327* & 0.988979555746338*** \\
% LLP & 0.0896915607291051 & -0.300773510817976*** & -0.342386015914701*** & -0.411524441008343*** & 0.134054634214833** & 0.869216107742852*** \\
% \addlinespace
% T2LL & 0.0830609699877066 & -0.288809520550953*** & -0.370891003440786*** &
% -0.415459123322053*** & 0.121219714387235* & {---} \\
% \bottomrule
% \end{tabular}
% \end{table}

