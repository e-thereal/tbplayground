%%This is a very basic article template.
%%There is just one section and two subsections.
\documentclass{assign}

\usepackage{lmodern}

\newcommand{\email}[1]{\href{mailto:#1}{\nolinkurl{#1}}}

\title{
  {\normalsize \textsc{Design Document}\\[0.5em]}
  DarkVR: The Virtual Blind Reality
}

\author{Tom Brosch (17362104)}
\date{\email{brosch.tom@gmail.com}}

\begin{document}

\maketitle
%http://wiibrew.org/wiki/Wiimote/Library

\section{Introduction}

\subsection{Purpose}
The purpose of this document is to describe the implementation of the
\verb\DarkVR\ system. The \verb\DarkVR\ software is designed to emulated a
virtual reality relying on non-visual cues.

\subsection{Scope}

This document covers all implementational details of the \verb\DarkVR\
system. In particular, it covers \ldots

%\subsection{Acronyms}

\section{Design Overview}

\subsection{Description of Problem}



\subsection{Technologies Used}

We will use the following technologies:
\begin{itemize}
  \item Head-tracking using webcams
  \item 
  \item Sound cards that allow true 3D audio playback
  \item Headphones
\end{itemize}

\subsection{System Architecture}

\paragraph{Conference Room}
Description of a conference room. All participants are placed on a round table.
Remote participants are replaced by 3D displays. All participants wear
headphones. Microphones located on different locations in the room record the
sounds.

\paragraph{Head Tracking System}

Use head tracking data from the visualization system. (described in the
imaginary visualization document)

\paragraph{Audio System}

Software modules are organized in a pipeline as follows:
\begin{enumerate}
  \item Separate different sound streams $s_i$ using the multiple records $r_j$
  of the same mixed signal.
  \item For each stream $s_i$, calculate the spacial location based on the time
  difference in each recording.
  \item Transmit the separated and localized signals.
  \item Apply location-dependant filters to each audio signal and blend them
  together.
\end{enumerate}

\subsection{System Operation}

The system processes a constant stream of audio input. No event-based user input
is processed.

\section{User Study}

Comparison of standard teleconferencing system (speaker) with 3D audio
conferencing system. Subjects will have two meetings using both technologies.
Subjects will be interviewed after both sessions. The following questions will
be addressed:
\begin{itemize}
  \item Overall sound quality.
  \item Which system makes it easier to understand what someone is saying?
  \item Which system makes it easier to understand who is talking?
  \item Which system makes it easier to concentrate on one participant, when
  more than one participant is talking.
\end{itemize}

\end{document}
