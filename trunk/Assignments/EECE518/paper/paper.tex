% This is "sig-alternate.tex" V1.9 April 2009
% This file should be compiled with V2.4 of "sig-alternate.cls" April 2009
%
% This example file demonstrates the use of the 'sig-alternate.cls'
% V2.4 LaTeX2e document class file. It is for those submitting
% articles to ACM Conference Proceedings WHO DO NOT WISH TO
% STRICTLY ADHERE TO THE SIGS (PUBS-BOARD-ENDORSED) STYLE.
% The 'sig-alternate.cls' file will produce a similar-looking,
% albeit, 'tighter' paper resulting in, invariably, fewer pages.
%
% ----------------------------------------------------------------------------------------------------------------
% This .tex file (and associated .cls V2.4) produces:
%       1) The Permission Statement
%       2) The Conference (location) Info information
%       3) The Copyright Line with ACM data
%       4) NO page numbers
%
% as against the acm_proc_article-sp.cls file which
% DOES NOT produce 1) thru' 3) above.
%
% Using 'sig-alternate.cls' you have control, however, from within
% the source .tex file, over both the CopyrightYear
% (defaulted to 200X) and the ACM Copyright Data
% (defaulted to X-XXXXX-XX-X/XX/XX).
% e.g.
% \CopyrightYear{2007} will cause 2007 to appear in the copyright line.
% \crdata{0-12345-67-8/90/12} will cause 0-12345-67-8/90/12 to appear in the copyright line.
%
% ---------------------------------------------------------------------------------------------------------------
% This .tex source is an example which *does* use
% the .bib file (from which the .bbl file % is produced).
% REMEMBER HOWEVER: After having produced the .bbl file,
% and prior to final submission, you *NEED* to 'insert'
% your .bbl file into your source .tex file so as to provide
% ONE 'self-contained' source file.
%
% ================= IF YOU HAVE QUESTIONS =======================
% Questions regarding the SIGS styles, SIGS policies and
% procedures, Conferences etc. should be sent to
% Adrienne Griscti (griscti@acm.org)
%
% Technical questions _only_ to
% Gerald Murray (murray@hq.acm.org)
% ===============================================================
%
% For tracking purposes - this is V1.9 - April 2009
\documentclass{sig-alternate}

\usepackage{tabularx}
\usepackage{xspace}
\usepackage{hyperref}

\newcommand{\darkvr}{\textsc{DarkVR}\xspace}

\begin{document}
%
% --- Author Metadata here ---
%\conferenceinfo{ }
%\CopyrightYear{ } % Allows default copyright year (20XX) to be over-ridden -
% IF NEED BE.
%\crdata{ }  % Allows default copyright data
% (0-89791-88-6/97/05) to be over-ridden - IF NEED BE.
% --- End of Author Metadata ---

\title{DarkVR: The Virtual Blind Reality}
%\subtitle{[Extended Abstract]
%
% You need the command \numberofauthors to handle the 'placement
% and alignment' of the authors beneath the title.
%
% For aesthetic reasons, we recommend 'three authors at a time'
% i.e. three 'name/affiliation blocks' be placed beneath the title.
%
% NOTE: You are NOT restricted in how many 'rows' of
% "name/affiliations" may appear. We just ask that you restrict
% the number of 'columns' to three.
%
% Because of the available 'opening page real-estate'
% we ask you to refrain from putting more than six authors
% (two rows with three columns) beneath the article title.
% More than six makes the first-page appear very cluttered indeed.
%
% Use the \alignauthor commands to handle the names
% and affiliations for an 'aesthetic maximum' of six authors.
% Add names, affiliations, addresses for
% the seventh etc. author(s) as the argument for the
% \additionalauthors command.
% These 'additional authors' will be output/set for you
% without further effort on your part as the last section in
% the body of your article BEFORE References or any Appendices.

\numberofauthors{1} %  in this sample file, there are a *total*
% of EIGHT authors. SIX appear on the 'first-page' (for formatting
% reasons) and the remaining two appear in the \additionalauthors section.
%
\author{
% You can go ahead and credit any number of authors here,
% e.g. one 'row of three' or two rows (consisting of one row of three
% and a second row of one, two or three).
%
% The command \alignauthor (no curly braces needed) should
% precede each author name, affiliation/snail-mail address and
% e-mail address. Additionally, tag each line of
% affiliation/address with \affaddr, and tag the
% e-mail address with \email.
%
% 1st. author
\alignauthor
Tom Brosch\\
       \affaddr{MS/MRI Research Group}\\
       %\affaddr{1932 Wallamaloo Lane}\\
       \affaddr{Vancouver, BC, Canada}\\
       \email{tombr@msmri.medicine.ubc.ca}
}
% There's nothing stopping you putting the seventh, eighth, etc.
% author on the opening page (as the 'third row') but we ask,
% for aesthetic reasons that you place these 'additional authors'
% in the \additional authors block, viz.
\date{\today}
% Just remember to make sure that the TOTAL number of authors
% is the number that will appear on the first page PLUS the
% number that will appear in the \additionalauthors section.

\maketitle
\begin{abstract}
I present a virtual environment designed to give the blind experience to sighted
people. The aim is the raise awareness for the problems for the problems of
visually impaired in an engaging way. In \darkvr, the user can experience
different virtual worlds relying mostly on hearing. A preliminary user-study
with 6 subjects showed that the approach is promising. 
\end{abstract}

% A category with the (minimum) three required fields
\category{H.4}{Information Systems Applications}{Miscellaneous}
%A category including the fourth, optional field follows...
\category{D.2.8}{Software Engineering}{Metrics}[complexity measures, performance measures]

\terms{Theory}

\keywords{ACM proceedings, \LaTeX, text tagging}

\section{Introduction}

The project aims to create a virtual reality environment without visual cues and
without substituting visual cues by auditory cues (e.g. playing the sound of a
creaking door when you stand right in front of a door). The aim of the project
is, therefore, to create and experience a virtual world as close as possible to
how a blind would experience the real world.


\section{Related Work}
This work is most closely related to the \textsc{Dialogue in the Dark}
project\cite{dialog}. \textsc{Dialogue in the Dark} is an exhibition which lets
non-blind people experience the world of the blinds and, thereby, raises
awareness for the problems blind people are faced everyday while navigate
through the world. Since the created environments are replicas build in an
indoor space, the number of scenarios that are rebuild is very limited. A
virtual reality can help bringing the blindness-experience to scenarios that are
too expensive to rebuild (e.g. navigating through an entire city) or too
dangerous (e.g. due to traffic).

The work is also related to video games for blinds. Video games for blinds can
be roughly classified into two categories: (1) games, which were initially
designed for blinds, and (2) games which have been made accessible to blinds.
Games of the first category often either use a combination of a
narrator, which introduces and describes a new scene, and auditory feedback,
which helps the user to navigate through the virtual environment (e.g. ``Der Tag
wird zur Nacht''\cite{dertag}); or use sound as the primary gaming concept (e.g.
``AudiOdyssey''\cite{audiodyssey}).

There has also been a great amount of research in how to make games accessible
for disabled people (e.g. \cite{chile, terraformers, secondlife,
tankcommander}). See \cite{survey} for a survey of accessible games. A common
method is to replace visuals with audio. This can be done on different levels of
abstraction. Audio cues are the least abstract way of replacing visuals with
audio. Audio cues are sounds that an object would make in the real world (e.g.
footsteps of a walking person). Auditory Icons or ear cons are sounds which are
associated with an object, even though the object would not make this sound in
the current situation (e.g. playing the sound of a creaking door when you stand
in front of it, even if the door is not moving). Sonification is used to
translate physical properties of an object into sound. These sounds are usually
artificial. Properties like pitch or frequency are used to represent other
properties like distance.

In contrast to the many attempts to making a virtual world accessible for the
blinds, this project aims to create an experience which is as closed as
possible to the real experience of blindness. Therefore, accessibility methods
like auditory icons/ear cons and sonifications are not used on purpose. Instead,
the user navigates through the world relying on 3D binaural audio and devices
like a virtual long cane or a virtual GPS localizer. \darkvr, therefore,
does not only give the possibility to experience blindness, it also gives the
possibility to try out imaginary devices that might help blinds in the
future to navigate and orientate in the real world.

\section{Method}

\subsection{Architectural Overview}

Game loop. Some components make changes to the scene graph. Afterwards, the
scenegraph is traversed in order to create the outputs. Show how the scene graph
looks like. Use open scene graph for 3D rendering. Content is provided using
collada 3D models. Sound and material properties are encoded.

\subsection{Game basics}

Controlled with the keyboard. Different settings of the rotation: continuous,
discrete. Movement with the keyboard controls body transformation matrix.

Full body Cylinder. Avoid collision while rotating the body. World is static and
is translated to the bulletin collision shapes right at the beginning of the
program.

Using a magnetic tracker. Calibration step gets initial transformation. Apply
the inverse transformations and then the current transformation and the the
initial transformation again. Tracking independent of the position of the
tracker on the head.

Tracker is attached to something else. No calibration necessary. Reference
point is the head position.

\subsection{Sound System}

Used FMOD for 3D spacial sounds. 3D model is converted to FMODs geometry model
in order to provide realistic sound occlusion characteristics.

Sounds are played randomly or in a loop. Placed in 3D. Characteristics: volume,
minimum distance. (add equation of the decay).

Ground has material properties that determine the sound of the foot steps and
the sound of a hitting cane.

\section{Study Design}

Objective of study was to evaluate if the use of the program can raise awareness
for the problems of visually impaired. A questionnaire was used to asses the
awareness before and after the experiment. In the experiment, the subjects had
to solve several tasks in the environment. Guided by the experimenter.

Give some details about the questions asked. Describe the tasks.
\begin{enumerate}
  \item Make familiar with the controls using vision
  \item Follow a trail in the zoo. Free movement.
  \item Find and identify animals in the zoo. Requires to move around. Not
  necessary to stay on the trail
  \item Apartment scene: Discrete rotations. Guided by spoken instructions like
  move forward, stop, turn to the left. Introduced to the apartment. Afterwards,
  recognize the room.
\end{enumerate}
Performance during these tasks not measured. Purpose of the tasks was to let the
subjects experience the VR.

\section{Results}

Technical questions. Program needs improvement. Difficult to navigate without
force feedback.

Most subjects rated the difficulty after the experiment higher than before. They
found new things. Conclude that the program helped to encourage to think about
the problems in an engaging way. Success.

\section{Discussion and Future Work}

First step towards the goal. Need more realistic scenarios that show-case
problems of visually impaired. Need haptics to facilitate these tasks. The
devices could allow to make the apartment quests more free. Really explore the
apartment. Less guidance means better understand the problems.

\section{Acknowledgments}
Dr. Sidney Fels for guidance and the MAGIC lab for providing magnetic trackers.

%
% The following two commands are all you need in the
% initial runs of your .tex file to
% produce the bibliography for the citations in your paper.
\bibliographystyle{abbrv}
\bibliography{sigproc}  % sigproc.bib is the name of the Bibliography in this case
\end{document}
